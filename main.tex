


\documentclass[10pt]{book}
\usepackage[utf8]{inputenc}
\usepackage[spanish]{babel}
\usepackage{geometry}
\usepackage{multicol}

\usepackage{enumitem}


\usepackage{xcolor}
\usepackage{graphics}

\pagecolor[HTML]{1e1e1e}
%\pagecolor[rgb]{0,0,0}
% \color[HTML]{ced5e0}% fandango
\color{gray!50}% fandango





\usepackage{tikz}
\usetikzlibrary{3d} % Biblioteca para gráficos en 3D
\usetikzlibrary{shapes.geometric}
\usetikzlibrary{patterns}

% \usepackage{pgfplots}
% \pgfplotsset{compat=1.18} % Asegura compatibilidad con PGFPlots reciente

\geometry{a4paper, margin=1in}


\usepackage[%
  breaklinks=true,
  colorlinks=false,
  % colorlinks=true,
  % linkcolor=color-link,
  % urlcolor=color-url,
  % hyperfootnotes=true
  citecolor=black,
  filecolor=magenta,
  hidelinks
]{hyperref}



\usepackage{array}
\usepackage{amsmath,amssymb,amsfonts,amsthm}
\decimalpoint






% Énfasis fuerte (en negrita)
\newcommand{\semph}[1]{\textbf{#1}}








% Columnas personalizadas de entorno array. Se requiere del paquete array.

% Columnas de matemáticas
\newcolumntype{L}{>{\displaystyle}l}
\newcolumntype{C}{>{\displaystyle}c}

% Columna de texto normal
% \newcolumntype{T}{>{\raggedright\arraybackslash}p{0.5\textwidth}}
\newcolumntype{T}{>{\raggedleft\arraybackslash}p{0.5\textwidth}}







% Entornos
% ----------------------------------------------------------------
\newtheorem{theorem}{Teorema}[section]
\newtheorem{corollary}{Corolario}[theorem]
\newtheorem{lemma}[theorem]{Lema}
\newtheorem{proposition}[theorem]{Proposición}

\newtheorem{deffinition}{Definición}[section]

\newtheorem{properties}{Propiedades}[section]

\newtheorem{exercise}[theorem]{Ejercicio}
\newtheorem{example}[theorem]{Ejemplo}




% Conjuntos numéricos
\newcommand{\nset}{\mathbb{N}}
\newcommand{\zset}{\mathbb{Z}}
\newcommand{\qset}{\mathbb{Q}}
\newcommand{\rset}{\mathbb{R}}
\newcommand{\cset}{\mathbb{C}}


\newcommand{\st}{\ :\ } % tal que... (such that)
\newcommand{\imset}[1]{\text{Im}(#1)} % Conjunto imagen de una función/aplicación. Im()


\newcommand{\card}[1]{\mathrm{card}{#1}} % cardinal de un conjunto
\newcommand{\powset}[1]{\mathcal{P}{#1}} % conjunto de las partes de un conjunto



\newcommand{\contrad}{\textbf{0}}
\newcommand{\tautol}{\textbf{1}}




% Relaciones
\newcommand{\rrel}{\mathcal{R}}
\newcommand{\srel}{\mathcal{S}}
\newcommand{\grel}{\mathcal{G}}
\newcommand{\erel}{\mathcal{E}}



\newcommand{\mmax}[1]{\text{máx}{#1}} % máximo de un conjunto
\newcommand{\mmin}[1]{\text{mín}{#1}} % mínimo de un conjunto
\newcommand{\msup}[1]{\text{sup}{#1}} % supremo de un conjunto
\newcommand{\minf}[1]{\text{ínf}{#1}} % ínfimo de un conjunto


\newcommand{\mmaxL}[1]{\text{máx}_L{#1}}
\newcommand{\mminL}[1]{\text{mín}_L{#1}}
\newcommand{\mmaxP}[1]{\text{máx}_P{#1}}
\newcommand{\mminP}[1]{\text{mín}_P{#1}}
\newcommand{\msupL}[1]{\text{sup}_L{#1}}
\newcommand{\minfL}[1]{\text{ínf}_L{#1}}
\newcommand{\msupP}[1]{\text{sup}_P{#1}}
\newcommand{\minfP}[1]{\text{ínf}_P{#1}}








\title{Apuntes de Lenguaje Matemático, Conjuntos y Números}

\author{Carlos E.~Tafur Egido}
\date{2025-01-08}




\begin{document}





\maketitle
\tableofcontents







\chapter{Lógica de proposiciones}%
\label{ch:logica-prop}


  \section{Introducción a la lógica}
  \input{./fuentes/logica-proposiciones/introd-logica.tex}


  \section{Primeros pasos}
  \input{./fuentes/logica-proposiciones/primeros-pasos.tex}


  \section{Operadores básicos}
  \input{./fuentes/logica-proposiciones/operadores-logicos.tex}


  \section{Construcción de proposiciones}
  \input{./fuentes/logica-proposiciones/constr-nuevas-prop.tex}


  \section{Reglas de la lógica}
  \input{./fuentes/logica-proposiciones/reglas-logica.tex}


  \section{Validación de proposiciones}
  \input{./fuentes/logica-proposiciones/validacion-prop.tex}


  \section{Forma clausulada de proposiciones}
  \input{./fuentes/logica-proposiciones/forma-clausulada.tex}


  \section{Comentarios}

    \subsection{Sistemas formales}
    \input{./fuentes/logica-proposiciones/sistemas-formales.tex}

    \subsection{Sistema axiomático de Whitehead y Russell}
    \input{./fuentes/logica-proposiciones/comentarios/sist-axiomatico-russell.tex}

    \subsection{Presentación de resultados en matemáticas}
    \input{./fuentes/logica-proposiciones/comentarios/prop-teoremas.tex}

    \subsection{Métodos de demostración}
    \input{./fuentes/logica-proposiciones/comentarios/metodos-demostracion.tex}








\chapter{Lógica de predicados y teoría de conjuntos}%
\label{ch:conjuntos}


  \section{Introducción}
  



Las eucaciones diofánticas (\emph{Diophantine equations}) son TKTK.











  \section{Conjuntos}
  \input{./fuentes/conjuntos/ideas-conjuntos-pred.tex}


  \section{Predicados}
  \input{./fuentes/conjuntos/ideas-conjuntos-pred/predicados.tex}


  \section{Cuantificadores}
  \input{./fuentes/conjuntos/ideas-conjuntos-pred/cuantificadores.tex}


  \section{Complemento y partes}
  \input{./fuentes/conjuntos/ideas-conjuntos-pred/compl-partes.tex}


  \section{Operaciones}

    \subsection{Unión}
    \input{./fuentes/conjuntos/operaciones/union.tex}

    \subsection{Intersección}
    \input{./fuentes/conjuntos/operaciones/interseccion.tex}


  \section{Álgebra de conjuntos}
  \input{./fuentes/conjuntos/algebra.tex}


  \section{Producto}
  \input{./fuentes/conjuntos/producto-cartesiano.tex}


  \section{Comentarios}
  \input{./fuentes/conjuntos/comentarios.tex}

    \subsection{Sobre el método de inducción}
    \input{./fuentes/conjuntos/comentarios/metodo-induccion.tex}

    \subsection{Sobre la teoría de conjuntos}
    \input{./fuentes/conjuntos/comentarios/teoria-conjuntos.tex}



\chapter{Relaciones}%
\label{ch:relaciones}
\input{./fuentes/relaciones.tex}


  \section{Introducción}
  \input{./fuentes/relaciones/introd.tex}


  \section{Propiedades básicas de una relación}
  \input{./fuentes/relaciones/propiedades-basicas-rel.tex}


  \section{Relación de equivalencia}
  \input{./fuentes/relaciones/rel-equivalencia.tex}


  \section{Relación de orden}
  \input{./fuentes/relaciones/rel-orden.tex}




\chapter{Aplicaciones}%
\label{ch:aplicaciones}


  \section{Introducción}
  \input{./fuentes/aplicaciones/introd.tex}


  \section{Equipotencia de conjuntos}\label{sec:equipotencia-conjuntos}
  \input{./fuentes/aplicaciones/equipotencia.tex}





\chapter{Operaciones internas y estructuras algebraicas}%
\label{ch:op_internas}


  \section{Introducción}
  



Las eucaciones diofánticas (\emph{Diophantine equations}) son TKTK.











  \section{Operaciones internas}
  \input{./fuentes/op-internas-estructuras/op-internas.tex}

    \subsection{Propiedades}
    \input{./fuentes/op-internas-estructuras/op-internas/propiedades.tex}


  \section{Grupos}
  \input{./fuentes/op-internas-estructuras/grupos.tex}


  \section{Anillos}
  \input{./fuentes/op-internas-estructuras/anillos.tex}


  \section{Cuerpos}
  \input{./fuentes/op-internas-estructuras/cuerpos.tex}


  \section{Orden y operaciones}
  \input{./fuentes/op-internas-estructuras/orden-operaciones.tex}


  \section{Homomorfismos}
  \input{./fuentes/op-internas-estructuras/homomorfismos.tex}





\chapter{Números naturales}%
\label{ch:naturales}


  \section{Introducción}
  



Las eucaciones diofánticas (\emph{Diophantine equations}) son TKTK.











  \section{Los números naturales}
  \input{./fuentes/naturales/numeros-naturales.tex}


  \section{Conjuntos finitos}
  \input{./fuentes/naturales/conjuntos-finitos.tex}


  \section{Conjuntos infinitos}
  \input{./fuentes/naturales/conjuntos-infinitos.tex}





\chapter{Números enteros}%
\label{ch:enteros}








\input{./bibliografia.tex}



\end{document}



