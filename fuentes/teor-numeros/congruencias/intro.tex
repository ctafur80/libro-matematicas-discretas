


Viene en \cite{burton} capítulo 4, pág. 61.

Las relaciones de congruencia son de gran utilidad para estudiar la
divisibilidad en $\zset$. En el fondo, es lo mismo que la aritmética
modular.

\begin{deffinition}
  Dados $a, b \in \zset$ y $m \in \nset^{+}$. Se dice que $a$ y $b$ son
  \m{congruentes} (\engm{congruents}) módulo $m$, y se escribe $a \equiv b
  \pmod m$, si y solo si $m \mid (a - b)$. En caso contrario, se dice que
  son \m{incongruentes} (\engm{incongruents}) módulo $m$.
\end{deffinition}

En este caso, al número $m$ se le llama \m{módulo} (\engm{module}) de la
congruencia.

Evidentemente, dos números cualesquiera son congruentes módulo 1, aunque la
congruencia módulo 1 en realidad no tiene ninguna relevancia ni es práctica
para ningún fin. Dos números serán congruentes módulo 2 cuando tienen la
misma paridad, es decir, si ambos son pares o bien ambos son impares.

Si hacemos la división en $\zset$ con resto de un número $a \in \zset$ entre
un número $m \in \nset^{+}$, tenemos que existen $q, r \in \zset$ con $0
\leq r < m$ para los que

$$ a = mq + r $$

\noindent De aquí, reordenando términos, tenemos $a - r = mq$, con lo que $a
\equiv r \pmod m$. Puesto que la variable $r$ puede tomar los valores
enteros entre $0 \leq r < m$, podemos afirmar que, en general, cualquier $a
\in \zset$ es congruente módulo $m$ con uno y solo uno de los valores $0, 1,
2, 3, \ldots, m-1$, cosa que se justifica por el mismo Teorema que justifica
la existencia de la división en $\zset$ con resto. En particular, se da que
$a \equiv 0 \pmod m$ si y solo si $m \mid a$.

A ese $r$ entre 0 y $m-1$ se le conoce como \m{menor residuo no megativo}
módulo $m$ (\engm{least nonnegative residue module $m$}), y al conjunto de
los $m$ enteros $0, 1, 2, 3, \ldots, m-1$, los \m{menores residuos no
negativos} módulo $m$ (\engm{least nonnegative residues modulo $m$}).

Por lo general, una colección de enteros $a_1, a_2, a_3, \ldots, a_m$ se
dice que es un \m{conjunto (o sistema) completo de residuos} módulo $m$ si
todo entero es congruente módulo $m$ a uno y solo uno de los $a_i$, siendo
$1 \leq i \leq m$. Dicho de otro modo, los elementos de $a_1, a_2, a_3,
\ldots, a_m$ son congruentes respectivamente con los elementos de alguna
permutación de los menores residuos no negativos módulo $m$, es decir, de
$0, 1, 2, 3, \ldots, m-1$.

También se podría decir que un conjunto de números enteros no forman un
conjunto completo de residuos módulo $m$ si y solo si no hay dos de estos
que sean congruentes módulo $m$.

El teorema siguiente proporciona una caracterización útil de la congruencia
módulo $m$ en términos del resto de la división por $m$.

\begin{theorem}[de la Caracterización de la Congruencia]
  Dados $a, b \in \zset$ y $m \in \nset^{+}$. Entonces, $a \equiv b \pmod m$
  si y solo si al dividir tanto a $a$ como a $b$ por $m$, se obtiene el
  mismo resto.
\end{theorem}

\begin{proof}
  Por un lado, puesto que $m \mid (a - b)$, tenemos que existirá algún $k
  \in \zset$ para el que $a = b + km$. Podemos hacer la división entera con
  resto de $b$ entre $m$ y tendremos

  $$ b = qm + r $$

  \noindent para algún par de $q, r \in \zset$ con $0 \leq r < m$. Uniendo
  ambas ecuaciones,

  $$ a = b + km = (qm + r) + km = (q + k)m + r $$

  \noindent con lo que $a$ y $b$ tienen el mismo resto, $r$.

  Por otro lado, suponga que podemos escribir $a = q_1 m + r$ y $b = q_2 m +
  r$, para $q_1, q_2, r \in \zset$, siendo, siendo $0 \leq r < m$. Entonces,

  $$ a - b = (q_1 m + r) - (q_2 m + r) = (q_1 - q_2)m $$

  \noindent por lo que $m \mid (a - b)$.
\end{proof}

La congruencia se puede ver como una forma generalizada de la igualdad, en
el sentido de que su comportamiento con respecto a la suma y la
multiplicación es reminiscente a la igualdad ordinaria. En el teorema
siguiente se muestran algunas de las propiedades elementales de la igualdad
que se trasladan a las congruencias.

\begin{theorem}
  Dados $a, b, c, d \in \zset$ y $m \in \nset^{+}$. Se tiene

  \begin{enumerate}
    \item $a \equiv a \pmod m$. Propiedad reflexiva.
    \item Si $a \equiv b \pmod m$, entonces $b \equiv a \pmod m$. Propiedad
      simétrica.
    \item Si $a \equiv b \pmod m$ y $b \equiv c \pmod m$, entonces $a \equiv
      c \pmod m$. Propiedad transitiva.
    \item Si $a \equiv b \pmod m$ y $c \equiv d \pmod m$, entonces $a+c
      \equiv b+d \pmod m$ y $ac \equiv bd \pmod m$.
    \item Si $a \equiv b \pmod m$, entonces $a+c \equiv b+c \pmod m$ y $ac
      \equiv bc \pmod m$.
    \item Dado $k \in \nset^{+}$. Si $a \equiv b \pmod m$, entonces $a^k
      \equiv b^k \pmod m$.
  \end{enumerate}
\end{theorem}

Estas propiedades son muy útiles para operar con las congruencias, como
veremos luego, con lo que conviene aprendérselas de memoria. Ahora, veamos
sus demostraciones.

\begin{proof}
  -

  \begin{enumerate}
    \item Para cualquier $a \in \zset$, tenemos que $a - a = 0\cdot m$.

    \item Si $a \equiv b \pmod m$, entonces existe un $k \in \zset$ para el
      que $a - b = km$. Por tanto, $b - a = {-(km)} = ({-k})m$ y, como ${-k}
      \in \zset$, se tiene que $b \equiv a \pmod m$.

    \item Suponga que $a \equiv b \pmod m$ y $b \equiv c \pmod m$. Entonces,
      existen $h, k \in \zset$ para los que se cumple

      \begin{alignat*}{2}
        a - b &= hm \\
        b - c &= km
      \end{alignat*}

      \noindent Sumándolas, tenemos

      \begin{alignat*}{2}
        (a - b) + (b - c)   &= hm + km \\
        a - c               &= (h + k)m \\
      \end{alignat*}

      \noindent que es lo mismo que $a \equiv c \pmod m$.

    \item Del mismo modo que antes, si $a \equiv b \pmod m$ y $c \equiv d
      \pmod m$, entonces existirán $k_1, k_2 \in \zset$ para los que

      \begin{alignat*}{2}
        a - b   &= k_1 m \\
        c - d   &= k_2 m
      \end{alignat*}

      \noindent Sumándolas, tenemos

      \begin{alignat*}{2}
        (a - b) + (c - d)   &= k_1 m + k_2 m \\
        (a + c) - (b + d)   &= (k_1 + k_2)m
      \end{alignat*}

      \noindent que es lo mismo que $a + c \equiv b + d \pmod m$.

      Para la segunda afirmación, se tiene que

      $$ ac = (b + k_1 m)(d + k_2 m) = bd + (bk_2 + dk_1 + k_1 k_2 m)m $$

      \noindent que es lo mismo que decir que $ac \equiv bd \pmod m$.

    \item Se demuestra por la anterior teniendo en cuenta que $c \equiv c
      \pmod m$, para todo $c \in \zset$.

    \item Esta la demostraremos por inducción. De la propiedad 4, sabemos
      que si se diese la hipótesis de inducción, es decir, $a \equiv b \pmod
      m$ y $a^k \equiv b^k \pmod m$, se daría que $a \, a^k \equiv b \, b^k
      \pmod m$, que sería lo mismo que $a^{k+1} \equiv b^{k+1} \pmod m$, que
      sería la meta de inducción.

      % Pregunta. No habría que ver si se da el caso base?

  \end{enumerate}
\end{proof}

En el libro aparece otra. Como la 5.ta pero con división en lugar de
multiplicación.

Advierta que, concretamente, se cumplen las propiedades reflexiva, simétrica
y transitiva. Por lo tanto, la congruencia es una relación de equivalencia.

Tal y como hemos visto, si $a \equiv b \pmod m$, entonces se dará que $ac
\equiv bc \pmod m$, para todo $c \in \zset$. Sin embargo, lo contrario no
tiene por qué darse. Bastaría con dar un contraejemplo; por ejemplo, se da
$2 \cdot 4 \equiv 2 \cdot 1 \pmod 6$, pero $4 \not\equiv 1 \pmod 6$. Lo que
sí podemos afirmar a este respecto es lo que explica el teorema siguiente.

\begin{theorem}
  Dados $a, b, c \in \zset$ y $m \in \nset^{+}$. Si $ca \equiv cb \pmod m$,
  entonces $a \equiv b \pmod {m/d}$, siendo $d = \gcd(c, m)$.
\end{theorem}

\begin{proof}
  Por hipótesis, podemos escribir

  $$ c(a - b) = ca - cb = km $$

  \noindent para algún $k \in \zset$. Sabiendo que $d = \gcd(c, m)$,
  entonces, por el Corolario \ref{cor-mcd-div-mcd}, existirán $r, s \in
  \zset$ primos relativos para los que $c = dr$ y $m = ds$. Cuando
  sustituimos estos valores en la ecuación anterior y cancelamos el factor
  común $d$, tenemos

  $$ r(a - b) = ks $$

  \noindent Por tanto, $s \mid r(a - b)$ y $\gcd(r, s) = 1$. Por el Lema
  \ref{th-lema-euclides}, tenemos $s \mid (a - b)$, que es lo mismo que
  decir que $a \equiv b \pmod s$, o sea, $a \equiv b \pmod {m/d}$.
\end{proof}

\begin{corollary}
  Dados $a, b, c \in \zset$ y $m \in \nset^{+}$. Si $ca \equiv cb \pmod m$ y
  $\gcd(c, m) = 1$, entonces $a \equiv b \pmod m$.
\end{corollary}

\begin{corollary}
  Dados $a, b, c \in \zset$ y $p$ un número primo. Si $ca \equiv cb \pmod p$
  y $p \nmid c$, entonces $a \equiv b \pmod p$.
\end{corollary}

\begin{proof}
  La condición de que $p$ sea primo y $p \nmid c$ hacen que $\gcd(c, p) =
  1$, y ya sería el mismo caso que el del corolario anterior.
\end{proof}

Algo curioso con las congruencias, y que contrasta con el comportamiento de
la igualdad, es que se puede tener 0 como resultado multiplicando dos
números que no son congruentes con 0. Así, por ejemplo, se tiene que $4
\cdot 3 \equiv 0 \pmod {12}$ siendo $4 \not\equiv 0 \pmod {12}$ y $3
\not\equiv 0 \pmod {12}$.






\section{Representación binaria y decimal de números enteros}
