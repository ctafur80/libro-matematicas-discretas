



Una referencia en la que está explicado todo lo concerniente al Principio de
Inducción, en gran detalle, es \cite{idescent-newstead}.

Se podría decir, metafóricamente, que es como el efecto dominó.

El uso del Principio de Inducción es muy potente a la hora de hacer
demostraciones. Casi parece que estemos haciendo ``trampa'', pues suelen
resultar muy sencillas y mecánicas de realizar. El principal prroblema con
el que cuenta es que no nos guía para obtener conocimiento nuevo. Es decir,
es simplemente para comprobar si es cierto algo que sospechamos, pero la
intuición de que eso pueda ser cierto la debemos obtener por otros medios.
Es común ir poniendo ejemplos de casos concretos y, a simple vista, tratar
de intuir si existe cierta relación entre los resultados obtenidos.



En la tutoría, el profesor hizo la demostración del Teorema Fundamental de
la Aritmética mediante el Principio de Inducción Fuerte, cosa que se puede
consultar también en \cite{proofs-cummings} pág. 166. Dicho teorema se
demostró, cuando se presentó, en el capítulo \ref{ch-numeros-primos}, de
otra forma que resulta más complicada.




\begin{theorem}[Principio de Inducción]\label{princ-induccion}
  El conjunto $S = \{n \in \nset \st P_1 \ \text{y} \ P_2\}$ siendo

  \begin{center}
    \begin{tabular}{l p{0.8\textwidth}}
      $P_1$ & $0 \in S$. \\
      $P_2$ & Para todo $k \in \nset$ se da que, si $k \in S$, entonces
        $(k+1) \in S$.
    \end{tabular}
  \end{center}

  \noindent es igual a $\nset$.
\end{theorem}

\begin{proof}
  La demostración la haremos por contradicción.

  Supongamos que existe un subconjunto propio de $\nset$ que satisface $P_1$
  y $P_2$. A este lo designaremos por $S$. Sea $S'$ su complementario
  respecto a $\nset$, es decir, $S' = \nset \setminus S$. Ya que $S \neq
  \nset$, puesto que era un subconjunto propio, se tiene que $S' \neq
  \emptyset$. Entonces, por la Propiedad \ref{princ-buena-ord}, se tiene que
  $S'$ tiene un primer elemento. Decidimos designar a este como $a$.

  Ya que, por $P_1$, se tiene que $0 \in S$, entonces necesariamente se da
  que $a \geq 1$, con lo que $a - 1 \geq 0$. Como $a$ es el elemento menor
  de $S'$ y $a - 1 < a$, se tiene que $(a - 1) \notin S'$, o, lo que es lo
  mismo, $(a - 1) \in S$.

  Por otro lado, si se aplica $P_2$ a $a - 1$, se tendrá que, si $(a - 1)
  \in S$, cosa que hemos demostrado que es cierta, entonces $(a - 1) + 1$,
  es decir, $a$, pertenece a $S$.

  Esto último se contradice con lo que dijimos al definir al elemento $a$.
  Por tanto, la hipótesis de partida solo puede ser falsa. Esta era que
  existe un subconjunto propio de $\nset$ que satisface $P_1$ y $P_2$. Lo
  contrario será que $S = \nset$, tal y como queríamos demostrar.
\end{proof}

De este teorema, se deduce que esto no es solo válido para conjuntos, sino
que sirve para propiedades (predicados). Hay quien lo presenta como un
corolario de este, y hay quien no lo hace. TKTK.

Hay varias partes de las demostraciones con este método que conviene tener
claras. Son el caso base, la hipótesis de inducción, la meta de indicción y
el paso inductivo. Vienen explicadas en \cite{idescent-newstead}.


En realidad, aunque el caso base que hayamos usado sea el 0, cosa que hemos
hecho porque para nosotros los números naturales comienzan en dicho número,
hay quien considera al 1 como caso base. Da igual, y, de hecho, tal y como
explicamos a continuación, se puede elegir a cualquier número entero para el
caso base.

\begin{corollary}[Principio de Inducción con Caso Base Genérico]
  Dado $b \in \zset$. El conjunto $S = \{n \in \zset \st P_1 \ \mathrm{y} \
  P_2\}$ siendo

  \begin{center}
    \begin{tabular}{l p{0.8\textwidth}}
      $P_1$ & $b \in S$. \\
      $P_2$ & Para todo $k \in \zset$ con $k \geq b$ se da que, si $k \in
        S$, entonces $(k+1) \in S$.
    \end{tabular}
  \end{center}

  \noindent es igual a $\{n \in \zset \st n \geq b\}$
\end{corollary}

\begin{proof}
  Para esta demostración, vamos a auxiliarnos de un conjunto $T$ que tenga
  una biyección con $S$ pero, al contrario que este, sobre aquel sí se podrá
  realizar el Principio de Inducción con caso base 0, que es el único que
  conocemos hasta ahora. Debido a esto, sus elementos deben comenzar en el 0
  y ser todos los enteros mayores o iguales a este.

  % ¿Hay que tener en cuenta también que T y S tienen una biyección? TKTK.

  Aunque aún no lo hemos demostrado, la intención es que $S$ esté
  constituido por todos los enteros mayores o iguales que $b$. Por tanto,
  podríamos establecer una fórmula para transformar los elementos de $S$ en
  los de $T$. Así, tendríamos que los elementos de $T$, que designaremos por
  la variable $n$, tendrán la forma $n = r - b$, siendo $r$ los distintos
  valores de $S$. Así, se tendrá que los elementos de $T$ comienzan por el
  0, que es el caso base sobre el que aplicaremos el Principio de Inducción
  que conocemos hasta ahora. Por tanto, $r$ será una función sobre $n$, cosa
  que designaremos por $r(n)$. La relación entre $n$ y $r$ se podría
  expresar también de la forma $r = n + b$, que será la que usaremos
  normalmente.

  Dada esta explicación, es evidente que el conjunto $T$ será

  $$ T = \{n \in \zset \st n + b = r \in S\} $$

  Vamos a analizar el caso base. Por un lado, el 0 es un elemento de $T$,
  pues, para $n = 0$, se tiene $r(0) = 0 + b = b$, y ese valor, $b$, sabemos
  por $P_1$ que pertenece a $S$.

  Ahora, vamos a analizar el paso inductivo. Supongamos que se da la
  hipótesis, es decir, que para un $k \in \zset$ se da que $k \in T$. Por la
  definición de $T$, se da necesariamente que, si $k \in T$, entonces $k + b
  \in S$, o, si lo prefiere, $r(k) \in S$.

  Si $k + b \in S$, entonces, aplicando $P_2$, se tendrá que $r(k) + 1 = (k
  + b) + 1 \in S$. Pero $(k + b) + 1 = (k + 1) + b$, con lo que $r + 1 \in
  T$.

  Ahora, aplicaremos a $T$ el Principio de Inducción con caso base 0. Por un
  lado, como vimos al principio, el caso base, $k = 0$, se cumple. Por el
  otro, el paso inductivo, tal y como acabamos de demostrar, también se
  cumple. Por tanto, se tiene que $T = \nset$.

  Ahora, veamos la definición de $S$ en base a $T$. Sería

  $$ S = \{r \in \zset \st r - b = n \in T\} $$

  Entonces, puesto que $T = \nset$, se puede decir que

  $$ S = \{r \in \zset \st r \geq b\} $$

  % Habría que hablar de los casos en los que b < 0. TKTK.
\end{proof}

Al igual que para el caso base 0, existe también un principio equivalente a
este último pero para propiedades.



Con estas dos proposiciones, que para muchos son en realidad una sola,
tenemos ya herramientas para dos propósitos principales: por un lado,
realizar demostraciones, pero, también, hacer definiciones que contengan
alguna clase de autorreferencia. Lo normal es que, para el caso de las
demostraciones, se use el adjetivo \m{inductiva} o se diga \m{por
inducción}, pero, para las definiciones, se opte por calificarlas de
\m{recurrentes} o \m{por recursión}. En realidad, se trata de lo mismo, y
hay quien opta por decir \m{por inducción} para los dos tipos, como se
explica en \cite{stillwell-elem-maths} pág. 308.





\begin{theorem}[Principio de Inducción Fuerte]
  El conjunto $S = \{n \in \nset \st P_1 \ \text{y} \ P_2\}$ siendo

  \begin{center}
    \begin{tabular}{l p{0.8\textwidth}}
      $P_1$ & $0 \in S$. \\
      $P_2$ & Para todo $n \in \nset$ se da que, si para todo $k \in \nset$
        con $0 \leq k < n$ se tiene que $k \in S$, entonces $n \in S$.
    \end{tabular}
  \end{center}

  \noindent es igual a $\nset$.
\end{theorem}

Al Principio de Inducción Fuerte también hay quien lo llama Principio de
Inducción Completa.

\begin{proof}
  Esta demostración es análoga a la del teorema \ref{princ-induccion}. Se
  hará también por contradicción.

  Supongamos que existe un subconjunto propio de $\nset$ que satisface $P_1$
  y $P_2$. A este lo designaremos por $S$. Sea $S'$ su complementario
  respecto a $\nset$, es decir, $S' = \nset \setminus S$. Ya que $S \neq
  \nset$, puesto que era un subconjunto propio, se tiene que $S' \neq
  \emptyset$. Entonces, por la Propiedad \ref{princ-buena-ord}, se tiene que
  $S'$ tiene un primer elemento. Decidimos designar a este como $a$.

  Ya que, por $P_1$, se tiene que $0 \in S$, entonces necesariamente se da
  que $a \geq 1$. Todos los números entre 0 y $a - 1$, es decir, $0, 1, 2,
  3, \ldots, a - 1$, pertenecen a $S$, puesto que $a$ es el elemento menor
  de $S'$.

  Por otro lado, si se aplica $P_2$ a $0, 1, 2, 3, \ldots, a - 1$, se tendrá
  que, si todos ellos pertenecen a $S$, cosa que hemos demostrado que es
  cierta, entonces $a$ pertenece a $S$.

  Esto último se contradice con lo que dijimos al definir al elemento $a$.
  Por tanto, la hipótesis de partida solo puede ser falsa. Esta era que
  existe un subconjunto propio de $\nset$ que satisface $P_1$ y $P_2$. Lo
  contrario será que $S = \nset$, tal y como queríamos demostrar.
\end{proof}

Tendría también su versión genérica.

\begin{theorem}[Principio de Inducción Fuerte con Caso Base Genérico]
  Dado $b \in \zset$. El conjunto $S = \{n \in \nset \st P_1 \ \text{y} \
  P_2\}$ siendo

  \begin{center}
    \begin{tabular}{l p{0.8\textwidth}}
      $P_1$ & $b \in S$. \\
      $P_2$ & Para todo $n \in \nset$ se da que, si para todo $k \in \nset$
        con $b \leq k < n$ se tiene que $k \in S$, entonces $n \in S$.
    \end{tabular}
  \end{center}

  \noindent es igual a $\{n \in \zset \st n \geq b\}$
\end{theorem}













