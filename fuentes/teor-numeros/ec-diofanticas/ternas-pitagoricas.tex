



Viene bastante bien explicado en \cite{burton} sección 12.1
pág. 245, así como en \cite{rosen} sección 13.1 pág. 522.
Prefiero la explicación del último.

Ahora, vamos a considerar la ecuación $x^2 + y^2 = z^2$ con $x, y, z \in
\zset^{+}$. Como puede imaginar, las soluciones serían, desde el punto de
vista de la geometría, todos los triángulos rectángulos con lados de
longitud entera (y, evidentemente, positiva); cosa que es evidente por el
Teorema de Pitágoras. Este problema lo estudiaron los pitagóricos, y estos
encontraron una cantidad infinita de soluciones, pero fue Euclides quien dio
la solución general al problema, en su libro \emph{Elementos}.

Primero, advierta que, si $(x_0, y_0, z_0)$ es una solución, cosa que se
suele designar por \emph{terna pitagórica} (\engm{Pythagorean triple}), también
lo serán $(\lambda x_0, \lambda y_0, \lambda z_0)$ para todos los $\lambda
\in \zset$ con $\lambda \neq 0$, ya que

$$ \lambda^2 x_0^2 + \lambda^2 y_0^2 = \lambda^2(x_0 + y_0) = \lambda^2 z_0
$$

Por otro lado, si $d = \gcd(x_0, y_0, z_0)$, entonces

$$ \left( \frac{x_0}{d}, \frac{y_0}{d}, \frac{z_0}{d} \right) $$

\noindent también será una solución, ya que esas tres divisiones tienen que
ser exactas, necesariamente, es decir, tendrán resto 0, al darse $d \mid
x_0$, $d \mid y_0$ y $d \mid z_0$. Además, se da que $x_0/d$, $y_0/d$ y
$z_0/d$ son primos relativos, por el Corolario \ref{cor-mcd-div-mcd}. A una
solución con esta particularidad, es decir, en la que los tres términos son
primos relativos, recibe el nombre de \emph{terna pitagórica primitiva}
(\engm{primitive Pythagorean triple}), y, como veremos en el teorema
siguiente, basta con encontrar esta solución para obtener una forma general
para todas las soluciones del problema. Es decir, para resolver el problema
bastará con encontrar las ternas pitagóricas primitivas, $x'_0, y'_0, z'_0
\in \zset^{+}$, y las demás soluciones vendrán dadas al multiplicarlas por
un parámetro $\lambda \in \zset$ con $\lambda \neq 0$, es decir, $(\lambda
x'_0, \lambda y'_0, \lambda z'_0)$. Es decir, construiremos la solución
general a partir de las ternas pitagóricas primitivas.

Antes de llegar a dar la solución general, vamos a dar algunos lemas que se
usarán en su demostración.

El lema siguiente explica que los elementos de una terna pitagórica
primitiva son primos relativos también dos a dos.

\begin{lemma}[de la Primalidad en las Ternas Pitagóricas
  Primitivas]\label{lema-terna-pit-prim-dos-a-dos}
  Si $(x, y, z)$ es una terna pitagórica primitiva, entonces sus componentes
  son primos relativos dos a dos.
\end{lemma}

\begin{proof}
  Suponga que $(x, y, z)$ es una terna pitagórica primitiva y $\gcd(x, y) >
  1$, pues si valiese 1 ya estaría demostrado. Debido a esta última
  condición, se podrá descomponer a $\gcd(x, y)$ en factores primos. Nos
  quedamos con uno cualquiera de ellos, $p$. Se cumple entonces que $p \mid
  \gcd(x, y)$. Por tanto, $p \mid x$ y $p \mid y$. Debido a esto, por las
  proposiciones \ref{prop-divide-al-multiplo} y \ref{princ-dos-de-tres}, se
  tiene que $p \mid (x^2 + y^2)$, o, lo que es lo mismo, $p \mid z^2$.
  También, como $p \mid z^2$ siendo $p$ primo, aplicando el Corolario
  \ref{cor-lema-euclides-num-primos} se tiene que $p \mid z$. Uniendo esto,
  se tendría que $\gcd(x, y, z) = p > 1$, lo cual se contradice con la
  condición de que esa terna pitagórica sea primitiva, con lo que, por
  contradicción, se puede afirmar que $\gcd(x, y) = 1$.

  Este mismo procedimiento se puede usar para los otros pares de elementos
  de la terna, es decir, se da que $\gcd(x, y) = \gcd(x, z) = \gcd(y, z) =
  1$.
\end{proof}

\begin{lemma}[de la Paridad de Ternas Pitagóricas
  Primitivas]\label{lema-ternas-pit-prim-paridad}
  Dada la terna pitagórica primitiva $(x, y, z)$. Se tiene que $x$ e $y$
  tienen distinta paridad.
\end{lemma}

\begin{proof}
  Analizaremos los distintos casos y veremos que no pueden tener la misma
  paridad.

  Si $x$ e $y$ son pares, se tendrán $m, n \in \zset$ tales que $x = 2m$ e
  $y = 2n$. Entonces,

  $$ x^2 + y^2 = (2m)^2 + (2n)^2 = 4m^2 + 4n^2 = 2(2m^2 + 2n^2) $$

  \noindent con lo que $2 \mid (x^2 + z^2)$, que es lo mismo que $2 \mid
  z^2$. Tenemos que $2 \mid z^2$ y que 2 es primo, si aplicamos el Corolario
  \ref{cor-lema-euclides-num-primos} tenemos que entonces $2 \mid z$. De
  esto se deduce que $\gcd(x, y, z) \geq 2$, cosa que se contradice con la
  hipótesis de que se trate de ternas pitagóricas primitivas.

  Por otro lado, supongamos que $x$ e $y$ son ambos impares, es decir, $x_0
  = 2t + 1$ e $y_0 = 2q + 1$, para $t, q \in \zset$. Se tendrá entonces que

  $$ z_0^2 = (2t + 1)^2 + (2q + 1)^2 = 4t^2 + 4t + 1 + 4q^2 + 4q + 1 = 4(t^2
  + t + q^2 + q) + 2 = 4m + 2 $$

  \noindent para $m \in \zset$. Pero esto no se puede dar, tal y como vimos
  en un ejercicio de la sección de la división en $\zset$ con resto. Según
  vimos, todo número entero al cuadrado se podrá poner de la forma $4m$ o
  $4m + 1$, pero no como $4m + 2$ o $4m + 3$.

  Por tanto, quedan descartadas las posibilidades de que $x$ e $y$ de la
  terna pitagórica primitiva tengan la misma paridad.
\end{proof}

Ahora, veamos un teorema para hallar ternas pitagóricas primitivas y, por
tanto, soluciones enteras de la ecuación $x^2 + y^2 = z^2$. No olvide que, a
partir de una terna pitagórica primitiva podemos sacar todas las demás.

% Se hallan para los valores positivos, pues luego se pueden multiplicar por
% algún lambda TKTK

\begin{theorem}[Forma de las Ternas Pitagóricas Primitivas]
  La terna $(x, y, z)$ con $x, y, z \in \zset$ es una terna pitagórica
  primitiva con $y \equiv 0 \pmod 2$ si y solo si existen $m, n \in
  \zset^{+}$ con $\gcd(m, n) = 1$, $m > n$ y $m \not\equiv n \pmod 2$ tales
  que

  \begin{alignat*}{2}
    x &= m^2 - n^2 \\
    y &= 2mn \\
    z &= m^2 + n^2
  \end{alignat*}
\end{theorem}

En la demostración, nos servirán los lemas anteriores.

\begin{proof}
  Sea $(x, y, z)$ una terna pitagórica primitiva en la que $y$ es par.
  Mostraremos que existen $m, n \in \zset$ de la forma que se especifica en
  el teorema.

  En la demostración, vamos a restringirnos a los valores $x, y, z \in
  \nset^{+}$, pues, al ser 2 el exponente, cualquier valor de estas
  variables podremos luego sustiruirlo por su inverso aditivo y la ecuación
  seguirá cumpliéndose.

  Como en el enunciado se dice que $y$ es par, no queda otra que $x$ sea
  impar, por aplicación del Lema \ref{lema-ternas-pit-prim-paridad}.
  También, podemos saber la paridad de $z$ a partir de estos datos y la
  ecuación que tienen que satisfacer, es decir, $x^2 + y^2 = z^2$. Por un
  lado, $x^2$ será impar, ya que el cuadrado de un número impar es también
  impar. Por otro, $y^2$ será par, ya que el cuadrado de un número par es
  par. Entonces, $z^2$ será impar, al ser la suma de un par y un impar. Pero
  también sabemos que, si $z^2$ es impar, entonces $z$ es impar, cosa que
  puede demostrarse fácilmente por su contrarrecíproco.

  Así, tendremos que $z + x$ y $z - x$ serán pares, pues son suma y resta de
  dos números impares. Por tanto, existirán $r, s \in \zset$ tales que

  $$ r = \frac{z + x}{2} \quad \text{;} \quad s = \frac{z - x}{2} $$

  Puesto que se tiene que cumplir $x^2 + y^2 = z^2$, se tiene que $y^2 = z^2
  - x^2 = (z + x)(z - x)$, por lo que

  $$ \left( \frac{y}{2} \right)^2 = \left( \frac{z + x}{2} \right) \cdot
  \left( \frac{z - x}{2} \right) = rs $$

  Ahora veremos que $\gcd(r, s) = 1$. Llamaremos $d$ a $\gcd(r, s)$, es
  decir, $d = \gcd(r, s)$. Puesto que $d \mid r$ y $d \mid s$, se tiene que
  $d \mid (r + s)$, o, lo que es lo mismo, $d \mid z$. También, $d \mid (r -
  s)$, que es igual que decir que $d \mid x$. Entonces, por la definición de
  máximo común divisor, se tendrá que $d \mid \gcd(x, z)$, pero, puesto que
  $x$ y $z$ son elementos de una terna pitagórica primitiva, se tendrá,
  según el Lema \ref{lema-terna-pit-prim-dos-a-dos}, que $\gcd(x, z) = 1$,
  con lo que tenemos que $d = 1$, es decir, $\gcd(r, s) = 1$.

  Sabiendo esto y que su producto es un cuadrado perfecto, concretamente,
  como hemos visto, $(y/2)^2$, aplicando la Proposición
  \ref{prod-cuadrado-perfecto} se tiene que $r$ y $s$ serán, a su vez,
  cuadrados perfectos, por lo que existen $m, n \in \zset^{+}$ tales que $r
  = m^2$ y $s = n^2$. Escribiendo entonces $x$, $y$ y $z$ en términos de $r$
  y $s$, y estos, a su vez, en términos de $m$ y $n$, tenemos

  \begin{alignat*}{2}
    x &= r - s = m^2 - n^2 \\
    y &= \sqrt{4rs} = \sqrt{4m^2n^2} = 2mn \\
    z &= r + s = m^2 + n^2
  \end{alignat*}

  Vemos también que $\gcd(m, n) = 1$, pues cualquier divisor común de $m$ y
  $n$ debe ser también divisor de $m^2 - n^2$, de $2mn$ y de $m^2 + n^2$,
  que son $x$, $y$ y $z$, respectivamente, para los que se cumple $\gcd(x,
  y, z) = 1$. Es decir, $\gcd(x, y, z) = 1$ es condición suficiente para que
  $\gcd(m, n) = 1$.

  También advertimos que no pueden ser ambos impares, pues, de serlo,
  entonces $x$, $y$ y $z$ serían todos pares, cosa que contradice que
  $\gcd(x, y, z) = 1$. Tampoco pueden ser pares, puesto que $\gcd(m, n) =
  1$. Por tanto, no queda otra que $m \not\equiv n \pmod 2$, es decir, $m$ y
  $n$ tendrán paridad diferente. Esto muestra que toda terna pitagórica
  primitiva tiene la forma que muestra el teorema.

  Para completar la demostración, debemos mostrar que toda terna $(x, y, z)$
  para la que se cumpla

  \begin{alignat*}{2}
    x &= m^2 - n^2 \\
    y &= 2mn \\
    z &= m^2 + n^2
  \end{alignat*}

  \noindent con $m, n \in \zset^{+}$, $\gcd(m, n) = 1$ y $m \not\equiv n
  \pmod 2$ es una terna pitagórica primitiva.

  Primero, advierta que, para todos los $m, n \in \zset^{+}$, la terna $(m^2
  - n^2, 2mn, m^2 + n^2)$ es pitagórica.

  \begin{alignat*}{2}
    x^2 + y^2 &= (m^2 - n^2)^2 + (2mn)^2 = (m^4 - 2m^2n^2 + n^4) +
      4m^2n^2 \\
              &= m^4 + 2m^2n^2 + n^4 (m^2 + n^2)^2 = z^2
  \end{alignat*}

  Para ver que es además primitiva, debemos mostrar que estos valores de
  $x$, $y$ y $z$ son primos relativos. Supongamos que no lo son. Tendríamos
  que existe un $d \in \nset$ con $d \neq 0$ tal que $\gcd(x, y, z) = d$.
  Existirá entonces un número primo $p$ tal que $p \mid d$.

  Advierta que $p \neq 2$, ya que $x$ es impar al ser $x = m^2 - n^2$
  teniendo $m^2$ y $n^2$ distinta paridad.

  Por otro lado, como $d \mid x$ y $d \mid z$, se tendrá también, ya que $p
  \mid d$, que $p \mid x$ y $p \mid z$. Entonces, $p \mid (x + z)$, que es
  lo mismo que $p \mid 2m^2$. También, $p \mid (x - z)$, que es lo mismo que
  $p \mid 2n^2$. Esto último, juntándolo con que $p$ es primo, hace que $p
  \mid m$ y $p \mid n$, cosa que contradice la hipótesis $\gcd(m, n) = 1$.
  Por tanto, $\gcd(x, y, z) = 1$, con lo que $(x, y, z)$ será una terna
  pitagórica primitiva.
\end{proof}

\iffalse Como en el Burton
\begin{theorem}[Forma de las Ternas Pitagóricas Primmmitivas]
  Las soluciones de la ecuación pitagórica $x^2 + y^2 = z^2$ que satisfacen
  las condiciones

  $$ \gcd(x, y, z) = 1 \quad \text{;} \quad 2 \mid x \quad \text{;} \quad x,
  y, z \in \zset^{+} $$

  \noindent vienen dadas por las fórmulas

  \begin{alignat*}{2}
    x &= 2st \\
    y &= s^2 - t^2 \\
    z &= s^2 + t^2
  \end{alignat*}

  \noindent para $s, t \in \zset^{+}$ con $s > t$ tales que $\gcd(s, t) = 1$
  y teniendo $s$ y $t$ distinta paridad.
\end{theorem}

\begin{proof}
  Supongamos que $(x_0, y_0, z_0)$ es una terna pitagórica primitiva, es
  decir, es una solución de la ecuación $x^2 + y^2 = z^2$, con $x_0, y_0,
  z_0 \in \zset^{+}$ siendo $\gcd(x_0, y_0, z_0) = 1$. Entonces, se tiene
  que dar necesariamente que $x_0$ e $y_0$ tengan distinta paridad, cosa que
  justificamos a continuación.

  Por un lado, si $x_0$ e $y_0$ son pares, es decir, si $x_0 = 2t$ e $y_0 =
  2q$, para $t, q \in \zset$, se tendrá que

  $$ z_0^2 = (2t)^2 + (2q)^2 = 4t^2 + 4q^2 = 4(t^2 + q^2) = (2m)^2 $$

  \noindent para $m \in \zset$, cosa que contradice que $\gcd(x_0, y_0, z_0)
  = 1$, puesto que, por ejemplo, el $\gcd(x_0, z_0)$ ya no sería 1, al ser
  ambos ahora múltiplos de 2.

  Por otro lado, si ambos son impares, es decir, si $x_0 = 2t + 1$ e $y_0 =
  2q + 1$, para $t, q \in \zset$, se tendrá que

  $$ z_0^2 = (2t + 1)^2 + (2q + 1)^2 = 4t^2 + 4t + 1 + 4q^2 + 4q + 1 = 4(t^2
  + t + q^2 + q) + 2 = 4m + 2 $$

  \noindent para $m \in \zset$. Pero esto no se puede dar, tal y como vimos
  en un ejercicio de la sección de la división en $\zset$ con resto. Según
  vimos, todo número entero al cuadrado se podrá poner de la forma $4m$ o
  $4m + 1$, pero no como $4m + 2$ o $4m + 3$.

  Por tanto, $x_0$ e $y_0$ han de tener paridad distinta, necesariamente.
  Supongamos el caso en que $x_0$ es par e $y_0$ impar. Se tendrá que $z_0$
  será impar, ya que

  % Se tendría que analizar también el otro caso? x_0 impar e y_0 par?

  $$ z_0^2 = (2t)^2 + (2q + 1)^2 = 4t^2 + 4q^2 + 4q + 1 = 2(2t^2 + 2q^2 2q)
  + 1 $$

  Por tanto, tanto $z_0 + y_0$ como $z_0 - y_0$ son pares, al ser suma o
  resta de dos números impares. Por tanto, podemos escribir

  \begin{alignat*}{2}
    z_0 - y_0   &= 2p \\
    z_0 + y_0   &= 2q
  \end{alignat*}

  \noindent para $p, q \in \zset^{+}$. No confunda esta $q$ con la anterior.
  Tenemos, entonces, que

  $$ (z_0 - y_0)(z_0 + y_0) = z_0^2 - y_0^2 = x_0^2  $$

  \noindent y, por otro lado, que

  $$ (z_0 - y_0)(z_0 + y_0) = 2p \cdot 2q = 4pq $$

  \noindent y, uniendo ambas, tenemos

  $$ \left( \frac{x_0}{2} \right)^2 = pq $$

  Vamos a hacer ahora algunas observaciones sobre estos resultados. Por un
  lado, se tiene que $\gcd(p, q) = 1$, ya que, si $\gcd(p, q) \neq 1 = d$,
  entonces $d \mid p$ y $d \mid q$, con lo que se tendría que $d \mid (p +
  q)$ y $d \mid (q - p)$, que, manipulando las ecuaciones

  \begin{alignat*}{2}
    z_0 - y_0   &= 2p \\
    z_0 + y_0   &= 2q
  \end{alignat*}

  \noindent y llegando a las expresiones equivalentes

  \begin{alignat*}{2}
    y_0   &= q - p \\
    z_0   &= p + q
  \end{alignat*}

  \noindent vemos que es lo mismo que decir $d \mid y_0$ y $d \mid z_0$,
  cosa que entra en contradicción con la hipótesis de que $x_0$ e $y_0$ sean
  primos relativos. Por tanto, no queda otra que $p$ y $q$ sean primos
  relativos. Si unimos a esto la condición $2 \mid x$ de este teorema, se
  tendrá entonces que $p \, q$ será el cuadrado de un número entero
  positivo. Aplicando aquí la Proposición TKTK, como $p$ y $q$ son primos
  relativos y $p \cdot q$ es un cuadrado perfecto, se tiene que tanto $p$
  como $q$ serán también cuadrados perfectos. Por tanto, tenemos que existen
  $s, t \in \zset^{+}$ tales que $p = t^2$ y $q = s^2$. Entonces,
  sustituyendo a $p$ y $q$ por esos valores, podemos escribir

  \begin{alignat*}{2}
    z_0 - y_0                       &= 2t^2 \\
    z_0 + y_0                       &= 2s^2 \\
    \left( \frac{x_0}{2} \right)^2  &= s^2 t^2
  \end{alignat*}

  \noindent y, operando, obtenemos

  \begin{alignat*}{2}
    z_0   &= s^2 + t^2 \\
    y_0   &= s^2 - t^2 \\
    x_0   &= 2st
  \end{alignat*}

  Además, se da que $s$ y $t$ son primos relativos, ya que, de no serlo,
  tampoco lo serían $p$ y $q$, y antes demostramos que sí lo eran. También,
  observemos que $s$ y $t$ no pueden tener la misma paridad, ya que, si la
  tuviesen, entonces $y_0$ sería par, en contra de la hipótesis que estamos
  siguiendo. Por último, de $2q > 2p$ se deduce que $s > t$.

  Inversamente, supongamos que se cumple

  \begin{alignat*}{2}
    x   &= 2st \\
    y   &= s^2 - t^2 \\
    z   &= s^2 + t^2
  \end{alignat*}

  \noindent con las condiciones $s > t > 0$, $\gcd(s, t) = 1$ y $s$ y $t$
  con paridad distinta. Entonces,

  $$ x^2 + y^2 = 4s^2 t^2 + (s^2 - t^2)^2 = (s^2 + t^2)^2 = z^2 $$

  \noindent cosa que prueba que es una terna pitagórica, es decir, que es
  solución de la ecuación $x^2 + y^2 = z^2$.

  Para comprobar que la terna $(2st, s^2 - t^2, s^2 + t^2)$ es primitiva,
  tenemos que comprobar que $\gcd(x, y, z) = 1$. Sea $p$ un factor primo de
  $\gcd(x, y, z)$. Como $p \mid z$ y $z$ es impar, entonces $p \neq 2$. De
  $p \mid y$ y $p \mid z$, deducimos que $p \mid (z + y)$ y $p \mid (z -
  y)$, o, lo que es lo mismo, $p \mid 2s^2$ y $p \mid 2t^2$, y, al ser $p
  \neq 2$, entonces $p \mid s^2$ y $p \mid t^2$, y, de aquí, $p \mid s$ y
  $p \mid t$, cosa que contradice la hipótesis de que $s$ y $t$ sean primos
  relativos. Por tanto, $(2st,  s^2 - t^2, s^2 + t^2)$ con las condiciones
  anteriores es una terna primitiva pitagórica. Cualquier otra solucíón de
  la ecuación $x^2 + y^2 = z^2$ será de la forma

  \begin{alignat*}{2}
    x &= \lambda 2st \\
    y &= \lambda(s^2 - t^2) \\
    z &= \lambda(s^2 + t^2)
  \end{alignat*}
\end{proof}
\fi







% ------------------------------------------------

\iffalse En realidad, es un caso particular del Lema de Euclides.

Otras demostraciones accesorias

\begin{proposition}
  Dado un $n \in \zset$. Si $2 \mid n^2$, entonces $2 \mid n$.
\end{proposition}

\begin{proof}
  Un enunciado alternativo sería decir que, si $n^2$ es par, entonces $n$
  también será par.

  Vamos a tratar de demostrar su contrarrecíproco, que sería que, si $n$ es
  impar, entonces $n^2$ también será impar. Tenemos que existe un $k \in
  \zset$ tal que $n = 2k + 1$. Entonces,

  $$ n^2 = (2k + 1)^2 = 4k^2 + 4k + 1 = 2(2k^2 + 2k) + 1 $$

  \noindent con lo que ya ha quedado demostrado.
\end{proof}

\fi

Así, tenemos las cinco primeras térnas pritagóricas primitivas, que
mostramos en la tabla \ref{primeras-ternas-pitagoricas-primitivas}.

\begin{figure}\caption{Primeras ternas pitagóricas primitivas}
  \centering
  \begin{tabular}{| c | c | c | c | c |}
    \hline
    $m$ & $n$ & $x = m^2 - n^2$ & $y = 2mn$ & $z = m^2 + n^2$ \\
    \hline
    \hline
    2   & 1   & 3   & 4   & 5 \\
    \hline
    3   & 2   & 5   & 12  & 13 \\
    \hline
    4   & 1   & 15  & 8   & 17 \\
    \hline
    4   & 3   & 7   & 24  & 25 \\
    \hline
    5   & 2   & 21  & 20  & 29 \\
    \hline
  \end{tabular}\label{primeras-ternas-pitagoricas-primitivas}
\end{figure}
