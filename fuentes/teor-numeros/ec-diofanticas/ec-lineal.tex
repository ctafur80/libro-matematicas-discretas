


\begin{theorem}[Condición para que la Ecuación Diofántica Lineal Tenga
  Solución]
  Dados $a, b, n \in \nset$. La ecuación lineal $ax + by = n$ tiene solución
  entera si y solo si $n$ es un múltiplo de $\gcd(a, b)$.
\end{theorem}

\begin{proof}
  Veamos primero el caso en el que $a \neq 0$ y $b \neq 0$, dejando para el
  final las particularidades en las que hay algún 0 involucrado.

  Por hacer más cómoda la notación, definiremos $d = \gcd(a, b)$.

  Vamos a demostrar primero que, si $x_0, y_0 \in \zset$ son soluciones de
  la ecuación, de $x$ e $y$ respectivamente, es decir, se cumple

  $$ ax_0 + by_0 = n $$

  \noindent entonces $d \mid n$.

  Como $d$ es divisor común de $a$ y $b$, se cumple, por la Proposición
  \ref{prop-divide-al-multiplo} y la \ref{princ-dos-de-tres}, que $d \mid
  (ax_0 + by_0)$, o, lo que es lo mismo, $d \mid n$.

  Ahora, hay que demostrar el condicional en el otro sentido. Partimos de
  que se cumple $d \mid n$ y deseamos demostrar que existen $x_0, y_0 \in
  \zset$ tales que

  $$ ax_0 + by_0 = n $$

  Como $d = \gcd(a, b)$, existirán, por el Lema \ref{th-bezout}, $k, l \in
  \zset$ tales que

  $$ d = ak + bl $$

  Además, como $d \mid n$, existe un $m \in \zset$ para el que se cumple $n
  = md$. Por tanto,

  \begin{alignat*}{2}
    d             &= ak + bl \\
    d\frac{n}{d}  &= (ak + bl)\frac{n}{d} \\
    n             &= a\frac{n}{d}k + b\frac{n}{d}l =
      a\left(\frac{n}{d}k\right) + b\left(\frac{n}{d}l\right)
  \end{alignat*}

  Cambiando de nombre a esos coeficientes tal que

  $$ x_0 = \frac{n}{d}k \quad \text{;} \quad y_0 = \frac{n}{d}l $$

  \noindent tendríamos ya una solición particular de la ecuación diofántica
  lineal.

  Para el caso en que $n = 0$, los valores $x_0 = y_0 = 0$ son solución de
  la ecuación.

  % TODO. Caso a \neq 0 y b = 0.
\end{proof}

En el fondo esto ya lo habíamos demostrado, pues es lo mismo que el
Corolario \ref{cor-mult-mcd}, que dice que las combinaciones lineales
enteras de dos números generan todos los múltiplos de su máximo común
divisor.

\begin{theorem}[Solución Particular de la Ecuación Diofántica Lineal]
  La solución de la ecuación diofántica lineal tiene la forma siguiente:

  $$ x_0 = \frac{n}{d}k \quad \text{;} \quad y_0 = \frac{n}{d}l $$

  \noindent siendo $k, l \in \zset$ los coeficientes de Bézout de $a$ y $b$
  y siendo $d = \gcd(a, b)$.
\end{theorem}

La demostración se encuentra dentro del teorema anterior.

\iffalse
Si se fija, una forma de obtener esas soluciones particulares sería
obteniendo el $\gcd(a, b)$ y un par de coeficientes de Bézout, $k$ y $l$, y
usarlos en las ecuaciones que hemos deducido para $x_0$ e $y_0$. Eso es
básicamente lo que se hace en el algoritmo siguiente.

Veamos ahora un algoritmo para encontrar la solución. Dada la ecuación $ax +
by = n$, calculemos en primer lugar el máximo común divisor de $a$ y $b$
mediante el Algoritmo de Euclides. Sabemos que

\begin{alignat*}{2}
  a       &=  bq_1 + r_1 \\
  b       &=  r_1q_2 + r_2 \\
  r_1     &=  r_2q_2 + r_3 \\
  \vdots  &= \vdots + \vdots \\
  r_{t-2} &= r_{t-1}q_t + r_t \\
  r_{t-1} &= r_tq_{t+1} + 0 \\
\end{alignat*}

\noindent donde $r_t = \gcd(a, b) = d$. Por tanto, $d = r_{t-2} -
r_{t-1}q_t$. También, podemos poner $r_{t-1}$ según la ecuación anterior,
teniendo así

$$ d = r_{t-2} - (r_{t-3} - r_{t-2}q_{t-1})q_t $$

\noindent que, reordenando los términos, es lo mismo que

$$ d = {-r_{t-3}} + r_{t-2} (1 + q_tq_{t-1}) $$

Si continuamos ascendiendo por la pila de igualdades del Algoritmo de
Euclides, al final llegamos a poder escribir $d$ como

$$ d = ak + bl $$

\noindent donde $k$ y $l$ son expresiones en función de $q_1, q_2, q_3,
\ldots, q_t$, pero al fin y al cabo no son más que un par de coeficientes de
Bézout. Por el Teorema \ref{teor-sol-ec-diof-lin}, una solución de la
ecuación será

% Lo dice en la demostración. Quizás, sea mejor cambiar el teorema.

$$ x_0 = \frac{nk}{d} \quad \text{;} \quad y_0 = \frac{nl}{d} $$

\fi

En cualquier caso, esa solución es una de las muchas soluciones que tiene la
ecuación. Veamos la forma de la solución general.

\begin{theorem}[Solución General de la Ecuación Diofántica
  Lineal]\label{teor-sol-gral-ec-diof-lin}
  Dados $a, b, n \in \zset$ con $a \neq 0$, $b \neq 0$ y $n \neq 0$ y $d =
  \gcd(a, b)$. La solución general de la ecuación

  $$ n = ax + by $$

  \noindent para $x, y \in \zset$ siendo $n$ un múltiplo de $d$, es

  $$ x = x_0 + t\frac{b}{d} \quad \text{;} \quad y = y_0 - t\frac{a}{d} $$

  \noindent siendo $t \in \zset$ el parámetro de esta y $(x_0, y_0)$ una
  solución particular de la ecuación diofántica lineal.
\end{theorem}

% TODO Ver si se pueden quitar las excepciones con 0.

\begin{proof}
  Supongamos que tenemos una solución particular $(x_0, y_0)$ para la
  ecuación, es decir, se cumple

  $$ ax_0 + by_0 = n $$

  Como $d \mid a$, $d \mid b$ y $d \mid n$, podemos dividir entre $d$ todas
  las partes de la ecuación y seguiríamos teniendo coeficientes enteros.
  Tendríamos

  $$ \frac{ax_0}{d} + \frac{by_0}{d} = \frac{n}{d} $$

  También tenemos, por el Teorema \ref{teor-mcd-multiplos},

  $$ \gcd\left(\frac{a}{d}, \frac{b}{d}\right) = \frac{d}{d} = 1 $$

  Vamos a designar por $(x_1, y_1)$ a otra solución particular. Se podría
  hacer lo mismo:

  $$ \frac{ax_1}{d} + \frac{by_1}{d} = \frac{n}{d} $$

  Entonces, restando esta y su análoga para $(x_0, y_0)$, tenemos

  $$ \frac{a}{d}(x_1 - x_0) + \frac{b}{d}(y_1 - y_0) = \frac{n}{d} -
  \frac{n}{d} = 0 $$

  \noindent y, reordenando,

  \begin{equation}\label{eq-temp-diofant-sol-general}
  \frac{a}{d}(x_1 - x_0) = \frac{b}{d}(y_0 - y_1)
  \end{equation}

  Por tanto, se tiene que

  $$ \frac{a}{d} \mid \frac{b}{d} (y_0 - y_1) $$

  \noindent y, por el Lema \ref{th-lema-euclides}, como se tiene que como
  $\gcd(a/d, b/d) = 1$, se dará que

  $$ \frac{a}{d} \mid (y_0 - y_1) $$

  \noindent o, lo que es lo mismo, existe un $t \in \zset$ para el que

  $$ y_0 - y_1 = t \frac{a}{d} $$

  \noindent o sea,

  $$ y_1 = y_0 - t \frac{a}{d} $$

  \noindent que sería la parte de $y$ de la ecuación general. Para hallar
  $x_1$, vamos a sustituir este resultado en la ecuación
  \ref{eq-temp-diofant-sol-general}. Tendríamos

  $$ \frac{a}{d}(x_1 - x_0) = \frac{b}{d}\left(y_0 - y_0 +
    t\frac{a}{d}\right) $$

  \noindent con lo que tenemos

  $$ x_1 = x_0 + t \frac{b}{d} $$

  Recíprocamente, se tiene que, para todo $t \in \zset$, cada par de valores
  $(x_1, y_1)$ de la forma

  $$ y_1 = y_0 - t \frac{a}{d} \quad \text{;} \quad x_1 = x_0 + t
  \frac{b}{d} $$

  \noindent será una solución particular de la ecuación diofántica lineal.
  Veamos por qué:

  \begin{alignat*}{2}
    ax_1 + by_1   &= a\left(x_0 + t\frac{b}{d}\right) + b\left(y_0 -
      t\frac{a}{d}\right) = ax_0 + at\frac{b}{d} + by_0 - bt\frac{a}{d} \\
                  &= ax_0 + by_0 = n
  \end{alignat*}

  Esa solución $(x_1, y_1)$ en realidad sería una solución genérica, con lo
  que es más adecuado designarla por $(x, y)$.
\end{proof}










