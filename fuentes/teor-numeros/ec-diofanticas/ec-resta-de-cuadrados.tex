

Viene explicado en \cite{burton} sección 5.4, pág. 97, y lo
llama el método de factorización de Fermat-Kraitchik.

Ahora, vamos a analizar la ecuación diofántica $x^2 - y^2 = n$ con $n \in
\nset$ con $n > 0$.

\begin{theorem}[Solución General de la Ecuación Diofántica Resta de
  Cuadrados]\label{th-sol-gral-ec-diof-cuad}
  Dado $n \in \nset^{+}$. La solución general de la ecuación $x^2 - y^2 =
  n$, para $x, y \in \zset$, es

  $$ x = \frac{a + b}{2} \quad \text{;} \quad y = \frac{a - b}{2} $$

  \noindent para todos los $a, b \in \zset$ con $a \equiv b \pmod 2$ y tales
  que $n = a \cdot b$.
\end{theorem}

\begin{proof}
  Primero demostremos que si $n$ se puede factorizar como dos números $a, b
  \in \zset$ teniendo ambos la misma paridad, entonces todo par de números
  con la forma

  $$ x = \frac{a + b}{2} \quad \text{;} \quad y = \frac{a - b}{2} $$

  \noindent es solución de la ecuación.

  Partimos de la ecuación diofántica:

  $$ n = x^2 - y^2 $$

  \noindent Esto podemos manipularlo del mismo modo a como se hace en el
  álgebra de los números reales:

  $$ n = x^2 - y^2 = (x + y)(x - y) $$

  Vamos a demostrar ahora que esos dos factores de la derecha tienen la
  misma paridad. Partimos de la expresión $x + y$:

  \begin{alignat*}{2}
    x + y           &= x + y \\
    x + y - y       &= x + y - y \\
    x + y - y + y   &= x + y - y + y = x - y + y + y \\
    x + y           &= (x - y) + 2y
  \end{alignat*}

  Como sabemos, la suma de un número par más uno par será un número par y,
  por otro lado, la suma de un número impar más uno par da uno impar. Así,
  en la ecuación anterior, ese $2y$ no afecta a la paridad, por lo que $x +
  y$ y $x - y$ tienen necesariamente la misma paridad. Así, hemos encontrado
  dos números de la misma paridad que multiplicados dan $n$.

  Ahora, pasaremos a llamar $a$ y $b$, respectivamente, a estos factores de
  $n$.

  \begin{alignat*}{2}
    x + y &= a \\
    x - y &= b
  \end{alignat*}

  Manipulando este sistema de ecuaciones, para obtener $x$ e $y$ en función
  de $a$ y $b$, tenemos

  \begin{alignat*}{2}
    (x + y) + (x - y)   &= a + b \\
    x + x + y - y       &= a + b \\
    2x                  &= a + b
  \end{alignat*}

  \noindent y

  \begin{alignat*}{2}
    (x + y) - (x - y)   &= a - b \\
    x + y - x + y       &= a - b \\
    2y                  &= a - b
  \end{alignat*}

  \noindent con lo que llegamos a que

  $$ x = \frac{a + b}{2} \quad \text{;} \quad y = \frac{a - b}{2} $$

  \noindent que es la forma general de la solución a esta ecuación
  diofántica.

  Ahora, demostraremos la implicación en el otro sentido. Partimos de que

  $$ x = \frac{a + b}{2} \quad \text{;} \quad y = \frac{a - b}{2} $$

  \noindent con $a, b \in \zset$ con paridad distinta es la solución general
  de esta ecuación diofántica y debemos demostrar que entonces $n$ es un
  producto de $a$ por $b$. Simplemente, hay que sustituir.

  \begin{alignat*}{2}
    n &= x^2 - y^2 = \left( \frac{a + b}{2} \right)^2 - \left( \frac{a -
      b}{2} \right)^2 = \frac{(a + b)^2 - (a - b)^2}{4} \\
      &= \frac{a^2 +2ab + b^2 - (a^2 - 2ab + b^2)}{4} = \frac{4ab}{4} = ab
  \end{alignat*}
\end{proof}

algo que está implícito es que $a$ y $b$ son menores que $n$. Es evidente
TKTK.

Si se fija, curiosamente, esta solución general (de grado 2) es más sencilla
que en el caso de la ecuación diofántica lineal (también llamada ``de grado
1'').

En este tipo de ecuación, el problema se encontrará en que $n$ sea un número
muy grande. En ese caso, resultará complicado comprobar si ese número será
posible factorizarlo.

Lo que sí que sabemos es que, si $n$ es par, entonces sí se podrá factorizar
como 2 por otro número. El problema está entonces en que $n$ sea impar. Para
el caso en que $n$ sea impar, el resultaado siguiente nos va a proporcionar
un algoritmo para determinar si $n = a \, b$ con $a$ y $b$ con la misma
paridad, que en este caso sería impar. A este algoritmo se le conoce como el
Algoritmo de Fermat.

Además, a dicho algoritmo también se le puede dar el uso de comprobar si un
número es compuesto o primo. Advierta que se hace para números impares,
pues, si un número es par y no es 2, será compuesto necesariamente. Es más
eficiente que la Criba de Eratóstenes.
