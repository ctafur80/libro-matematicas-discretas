

Viene bien explicado en \cite{rosen} sección 3.6, pág. 130.

Supongamos un número $n \in \nset$ impar y compuesto. Por el Teorema
\ref{th-sol-gral-ec-diof-cuad} se tendrá que existen $a, b \in \zset$ tales
que $n = a \, b$ con los que se cumple

$$ n = \left( \frac{a + b}{2} \right)^2 - \left( \frac{a - b}{2} \right)^2
$$






Si un número $n \in \nset$ con $n > 0$ impar es compuesto, existen $a, b \in
\zset$ impares con $1 < a < n$ y $1 < b < n$ para los que $n = ab$. Esto se
debe a que la única multiplicación que nos da impar es impar por impar. Las
demás posibilidades dan un número par. Por lo tanto, tanto $a$ como $b$ son
impares.

Podemos suponer también que $a \geq b > 1$, sin pérdida de generalidad. Si
alguno fuese uno, el resultado sería que $n$ es primo, pues la factorización
sería $n = a$ o $n = b$.

Si se fija, por el Teorema \ref{th-sol-gral-ec-diof-cuad}, estudiar si es
compuesto un número $n$ natural impar mayor que 1 se puede reducir a tartar
de resolver la ecuación diofántica $x^2 - y^2 = n$. Aquí, optaremos por
presentarla, normalmente, en la forma

$$ y^2 = x^2 - n $$

Lo que haremos será ir probando con los distintos valores de $x$ y comprobar
si se tiene un $y$ que haga que se cumpla la ecuación. Dicho de otra forma,
probar con valores de $x$ si $x^2 - n$ es un cuadrado perfecto. Es una forma
de hacerlo ``por fuerza bruta'', pero, al movernos en los números enteros,
no es un problema.

Pero no hay que partir desde el 1. Ya que, necesariamente, $y^2 \geq 0$,
partiremos desde un número $x$ tal que $x^2 \geq n$. Al primer número que
cumpla esto vamos a designarlo por la variable $q$, que en realidad depende
de $x$.

Así, empezamos por analizar si $q^2 - n$ es un cuadrado perfecto. En caso de
que lo sea, podríamos afirmar que $n$ es compuesto y terminaría el
algoritmo. Si lo desea, puede hallar esos dos factores que lo forman, aunque
no es necesario para nuestro propósito. Si, por el contrario, $q^2 - n$ no
fuese un cuadrado perfecto, se pasaría a analizar el siguiente, es decir,
$(q+1)^2 - n$. Y así sucesivamente.

Este proceso no es infinito, ya que, cuando llegamos al valor de $q$ tal que

$$ q = \frac{n + 1}{2} $$

\noindent como

$$ \left( \frac{n + 1}{2} \right)^2 - n = \frac{n^2 + 2n + 1 - 4n}{4} =
\left( \frac{n - 1}{2} \right)^2 $$

\noindent tendríamos que $y^2$ sería un cuadrado perfecto. Si sustituimos
entonces estos valores de $x$ e $y$ en las ecuaciones del Teorema
\ref{th-sol-gral-ec-diof-cuad}, tenemos que $a = n$ y $b = 1$, pero, en ese
caso, $n$ ya no podrá ser compuesto, sino primo, y terminaría el algoritmo.

Así, los únicos valores que analizará el algoritmo serán los números $m \in
\nset$ comprendidos entre

$$ q \leq m < \frac{n + 1}{2} $$

\noindent siendo $q \in \zset$ el primer número tal que $q^2 \geq n$.

En resumen. En caso de que no se encuentre ningún número $m$ en dicho rango
tal que $m^2 - n$ sea un cuadrado perfecto, se puede afirmar que $n$ es un
número primo. En caso contrario, se puede afirmar que existen $a, b \in
\zset$ tales que se cumple $n = a \, b$, y podemos hallar estos valores del
modo siguiente:

$$ x = m = \frac{a + b}{2} \quad \text{;} \quad y = \sqrt{m^2 - n} = \frac{a
- b}{2} $$

% Implementarlo en código Python. TODO.

Una implementación de este algoritmo en código Python sería la siguiente:

\begin{verbatim}
# Comprueba si el número dado es primo, por medio del algoritmo de factorización de Fermat.

import sys
import math


def obten_q(n):
    q = 1
    while q * q < n:
        q = q + 1

    return q


# num = 22733
num = int(sys.argv[1])


# Comprueba si se trata de un número válido para este algoritmo.
# TODO. También se podría hacer con una afirmación assert.
if num <= 1:
    print("El número introducido no es válido para usar el algoritmo de Fermat ya que "
          "no se trata de un número mayor que 1.")
    exit()

elif num % 2 == 0:
    print("El número introducido no es válido para usar el algoritmo de Fermat ya que "
          "no se trata de un número impar.")
    exit()



q = obten_q(num)
m = q

while m < (num + 1)/2:
    if math.sqrt(m*m - num) % 1 == 0:
        x = m
        y = int(math.sqrt(m*m - num))
        a = x + y
        b = x - y

        print(f"El número {num} es compuesto, pues se tiene que {num} = {a} x {b}.")
        exit()

    m = m + 1



print(f"El número {num} es primo.")
exit()
\end{verbatim}
