


Esta es la división que aprendió en la educación primaria, solo que ahora
justificamos por qué funciona.

\begin{theorem}[de la División en $\zset$ con Resto]%
\label{th-div-entera-con-resto}
  Dados $a, b \in \zset$ con $b \neq 0$. Existe un único par de números $q,
  r, \in \zset$ con $0 \leq r < |b|$ tales que

  $$ a = b \cdot q + r $$
\end{theorem}

Se trata de un teorema de existencia y unicidad.

\iffalse
Otra opción sería poner $0 \leq r \leq |b-1|$.
\fi

Lo que viene a decir este teorema es que existe una operación, que llamamos
división en $\zset$ con resto, de tal forma que podemos expresar TKTK. A
esta operación también la llaman \m{división euclidiana} (\emph{Euclidean
division}) o \m{división euclidea}.\footnotemark

\footnotetext{Este teorema puede ver que lo llaman de muchas formas
distintas. La que usa el libro es (teorema del) Algoritmo de la División,
pero me parece desacertada, pues en realidad no se trata de un algoritmo.
También, puede ver que lo llaman Lema de la División de Euclides o Teorema
de Euclides. La forma en que lo llamo yo, ``división en $\zset$ con resto'',
la he copiado de Paolo Aluffi, un profesor de la Universidad de Florida que
ha escrito algunos libros muy buenos sobre álgebra abstracta. También se
podría decir División Entera con Resto.}

A los distintos elementos que aparecen en esta forma de expresar la relación
entre dos números se les dan los nombres siguientes:

\begin{center}
\begin{tabular}{l|l}
  $a$ & dividendo o numerador \\
  $b$ & divisor o denominador \\
  $q$ & cociente \\
  $r$ & resto
\end{tabular}
\end{center}

Pero, ¿por qué se excluye de la definición el caso en el que el divisor,
$b$, valga 0? Porque, en ese caso, el intervalo en el que se encuentra $r$
sería el conjunto vacío, es decir, no tiene sentido decir $0 \leq r < 0$.
Además, en ese caso el cociente no sería único.

Vamos a ver ahora la demostración del teorema.

\begin{proof}
  La demostración va a ser de tipo constructivo.

  Vamos a dividirla en dos casos.

  Caso $b > 0$.

  Tomamos un valor $q \in \zset$ tal que sea el mayor con el que se cumpla
  $bq \leq a$. Es decir, $bq$ es el mayor múltiplo de $b$ que es menor o
  igual que $a$. Entonces, se da

  $$ bq \leq a < b(q+1) $$

  \noindent La desigualdad de la derecha tiene que darse necesariamente
  porque, si con nuestro $q$ no se cumple $a < b(q+1)$, es que lo hemos
  seleccionado mal. En ese caso, tendríamos que haber elegido $q+1$ o algún
  otro valor superior.

  Ahora, defino una variable $r = a - bq$, ya que deseo que se dé $a = bq +
  r$, y resto $bq$ en todas las partes de ambas desigualdades.

  \begin{alignat*}{2}
    bq - bq   &\leq a - bq   < b(q+1) - bq = bq + b - bq \\
    0         &\leq r        < b \\
    0         &\leq r        < |b|
  \end{alignat*}

  \noindent Se da que $b = |b|$ porque nos encontramos en el caso en el que
  $b > 0$. Por cierto, advierta que la desigualdad $0 \leq a - bq$ es
  coherente con que, tal y como dijimos, $bq$ es el mayor múltiplo de $b$
  menor o igual que $a$.

  Con esto, hemos demostrado la primera parte, es decir, que existen $q, r
  \in \zset$ tales que $a = bq + r$ con $0 \leq r < |b|$. Ojo, existen uno o
  varios. Es decir, hemos demostrado la existencia; ahora, queda demostrar
  la unicidad.

  Para esto, vamos a tratar de hacer una demostración por contradicción.
  Suponemos que lo anterior se cumple para dos pares de $q$ y $r$ distintos
  entre sí. Los designaremos por $q_1$ y $r_1$, y $q_2$ y $r_2$, y suponemos
  que se da, tal y como acabamos de decir, $r_1 \neq r_2$.

  Tenemos entonces que

  \begin{alignat*}{2}
    a = bq_1 + r_1  & \quad \text{con} \ 0 \leq r_1 < |b| = b \\
    a = bq_2 + r_2  & \quad \text{con} \ 0 \leq r_2 < |b| = b
  \end{alignat*}

  Vamos a quedarnos, de momento, con las ecuaciones de la izquierda. Si
  restamos ambas expresiones, tenemos

  \begin{alignat*}{2}
    a - a       &= b q_1 + r_1 - (b q_2 + r_2) \\
    0           &= b(q_1 - q_2) + (r_1 - r_2) \\
    r_2 - r_1   &= b(q_1 - q_2) \tag{*}\label{euclid-ref-1}
  \end{alignat*}

  Por tanto, tenemos que $b \mid (r_2 - r_1)$, ya que $(q_1 - q_2) \in
  \zset$. Por la Proposición \ref{prop-factor-men}, se tiene que

  $$ |b| \leq |r_2 - r_1| $$

  Ahora, nos centraremos en las desigualdades. Ya que se cumplen $0 \leq r_1
  < |b| = b$ y $0 \leq r_2 < |b| = b$, se dará necesariamente que, en la
  recta real (o entera), la distancia entre $r_1$ y $r_2$ será menor
  (estrictamente) que $b$, que es lo mismo que decir que

  $$ |r_2 - r_1| < b = |b| $$

  \noindent que justamente contradice la conclusión a la que acabábamos de
  llegar.

  Por tanto, al darse esta contradicción, la premisa de la que partimos, que
  era que $r_1 \neq r_2$ será falsa. Por lo tanto, se da que $r_1 = r_2$.

  Además, si $r_1 = r_2$, veamos qué sucederá al sustituirlos en la
  ecuación~\ref{euclid-ref-1}.

  $$ r_2 - r_1 = b(q_1 - q_2) $$

  \noindent Como $r_1 = r_2$, se tiene

  $$ 0 = b(q_1 - q_2) $$

  \noindent y, como partimos de la premisa de que $b \neq 0$, se tiene que
  $q_1 - q_2 = 0$, o, lo que es lo mismo, $q_1 = q_2$.

  Así, hemos demostrado que $q$ y $r$ son únicos.

  Caso $b < 0$.

  Si $b < 0$, entonces, ${-b} > 0$ y podemos aplicar lo que demostramos para
  el caso anterior. Tenemos que existen $q, r \in \zset$ siendo $0 \leq r <
  |{-b}|$, para los que se cumple

  $$ a = ({-b})q + r$$

  \noindent Ahora, podemos manipular la expresión anterior.

  $$ a = ({-b})q + r = ({-q})b + r \quad \text{con} \ 0 \leq r < |{-b}| =
  |b| $$

  \noindent con lo que hemos demostrado que existen $({-q}), r \in \zset$
  únicos con los que se cumple la división en $\zset$ con resto de $a$ entre
  $b$, siendo $b < 0$.
\end{proof}

Si no le convence la parte en la que se justifica que $|r_2 - r_1| < |b|$
por la distancia entre estos, podemos hacerlo alternativamente manipulando
expresiones. Partimos de

\[
  \begin{array}{c@{}c@{}c}
    0 &\leq r_1 &\leq |b| \\
    0 &\leq r_2 &\leq |b|
  \end{array}
\]

Ahora, operamos $0 \leq r_1$ para llegar a ${-r_1} \leq 0$.

De esto se deduce que

\[ r_2 - r_1 < |b| - r_1 \leq |b| \]

\noindent ya que ${-r_1} \leq 0$. Además, se tiene que

\[ |r_1 - r_1| \geq 0 \]

\noindent como es evidente. Por tanto, se tiene que

\[ 0 \leq |r_2 - r_1| < |b| \]

Advierta que, tal y como se ve en la primera parte de la demostración, $bq$
queda siempre por ``debajo'' de $a$, tanto si se trata de $a \geq 0$ como de
$a < 0$. Concretamente, a una distancia menor de $|b|$. Así, por ejemplo,
para los valores $a = {-15}$ y $b = 8$ se descompondrá del modo siguiente:

$$ {-15} = 8 \cdot ({-2}) + 1 = {-16} + 1 \quad \text{con} \ 0 \leq 1 < 8 $$

\noindent Es decir, ese $q$ tiene que valer ${-2}$ para que $bq$ quede por
``debajo'' de ${-15}$.

Existen otros tipos de divisiones, como, por ejemplo, una en la que el valor
absoluto del resto sería el mínimo posible. Esta viene explicada en
\cite{weissman}, pero aquí no la veremos.

Existe otra forma de llamar al resto de este tipo de división. Se trata del
\m{operador módulo} (\engm{modulus operator}).

\begin{deffinition}[Operador módulo]
  Dados $a, b \in \zset$ con $b \neq 0$, definimos el operador módulo, que
  designamos por `$\mathrm{MOD}$', como el resto $r$ que obtenemos al
  aplicar el Teorema \ref{th-div-entera-con-resto}.

  $$ a \opmod b = r $$

  \noindent tal que

  $$ b = aq + r \quad \text{con} \ 0 \leq r < |b| $$
\end{deffinition}

Existe otro operador que guarda relación con esta operación y se le conoce
como el \m{operador congruencia}. Lo designaríamos del modo siguiente:

$$ a \equiv r \pmod b $$

En cualquier caso, este operador se estudiará en profundidad en un capítulo
posterior.

\begin{properties}
  Si tenemos $a, b, c, d, m \in \zset$ con $m \neq 0$ y se da que

  $$ a \opmod m = c \opmod m \quad \text{y} \quad b \opmod m = d \opmod m $$

  \noindent entonces:

  \begin{enumerate}
    \item $(a + b) \opmod m = (c + d) \opmod m$.
    \item $(a \cdot b) \opmod m = (c \cdot d) \opmod m$.
  \end{enumerate}
\end{properties}

\begin{proof}
  Estas demostraciones son triviales.

  \begin{enumerate}
    \item Como se cumple $a \opmod m = c \opmod m$, se tiene que existen
      $r_1, q_1, q_2 \in \zset$ con $0 \leq r_1 < |m|$ tales que

      \begin{alignat*}{2}
        a &= mq_1 + r_1 \\
        c &= mq_2 + r_1
      \end{alignat*}

      Por otro lado, como se cumple $b \opmod m = d \opmod m$, se tiene que
      existen $r_2, q_3, q_4 \in \zset$ con $0 \leq r_2 < |m|$ tal que

      \begin{alignat*}{2}
        b &= mq_3 + r_2 \\
        d &= mq_4 + r_2
      \end{alignat*}

      Tenemos entonces

      \begin{alignat*}{2}
        a + b &= m(q_1 + q_3) + (r_1 + r_2) \\
        c + d &= m(q_2 + q_4) + (r_1 + r_2)
      \end{alignat*}

      \noindent Advierta que, ya que $r_1 > 0$ y $r_2 > 0$, se tiene también
      que $r_1 + r_2 > 0$.

      Ahora, se pueden dar dos situaciones. Por un lado, puede que $r_1 +
      r_2 < m$. En ese caso, por el Teorema TKTK, tenemos que

      \begin{alignat*}{2}
        a + b &= m(q_1 + q_3) + (r_1 + r_2) \\
        c + d &= m(q_2 + q_4) + (r_1 + r_2)
      \end{alignat*}

      \noindent serán la divisiones en $\zset$ con resto de $a + b$ y $c +
      d$, respectivamente, siendo $r_1 + r_2$ el resto de ambos y $q_1 +
      q_3$ y $q_2 + q_4$ los cocientes respectivos.

      En el caso de que $r_1 + r_2 \geq m$, lo que se hará será tomar un
      $r_3$ que sea tal que $r_3 = r_1 + r_2 - m$. Tenemos entonces

      \begin{alignat*}{2}
        a + b &= m(q_1 + q_3) + m + r_3 = m(q_1 + q_3 + 1) + r_3 \\
        c + d &= m(q_2 + q_4) + m + r_3 = m(q_2 + q_4 + 1) + r_3
      \end{alignat*}

      \noindent con $0 \geq r_3 < m$. Advierta que no puede darse $r_1 + r_2
      \geq 2m$, ya que $r_1 < m$ y $r_2 < m$. Se tiene entonces que se trata
      de divisiones en $\zset$ con resto con resto de ambos $r_3$ y $q_1 +
      q_3$ y $q_2 + q_4$ los cocientes respectivos.

    \item el caso del producto se demuestra de forma análoga.\qedhere
  \end{enumerate}
\end{proof}



