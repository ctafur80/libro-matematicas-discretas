


Puede consultar \cite{rosen} pág. 102.

Ahora, vamos a ver cómo se calcula el máximo común divisor de dos números.
La forma que sabemos de la educación primaria es factorizando. Así, por
ejemplo, para los números 6 y 14, tenemos que

$$ 6 = 2 \cdot 3 $$

\noindent y

$$ 14 = 2 \cdot 7 $$

\noindent con lo que $\gcd(6, 14) = 2$.

Pero, si el número es muy grande, el proceso se hace inasumible en la
práctica. Existe un algoritmo que lo hace más cómodo. Este tiene su
justificación en una proposición que aparece en el libro \m{Elementos} de
Euclides.

Ahora, vamos a ver una proposición que usaremos para poder demostrar la
existencia del Algoritmo de Euclides. Según esta, el máximo común divisor
del dividendo y divisor de una división en $\zset$ con resto es el mismo que
el del divisor y el resto en esa misma operación.

\begin{proposition}[del Máximo Común Divisor del Divisor y el
  Resto]\label{th-mcd-div-resto}
  Dados $a, b, q, r \in \zset$, cumpliéndose $a = bq + r$ siendo $0 \leq r <
  |b|$. Se cumple $\gcd(a, b) = \gcd(b, r)$.
\end{proposition}

\iffalse
Sea $g \in \zset$ con $g \geq 0$. Entonces, se cumple $g = \gcd(a, b)$ si y
solo si $g = \gcd(b, r)$.

Dados $a, b \in \zset$, tenemos, por el teorema de la división en $\zset$
con resto, que existen $q, r \in \zset$ únicos con $0 \leq r < |b|$ que
cumplen

$$ a = bq + r $$

\noindent Entonces, se cumple que

$$ \gcd(a, b) = \gcd(b, r) $$
\fi

En \cite{weissman} la dan algo más general, al no impone la
restricción de los valores de $r$. Serviría, por tanto, para divisiones
enteras con resto alternativas, como la mínima, por ejemplo.

\begin{proof}
  Designaremos por $d$ al $\gcd(a, b)$. Ya que $d = \gcd(a, b)$, se cumple
  necesariamente que $d \mid a$ y $d \mid b$. Si aislamos a $r$ en la
  igualdad $a = bq + r$, tenemos que $r = a - bq$. Entonces, como $d \mid a$
  y $d \mid b$, por las proposiciones \ref{prop-divide-al-multiplo} y
  \ref{princ-dos-de-tres} se tiene que $d \mid (a - bq)$, o, lo que es lo
  mismo, $d \mid r$. Con esto hemos demostrado que, si $d = \gcd(a, b)$,
  entonces, $d$ es un divisor común de $b$ y de $r$.

  Ahora, suponemos que existe un $c \in \zset$ tal que $c \mid b$ y $c \mid
  r$. Por las mismas proposiciones de antes, se tiene que $c \mid (bq + r)$,
  que es lo mismo que decir que $c \mid a$, por la ecuación anterior. Por lo
  tanto, $c$ es también, al igual que $d$, un divisor común de $a$ y de $b$.

  Pero, como por hipótesis hemos fijado la condición $d = \gcd(a, b)$,
  entonces, al demostrar que $c$ es un divisor común de $a$ y de $b$, tiene
  que darse necesariamente, por la definición de máximo común divisor, $c
  \mid d$. A su vez, si analizamos la relación de $c$ y $d$ como divisores
  de $b$ y $r$, tenemos que, como $c \mid d$ (cosa que acabamos de
  demostrar), se tendrá que $d = \gcd(b, r)$.
\end{proof}

\iffalse
Esta demostración está sacada de \cite{burton}, en la sección
2.4 \engm{The Euclidean Algorithm}, página 27. Aquí
\url{https://www.youtube.com/watch?v=rxL6nhYgI3E} hay otra demostración,
pero tiene un fallo y no sé si fiarme. Además, usa un teorema que no sé
demostrar. TKTK.
\fi

Si vamos encadenando esto mismo, tenemos entonces una forma de obtener el
$\gcd(a, b)$. Esto sería el Algoritmo de Euclides para la división. Está muy
bien explicado en \cite{texto-uned} en la pág. 21.

Partimos de la sexta de las proposiciones \ref{prop-mcd}, que dice que
$\gcd(a, b) = \gcd(|a|, |b|)$. Así, podemos suponer, sin pérdida de
generalidad, que $a \geq b > 0$.

Lo primero será dividir a $a$ por $b$, según la división en $\zset$ con
resto:

$$ a = bq_1 + r_1 \quad \text{con} \ 0 \leq r_1 < |b| = b $$

Si $r_1 = 0$, entonces $b \mid a$ y, por la cuarta de las propiedades
\ref{prop-mcd}, se tiene $\gcd(a, b) = b$, y ahí finalizaría el algoritmo al
tener ya la respuesta que buscábamos. Si, por el contrario, $r_1 \neq 0$,
procedemos a dividir al dividendo, $b$, entre el resto, $r_1$. Tenemos,

$$ b = r_1q_2 + r_2 \quad \text{con} \ 0 \leq r_2 < |r_1| = r_1 < b $$

Por lo mismo de antes, si $r_2 = 0$, entonces $r_1 = \gcd(b, r_1)$, pero,
por la proposición \ref{th-mcd-div-resto}, se tiene que $\gcd(a, b) =
\gcd(b, r_1)$, con lo que se tendrá que $\gcd(a, b) = r_1$.

Si, por el contrario, se tiene que $r_2 \neq 0$, pasamos a dividir a $r_1$
entre $r_2$.

$$ r_1 = r_2q_3 + r_3 \quad \text{con} \ 0 \leq r_3 < |r_2| = r_2 < r_1 < b
$$

Si $r_3 = 0$, entonces $r_2 = \gcd(r_1, r_2)$, pero también se tiene que
$\gcd(r_1, r_2) = \gcd(r_1, b) = \gcd(a, b)$, con lo que se tendrá que
$\gcd(a, b) = r_2$.

Si, por el contrario, se tiene que $r_3 \neq 0$, pasamos a dividir a $r_2$
entre $r_3$.

$$ r_2 = r_3q_4 + r_4 \quad \text{con} \ 0 \leq r_4 < |r_3| = r_3 < r_2 <
r_1 < b $$

\noindent Y así sucesivamente. El proceso continuaría hasta que se tenga un
resto $r_n = 0$ para algún $n \in \mathbb{N}^{+}$. Pero, ¿no podría darse el
caso de estar buscando indefinidamente ese resto con valor 0? Si se fija,
eso no puede suceder ya que, según la condición $0 \leq r_k < |r_{k-1}| =
r_{k-1}$ que se daría en la iteración $k$-ésima, en cada iteración ese resto
se acerca al 0, sin detenerse nunca, al tener una desigualdad estricta entre
$r_k$ y $r_{k-1}$. Por tanto, este algoritmo tiene siempre un número finito
de iteraciones y, por tanto, un número finito de pasos.

Analizando un caso genérico $n$-ésimo, se puede decir que, cuando en la
iteración $n$-ésima se obtenga un resto de 0, es decir, $r_n = 0$, se tendrá
que $\gcd(a, b) = r_{n-1}$. Este se puede obtener, en el resto de la
penúltima expresión (de división) o, si lo prefiere, del divisor en la
última.

Esta demostración también puede verla en
\url{https://www.youtube.com/watch?v=8cikffEcyPI}.

Puede ver en \cite{weissman} al comienzo del capítulo 1, pág. 25
y siguientes, una explicación muy creativa e intuitiva de lo que significa
este proceso, el algoritmo de Euclides. Lo que viene a explicar es que el
máximo común divisor de dos números nos indica a qué números nos podemos
mover haciendo combinaciones de saltos y a cuáles no. Y, si se fija, esto
mismo es lo que nos dice la Identidad de Bézout, es decir, con las
combinaciones lineales enteras $ax + by$ nos podremos mover por los
distintos valores de los múltiplos de $\gcd(a, b)$. Otro ejemplo curioso
sobre la aplicación de esta propiedad se encuentra en
\cite{leighton} sección 9.1.3 Die Hard pág. 344. Está sacado de un
problema que se les plantea a los protagonistas de la película \m{La Jungla
de Cristal 3}. TKTK.

En realidad, este algoritmo, como todos, lo normal es implementarlo en
\engm{software}. No obstante, para esta asignatura tenemos que saber hacerlo
con lázpiz y papel.

Una implementación en código Python sería la siguiente. TKTK.

\begin{verbatim}
# Da el máximo común divisor de dos números enteros a y b.

a = 12384
b = 4720

dividendo = a
divisor = b

resto = 0

while True:

    cociente = dividendo // divisor
    resto_ant = resto
    resto = dividendo % divisor

    # print(f"{dividendo} = {divisor} x {cociente} + {resto}")

    if resto == 0:
        mcd = resto_ant
        break

    dividendo = divisor
    divisor = resto

print()
print(f"mcd({a}, {b}) = {mcd}")
\end{verbatim}

Existe también el Algoritmo de Euclides Extendido. En este, además de
obtener el máximo común divisor, se obtienen los coeficientes de Bézout. Se
implementaría del modo siguiente en código Python. TKTK.

\begin{proposition}[del Escalado]\label{teor-mcd-multiplos}
  Dado un $k \in \nset$ siendo $k > 0$. Se cumple que $\gcd(ka, kb) = k \,
  \gcd(a, b)$.
\end{proposition}

La demostración siguiente es la que aparece en \cite{texto-uned} pág. 24 y
en \cite{burton} pág. 29. Hace uso del Algoritmo de Euclides. No sé si
existirá una demostración que no requiera de dicho proceso iterativo. Si no
la tienen estos textos, seguramente es que no la haya encontrado nadie aún.

\begin{proof}
  Básicamente, lo que vamos a hacer es aplicar el Algoritmo de Euclides.

  Nos quedamos con $a > 0$ y $b > 0$, sin pérdida de generalidad, pues
  $\gcd(a, b) = \gcd(|a|, |b|)$.

  Supongamos primero el caso $b \mid a$. En dicho caso, como sabemos por la
  proposición TKTK, $\gcd(a, b) = b$. Por tanto,

  $$ k \, \gcd(a, b) = kb $$

  Por otro lado, como consecuencia de $b \mid a$ también se da que $kb \mid
  ka$, se tiene que

  $$ \gcd(ka, kb) = kb $$

  \noindent y ahí se habría terminado la demostración.

  Caso $b \nmid a$. Se tiene también $kb \nmid ka$ (cosa que es muy fácil
  demostrar por su condicional contrarrecíproco) y, entonces,

  $$ ka = kbq + r'  \quad \text{con} \ 0 \leq r' < |kb| = kb $$

  \noindent en realidad, $0 < r'$, pero eso da igual.

  Aislando a $r'$ en la expresión, tenemos

  $$ r' = ka - kbq = k(a - bq) $$

  \noindent con lo que $k \mid r'$. Por tanto, existe un $r \in \zset$ tal
  que $r' = kr$.

  Así, manipulamos las desigualdades:

  $$ 0 \leq kr < kb $$

  \noindent y, por tanto,

  $$ 0 \leq r < b $$

  Así, tenemos

  $$ ka = kbq + kr  \quad \text{con} \ 0 \leq r < b $$

  Ahora, pasaríamos a una nueva iteración del Algoritmo de Euclides.
  Analizamos primero la hipótesis de que se dé que $kr \mid kb$. En ese
  caso, $\gcd(kb, kr) = kr$. Veamos por qué.

  Si $kr \mid kb$, se cumple también que $r \mid b$ y, por tanto, $\gcd(b,
  r) = r$, con lo que $k \, \gcd(b, r) = kr$. Uniendo entonces ambas
  expresiones, tenemos que $\gcd(kb, kr) = k \, \gcd(b, r)$ y ahí se
  terminaría la demostración.

  En caso contrario, es decir, $kr \nmid kb$, tendríamos que existen un par
  de números $q_1, r'' \in \zset$ únicos tales que

  $$ kb = krq_1 + r'' \quad \text{con} \ 0 \leq r'' < |kr| = kr $$

  Entonces, al igual que antes, aislando la $r''$, tenemos

  $$ r'' = kb - krq_1 = k(b - rq_1) $$

  \noindent con lo que $k \mid r''$. Entonces existe un $r_1 \in \zset$ tal
  que $r'' = k r_1$. Entonces tenemos

  $$ 0 \leq kr_1 < kr $$

  \noindent y, por tanto,

  $$ kb = krq_1 + kr_1 \quad \text{con} \ 0 \leq r_1 < r $$

  Ahora, pasaríamos a una neva iteración del Algoritmo. Si $kr_1 \mid kr$,
  entonces $\gcd(kr, kr_1) = kr_1$. Por otro lado, si $kr_1 \mid kr$,
  entonces $r_1 \mid r$ y, por tanto, $\gcd(r, r_1) = r_1$. Uniendo ambas,
  tenemos que $k \, \gcd(r, r_1) = kr_1 = \gcd(kr, kr_1)$.

  En caso de que $kr_1 \nmid kr$, tenemos que

  $$ kr = kr_1q_2 + r''' \quad \text{con} \ 0 \leq r''' < |kr_1| = kr_1 $$

  Entonces, al igual que antes, aislando la $r'''$, tenemos

  $$ r''' = kr - kr_1q_2 = k(r - r_1q_2) $$

  \noindent con lo que $k \mid r'''$. Entonces existe un $r_2 \in \zset$ tal
  que $r''' = k r_2$. Entonces tenemos

  $$ 0 \leq kr_2 < kr_1 $$

  \noindent y, por tanto,

  $$ kr_1 = kr_1q_2 + kr_2 \quad \text{con} \ 0 \leq r_2 < r_1 $$

  Ahora, pasaríamos a una neva iteración del Algoritmo. Si $kr_2 \mid kr_1$,
  entonces $\gcd(kr_1, kr_2) = kr_2$. Por otro lado, si $kr_2 \mid kr_1$,
  entonces $r_2 \mid r_1$ y, por tanto, $\gcd(r_1, r_2) = r_2$. Uniendo
  ambas, tenemos que $k \, \gcd(r_1, r_2) = kr_2 = \gcd(kr_1, kr_2)$.

  En caso de que $kr_2 \nmid kr_1$, se procedería como en las iteraciones
  anteriores. Existirá alguna en la que se termine por obtener un resto
  $r_{n+1} = 0$, al igual que en la demostración del Algoritmo de Euclides,
  ya que las desigualdades siguientes son estrictas:

  $$ 0 \leq \cdots < r_2 < r_1 < r $$

  Así, en la última iteración se tendrá

  $$ kr_{n-1} = (kr_n)q_{n+1} + 0 $$

  \noindent Se tiene entonces, por el algoritmo de Euclides, que $kr_n =
  \gcd(ka, kb)$. Por otro lado, como $r_n = \gcd(a, b)$, se tiene, por
  tanto, que

  $$ \gcd(ka, kb) = k \, \gcd(a, b) $$

  \noindent tal y como queríamos demostrar.

  Otra demostración, más breve, es haciendo uso del Corolario
  \ref{id-bezout-2}. Supongamos se tienen $m, n \in \zset$ para los que se
  tiene la menor combinación lineal positiva de $a$ y $b$, $am + bn$. Se
  tiene entonces que, para un $k \in \zset$,

  $$ \gcd(a, b) = am + bn $$

  \noindent Multiplicándolo por $k$, se tiene

  $$ k \, \gcd(a, b) = k(am + bn) = kam + kbn = ka(m) + kb(n) = \gcd(ka, kb)
  $$
\end{proof}

En realidad, se tiene una generalización del teorema anterior.

\begin{corollary}
  Se cumple, para cualquier $k \in \zset$,

  $$ \gcd(ka, kb) = |k| \, \gcd(a, b) $$
\end{corollary}

\begin{proof}
  Para el caso en el que $k > 0$, se demuestra por el teorema anterior. Para
  el caso en el que $k < 0$, se tiene ${-k} > 0$ y hacemos lo siguiente:

  $$ \gcd(ka, kb) = \gcd({-ka}, {-kb}) = {-k} \gcd(a, b) = |k|
  \, \gcd(a, b)$$

  La demostración del caso $k = 0$ es trivial.
\end{proof}

El corolario siguiente se usará mucho en demostraciones de las secciones
siguientes.

\begin{corollary}\label{cor-mcd-div-mcd}
  Dados $a, b \in \zset$ y $g = \gcd(a, b)$ siendo $g \neq 0$. Entonces,

  $$ \gcd(a/g, b/g) = 1 $$
\end{corollary}

% TODO Quizás se pueda poner que a y b sean distintos de 0 y quitar la
% excepción para g, pues creo que mcd(a, b) = 0 solo cuando a = b = 0.

\begin{proof}
  Dividiento entre $g$ en ambas partes de la igualdad $g = \gcd(a, b)$
  tenemos:

  \begin{alignat*}{2}
    \frac{g}{g} &= \frac{1}{g}\gcd(a, b) \\
    1 &= \gcd(a/g, b/g) \\
  \end{alignat*}

\noindent La última transformación es por una aplicación de la Proposición
\ref{teor-mcd-multiplos}.
\end{proof}
