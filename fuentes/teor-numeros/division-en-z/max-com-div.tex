




\begin{deffinition}[Divisor común de dos números]
  Dados $a, b \in \zset$. Un $d \in \zset$ tal que $d \mid a$ y $d \mid b$
  se dice que es un \emph{divisor común} de $a$ y de $b$.
\end{deffinition}

Se me ocurre la notación $d \mid \{a, b\}$ o, también, $\divset(a, b)$.

Algo que le puede resultar curioso es que es mucho más fácil encontrar
divisores comunes de dos números que divisores de estos por separado. Vea
\cite{stillwell-elem-maths} pág. 36.

\begin{properties}
  Dados $a, b, d \in \zset$.

  \begin{enumerate}
    \item El conjunto de divisores comunes de $a$ y $b$ no es vacío. Es
      decir,

      $$ \left\{d \in \zset \st d \mid a \ \text{y} \ d \mid b\right\} \neq
      \emptyset $$

    \item A menos que $a = b = 0$, $0 \nmid \{a, b\}$. Es decir, el 0 no es
      divisor común de dos números salvo si ambos valen 0.

    \item Siempre que no se dé que $a = b = 0$, el conjunto de divisores
      comunes de $a$ y $b$ será finito.
  \end{enumerate}
\end{properties}

\begin{proof}
  -
  \begin{enumerate}
    \item Se debe a que, para todo $n \in \zset$, se cumple que $1 \mid n$,
      así que el conjunto contendrá al menos al 1.

    \item Como el 0 no divide a ningún número distinto de 0, tampoco
      dividirá conjuntamente a $a$ y a $b$, salvo si ambos valen 0.

    \item Podríamos expresarlo de forma conjuntista. Tenemos que

      $$ \divset(a, b) = \{x \in \zset \st x \mid a \ \text{y} \ x \mid b\}
      = \{x \in \zset \st x \mid a\} \cap \{x \in \zset \st x \mid b\} =
      \divset(a) \cap \divset(b) $$

      Tal y como vimos, si un número $a \in \zset$ es distinto de 0,
      entonces $\divset(a)$ será finito. Lo mismo para $b$. Además, la
      intersección de dos conjuntos finitos ha de ser finita, por lo que
      $\divset(a, b)$ será finito.

      Veamos ahora los casos extraños. Si alguno de ambos es igual a 0, pero
      el otro no, por ejemplo, $a = 0$ y $b \neq 0$, se dará que
      $\divset(a)$ será un conjunto infinito pero no $\divset(b)$. La
      intersección de ambos conjuntos nos da $\divset(b)$.

      El caso en el que sería infinito sería cuando $a = b = 0$; de ahí que
      lo hayamos excluido.\qedhere

    \iffalse
    \item Fijémonos primero en una sola de las variables. Supongamos que $a
      \neq 0$. Tenemos que existe algún $d \in \zset$ para el que $d \mid
      a$. Entonces, por la proposición \ref{prop-factor-men}, se tendrá que
      $|d| \leq |a|$, o, lo que es lo mismo, ${-a} \leq q \leq a$. Es decir,
      $d \in [{-a}, a]_{\zset}$, intervalo que consta de un número finito de
      elementos.

      En cualquier caso, tendremos que hallar la intersección de este
      conjunto con el de los divisores comunes de la otra variable.

      Si la otra variable, $b$, vale 0, el conjunto de divisores de esta
      será todo $\zset$, como ya sabemos. Así, la intersección de ambos
      conjuntos de divisores será

      $$ q \mid \{a, b\} = \zset \cap [{-a}, a]_{\zset} = [{-a}, a]_{\zset}
      $$

      Si, por el contrario, $b$ es, al igual que $a$, distinto de 0, se
      hallará la intersección de ambos conjuntos, que serán ambos
      intervalos.

      $$ q \mid \{a, b\} = [{-a}, a]_{\zset} \cap [{-b}, b]_{\zset} $$

      \noindent cosa que habría que averiguar para cada caso en concreto; es
      decir, no podemos averiguar más, de momento, sobre el conjunto
      intersección. Lo que sí sabemos es que será finito, al ser la
      intersección de dos conjuntos finitos.
    \fi
    \end{enumerate}
\end{proof}

\begin{deffinition}[Máximo común divisor]
  Dados $a, b \in \zset$ y un $d \in \nset$ que sea divisor común de $a$ y
  $b$. Si para cualquier otro divisor común $d'$ de $a$ y $b$ se cumple que
  $d' \mid d$, a $d$ se le llamará \m{máximo común divisor} de $a$ y $b$,
  cosa que también se denota como $\gcd(a, b)$.
\end{deffinition}

Advierta que hemos mencionado que es un divisor tal que $d \in \nset$. Esto
es perfectamente factible. Imagine que hubiésemos puesto que $d \in \zset$.
Al ser $d$ un número tal que $d \mid a$ y $d \mid b$, se tiene, tal y como
vimos antes, que ${-d} \mid a$ y ${-d} \mid b$. Además, como ya vimos, si $d
\geq 0$, entonces ${-d} \leq 0$, y viceversa. Por tanto, podemos seleccionar
al que deseemos de $\{{-d}, d\}$ y usarlo en la definición.

De hecho, si hubiésemos especificado $d \in \zset$, se tendrían dos
soluciones igualmente válidas: $\{{-d}, d\}$.

En algunos libros usan la misma notación que los pares ordenados, es decir,
algo como $(a, b)$, para denotar a lo que aquí designaríamos por $\gcd(a,
b)$. Me parece que esa notación puede dar lugar a confisión en ciertos casos
y, por tanto, prefiero no usarla.

Veamos ahora algunas propiedades del máximo común divisor.

\begin{proposition}\label{prop-mcd}
  Dados $a, b \in \zset$, se cumple

  \begin{enumerate}
    \item $\gcd(0, 0) = 0$.
    \item Si $a \neq 0$, $\gcd(a, 0) = |a|$.
    \item $\gcd(a, b)$ es único.
    \item Si $a \mid b$, entonces $\gcd(a, b) = a$.
    \item $\gcd(a, b) = \gcd(|a|, |b|)$.
  \end{enumerate}
\end{proposition}

\begin{proof}
  -
  \begin{enumerate}
    \item La decisión de admitir que $0 \mid 0$ nos obliga a admitir también
      que $\gcd(0, 0) = 0$. Veamos por qué. Por un lado, 0 es un divisor
      común de 0 y 0. Pero, además, cualquier otro número divide a 0.

      \iffalse
      Alternativamente, se puede razonar como que 0 es el único múltiplo de
      0.
      \fi

    \item Podemos operar de forma similar con conjuntos, igual que antes:

      \[ \divset(a, 0) = \{d \in \zset \st d \mid a \ \text{y} \ d \mid 0\}
      = \{d \in \zset \st d \mid a\} \cap \zset = = \{d \in \zset \st d \mid
      a\} \]

      Ahora, tendríamos en cuenta los dos casos posibles: $a > 0$ y $a < 0$.
      En el primero, se tiene que cualquier divisor de $a$ es divisor de
      $a$, con lo que $\gcd(a, 0) = a$. En el otro, por esto mismo,
      $\gcd({-a}, 0) = {-a}$. Por tanto, $\gcd(a, 0) = |a|$

      Por cierto, sería lo mismo para $\gcd(0, a)$, ya que el orden de los
      argumentos no afecta al resultado de esta función.

    \item Para los casos en los que interviene el 0, se demostró en los
      puntos anteriores. En los otros casos, supongamos que se tienen dos
      números, $g_1$ y $g_2$, que son, de forma independiente, máximo común
      divisor de dos números dados, es decir, de $a$ y $b$.

      Atendiendo a la definición de máximo común divisor, por un lado se
      tendrá que $g_1 \mid g_2$, pero, desde el punto de vista contrario, se
      tendrá que $g_2 \mid g_1$.

      Al tratarse de casos de máximo común divisor, los números $g_1$ y
      $g_2$ serán necesariamente mayores que 0. Por lo tanto, por la
      Propiedad \ref{prop-factor-men}, tendremos, por un lado, $g_1 \leq
      g_2$, pero, por el otro, $g_2 \leq g_1$. Por tanto, se tiene que $g_1
      = g_2$, con lo que queda demostrado que $\gcd(a, b)$ tiene que ser
      único.

      Por cierto, como quizás sepa, al ser único y existir para cualquier
      combinación de valores de entrada, es una función. TKTK.

    \item La demostración es trivial ya que cualquier $d \in \zset$
      arbitrario que sea divisor común de $a$ y $b$ es evidente que divide a
      $a$.

      El caso particular en el que convendría fijarse sería cuando $a \neq
      0$ y $b = 0$. En ese caso, se da que, por la parte de $b$, se aporta
      todo $\mathbb{Z}$, y, por la de $a$, los divisores de $a$. La
      intersección de ambos nos da un conjunto cuyo elemento mayor será $a$.

    \item Se demuestra por una de las propiedades \ref{propi-divide-2}.
      Concretamente, de la que dice que, si $a \mid b$, entonces $a \mid
      ({-b})$.
      \qedhere
  \end{enumerate}
\end{proof}

La propiedad 4, aunque parezca muy simple y evidente, se usa con bastante
frecuencia en demostraciones posteriores.

Seguramente le enseñaron otra definición en la que se decía que $d' \leq d$,
en lugar de $d' \mid d$. Esa otra definición es también equivalente a esta.
El problema está en que se está haciendo uso de una relación de orden,
$\leq$, distinta a ``divide a'', que es también una relación de orden. Por
tanto, se podría decir que nuestra definición es ``autosuficiente'' y la
otra no lo es.

Tal y como se explica en \cite{burton} pág. 24, esto permite usarse también
en sistemas algebraicos en los que que no se tenga la relación $\leq$, con
lo que se trata de una definición más amplia.

En los textos de teoría de números, a nuestra definición la suelen calificar
de la ``moderna''.

Una ventaja con la que contaría la definición antigua es que en esta no es
necesario especificar que el máximo común divisor sea un número natural,
pues, como los factores siempre van en pares de la forma $\pm a$, el mayor
sería el valor absoluto de este, que es un número natural, pero no
tendríamos que haberlo especificado. Además, algunas demostraciones son más
sencillas de hacer con la demostración antigua, pero también pueden hacerse
con la moderna.

En cuanto a la terminología, el sintagma \m{máximo común divisor} es una
traducción del inglés \engm{greatest common divisor}, pero personalmente la
veo bastante ``artificial''. En mi opinión, sería más acertado traducirlo
como \emph{el mayor de los divisores comunes}. En cualquier caso, es una
forma de llamarlo tan asentada que no voy a tratar de llamarlo de otra
forma.

El concepto de máximo común divisor se puede extender a más de dos
argumentos. Vea \cite{rosen} pág. 98.

El siguiente lema es muy importante y, así aislado, quizás usted no advierta
por qué lo es. Conforme avance en la materia, verá que se recurre a este muy
a menudo.

Este lema nos dice que el máximo común divisor de dos números enteros
cualesquiera se puede expresar como combinación lineal (entera) de los
mismos. Más concretamente, se trata de la menor combinación lineal de estos,
como también se verá; pero esto último, más adelante.

\begin{lemma}[Identidad de Bézout\footnotemark]\label{th-bezout}
  Dados $a, b \in \zset$. Existen $x, y \in \zset$ tales que

  $$ \gcd(a, b) = ax + by $$
\end{lemma}

\footnotetext{En realidad, ahora sabemos que el nombre de este lema hace
honor erróneamente al matemático francés Étienne Bézout nacido en 1730. En
realidad, este lema fue publicado por primera vez en Occidente 150 años
antes por alguien, pero lo más sorprendente es que incluso fue descrito unos
1000 años antes que eso por dos matemáticos indios llamados Aryabhata y
Bhaskara.}

A esos valores $x$ y $y$ que aparecen en el lema se les llama
\m{coeficientes de Bézout}. Advierta que en la definición no se menciona que
deban ser únicos. De hecho, lo más frecuente es que no lo sean. A la
ecuación

$$ \gcd(a, b) = ax + by $$

\noindent se la conoce como la \m{identidad de Bézout}.

En \cite{texto-uned} ponen la condición de que $a \neq 0$ y $b \neq
0$, pero en realidad no sería necesario, pues, por un lado, si $a = b = 0$,
la igualdad se cumpliría para cualesquiera $x$ e $y$:

$$ \gcd(0, 0) = 0 = 0x + 0y $$

Para el caso $a \neq 0$ y $b = 0$, como sabemos que $\gcd(a, 0) = |a|$,
tendríamos que se cumpliría tanto para $a > 0$ como si $a < 0$. Los
coeficientes $x$ e $y$ serían, respectivamente, para estos casos, $(1, y)$ y
$({-1}, y)$. Como ve, el valor de $y$ da igual y, por esto mismo,
demostramos que los coeficientes de Bézout no tienen por qué ser únicos.

El caso de $a = 0$ y $b \neq 0$ ya lo hemos contemplado en el anterior, pues
el operador máximo común divisor es conmutativo, es decir, $\gcd(a, b) =
\gcd(b, a)$, propiedad que hereda de la operación ``divisor común de dos
números'', en la que es evidente.

Vamos a ver ahora la demostración del Lema \ref{th-bezout}.

\begin{proof}
  Los casos extraños, en los que aparece algún 0 en $a$ o $b$, ya los hemos
  mencionado antes, con lo que ahora suponemos el caso general en el que
  ninguno es 0.

  Vamos a hacer la demostración en 3 fases, ya que es algo extensa.

  Fase 1. Trataremos de demostrar que el elemento menor del conjunto de las
  combinaciones lineales (enteras) positivas de $a$ y $b$ es un divisor
  común de $a$ y $b$.

  Definimos el conjunto siguiente:

  $$ M = \{ax + by \ \st x, y \in \zset \ \text{y} \ ax + by > 0 \} $$

  \noindent Es decir, $M$ es el conjunto de todas las combinaciones lineales
  positivas de $a$ y $b$. ¿Por qué se ha tomado el conjunto en el que son
  mayores que 0? Porque, cuando $a$ y $b$ no valen 0, su máximo común
  divisor será positivo, tal y como vimos antes.

  $M$ no puede ser vacío ya que bien la combinación $a \cdot 1 + b \cdot 0$
  o bien la $a \cdot ({-1}) + b \cdot 0$ será positiva. Dependerá de si $a$
  es positiva o negativa. Recuerde que estamos excluyendo la posibilidad de
  que $a = 0$.

  Por otro lado, al ser números positivos todos los elementos de dicho
  conjunto, tendrán, por la Propiedad \ref{princ-buena-ord}, un primer
  elemento, es decir, un elemento que es menor o igual que todos los demás
  de $M$. Vamos a designar a este primer elemento por $d$ y, a los valores
  de $x$ e $y$ que lo producen los designaremos por $k$ y $l$,
  respectivamente. Es decir, tenemos en el conjunto $M$ al elemento

  $$ d = ak + bl $$

  Vamos a suponer ahora que ese primer elemento $d$ no divide a $a$, es
  decir, $d \nmid a$. Entonces, aplicando la división en $\zset$ con resto
  (Teorema \ref{th-div-entera-con-resto}), tendremos un único par de $q, r
  \in \zset$ siendo $0 \leq r < |d|$ y

  $$ a = qd + r $$

  \noindent Pero podemos ser más restrictivos con los valores posibles de
  $r$. Por un lado, sabemos que $d > 0$, ya que $d \in M$, por lo que el
  rango de valores de $r$ pasa a ser $0 \leq r < d$. Además, hemos dicho que
  $d \nmid a$, es decir, no es una división exacta, con lo que $r \neq 0$;
  nos queda, entonces, $0 < r < d$.

  Ahora, vamos a operar con $r$. Tenemos que

  $$ r = a - dq = a - q(ak + bl) = a - qak - qbl = a(1 - qk) + b(-ql) $$

  \noindent Es decir, $r$ es una combinación lineal de $a$ y $b$. Además,
  tal y como hemos visto, $r > 0$, con lo que $r$ será también un elemento
  de $M$, es decir, $r \in M$. Pero, por otro lado, también hemos visto que
  $r < d$, y habíamos dicho que $d$ era el elemento menor de $M$, con lo que
  hemos llegado a una contradicción. Esta nos obliga a rectificar la
  hipótesis $d \nmid a$, con lo que, al contrario de como supusimos, $d$, el
  menor elemento de $M$, será entonces un divisor de $a$.

  De forma análoga podemos proceder para deducir que $d \mid b$. Así,
  tenemos que $d$, el menor elemento de $M$, es un divisor común de $a$ y
  $b$, es decir, $d \mid \{a, b\}$.

  Fase 2. Trataremos de demostrar que el menor elemento de las combinaciones
  lineales positivas de $a$ y $b$ (es decir, el mínimo del conjunto $M$) es
  igual a $\gcd(a, b)$.

  Sea $d' \in \zset$ un divisor común de $a$ y $b$, $d' \mid \{a, b\}$.
  Entonces, existen $m, n \in \zset$ tales que $a = md'$ y $b = nd'$.
  Entonces,

  $$ d = ak + bl = md'k + nd'l = d'(mk + nl)$$

  \noindent con lo que $d' \mid d$. Juntando esto con que $d$ es un divisor
  común de $a$ y $b$, cosa que demostramos justo antes, se tiene que $d =
  \gcd(a, b)$.

  Una vez demostrado esto, y teniendo en cuenta que, tal y como demostramos
  en la Proposición \ref{prop-mcd}, el máximo común divisor de dos números
  es único, ya estaría demostrado el lema.
\end{proof}

Además, tal y como veremos, ese $\gcd(a, b)$ es el entero positivo más
pequeño que puede expresarse en la forma $ax + by$. En \cite{texto-uned},
esto último lo incluyen en el Lema \ref{th-bezout}, pero nosotros lo
deduciremos luego por separado. Bueno, en realidad ya se ha demostrado en
esta última demostración, pero se hará luego una vez más.

Se deduce inmediatamente el corolario siguiente. Este en realidad hace
alusión a los números primos relativos, cosa que veremos en el capítulo
siguiente. Basta con que, de momento, sepa que dos números enteros se dice
que son primos relativos si su máximo común divisor es 1.

\begin{corollary}[de la Combinación Lineal de Primos Relativos]%
\label{cor-comb-lin-primos-rel}
  Sean $a, b \in \zset$. Se cumple que $\gcd(a, b) = 1$ si y solo si existen
  $x, y \in \zset$ tales que

  $$ ax + by = 1 $$
\end{corollary}

Es consecuencia directa del Lema \ref{th-bezout}. Este corolario se usa en
varias demostraciones sobre números primos. Viene en \cite{rosen} pág. 97 y
en \cite{burton} pág. 23.

Ahora, vamos a ver un corolario muy importante. Este viene a decir que las
combinaciones lineales enteras de dos números generan a todos los múltiplos
de su máximo común divisor.

\begin{corollary}[de los Múltiplos del Máximo Común
  Divisor]\label{cor-mult-mcd}
  Para cualesquiera $a, b \in \zset$, se da

  $$ \{ax + by \st x, y \in \zset\} = \{k \, \gcd(a, b) \st k \in \zset\} $$
\end{corollary}

También sería un caso de particular relevancia la situación $a = b = 0$, y
por eso en muchos textos verá que excluyen ese caso. Pero, si lo analizamos,
podemos incluirlo sin problema. Si sustituimos los valores, vemos que ambos
conjuntos producen el conjunto $\{0\}$.

\begin{proof}
  \iffalse
  Llamamos $d$ al $\gcd(a, b)$. Por el Lema \ref{th-bezout}, tenemos que

  $$ d = ax + by $$

  \noindent para algún par de números $x, y \in \zset$. Multiplicando por un
  $k \in \zset$ en ambas partes de la igualdad tenemos:

  $$ kd = k(ax + by) = akx + bky = a(kx) + b(ky) $$

  Ahora, podemos aplicar un par de cambios de variables y pasar a llamar $x$
  a $kx$ e $y$ a $ky$, ya que la multiplicación de $(\zset, +, \cdot)$ es
  una operación interna. Por tanto,

  $$ k \gcd(a, b) = ax + by $$

  % Hay algún conflicto con las variables nuevas? Creo que no. TKTK.
  \fi

  Vamos a demostrar la igualdad mediante la doble inclusión.

  Primero, vamos a especificar algunas variables para ahorrarnos escribir
  demasiado. Por un lado, tenemos $d = \gcd(a, b)$. También, definimos los
  dos conjuntos siguientes:

  $$ T = \{ax + by \st x, y \in \zset\} \quad \text{y} \quad M = \{k \,
  \gcd(a, b) \st k \in \zset\} $$

  Para cualquier combinación lineal entera de $a$ y $b$, como $d \mid a$ y
  $d \mid b$, según las propiedades \ref{prop-divide-al-multiplo} y
  \ref{princ-dos-de-tres} se tendrá que $d \mid (ax + by)$. Así pues, un
  elemento arbitrario de $T$ será necesariamente también elemento de $M$, o,
  lo que es lo mismo, $T \subseteq M$.

  Ahora, trataremos de demostrar el subconjunto en el otro sentido. Según el
  Lema \ref{th-bezout}, se tiene

  $$ d = ax + by $$

  \noindent para algún par de números $x, y \in \zset$. Multiplicando por un
  $k \in \zset$ arbitrario, tenemos

  $$ kd = k(ax + by) = a(kx) + b(ky) $$

  \noindent con lo que $kd$ es una combinación lineal entera de $a$ y $b$.
  Es decir, hemos demostrado que todo múltiplo de $\gcd(a, b)$, es decir,
  todo elemento de $M$, será también un elemento de $T$. Por tanto, $M
  \subseteq T$.

  Como consecuencia de $T \subseteq M$ y $M \subseteq T$, se tiene que $T =
  M$.
\end{proof}

Es la misma teoría que se muestra en \cite{burton} pág. 22.

De este corolario podemos deducir algo interesante. Vamos a tratar de
averiguar cuál sería el menor $m$ positivo.

\begin{corollary}\label{id-bezout-2}
  Dados $a, b \in \zset$. $\gcd(a, b)$ es el menor entero positivo que se
  puede expresar como una combinación lineal entera de $a$ y $b$.
\end{corollary}

\begin{proof}
  El caso con $a = b = 0$ lo excluiríamos, pues en este caso el máximo común
  divisor vale 0 y no se puede generar, por tanto, un entero positivo.

  Si $a \neq 0$ y $b = 0$, tenemos, por un lado, que $\gcd(a, 0) = |a|$. La
  combinación lineal que le correspondería, sería, por tanto, $ax + 0y$ o
  $a({-x}) + 0y$, según sea positivo o negativo $a$. En ambos casos, el
  valor que tomase $y$ sería indiferente, al estar multiplicada por 0. El
  otro caso, $a = 0$ y $b \neq 0$ sería análogo a este.

  Caso $a \neq 0$ y $b \neq 0$. Se tiene siempre que $\gcd(a, b) > 0$. Como,
  según el Corolario \ref{cor-mult-mcd}, la combinación lineal entera genera
  todos los múltiplos de $\gcd(a, b)$, el primero será este.
\end{proof}

\iffalse
No sé si poner o quitar la excepción del 00. Para este caso no es entero
positivo, sino 0. TKTK.
\fi

De todos modos, esto ya se demuestra también en la demostración del Lema
\ref{th-bezout}.






