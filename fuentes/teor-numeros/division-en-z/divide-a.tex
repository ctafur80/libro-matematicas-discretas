


\begin{deffinition}[Múltiplo]\label{def-multiplo}
  Dados $a, k \in \zset$. Al número $a \cdot k$ se le conoce como
  \m{múltiplo} de $a$.
\end{deffinition}

Advierta que también se podría decir, en ese caso, que $a \cdot k$ es
múltiplo de $k$.

A partir de la operación múltiplo, se puede definir la operación ``divide a'',
que sería su inversa, es decir, la que deshace la acción de esta. Se podría
decir que un número divide a otro si este último es múltiplo del primero. Más
bien, se debería haber usado el sintagma ``si y solo si'', ya que son
definiciones equivalentes. En cualquier caso, aquí se puede ser algo impreciso,
pues no tiene excesiva relevancia.

\begin{deffinition}[Operador ``divide a'']
  Dados $a, b \in \zset$, decimos que ``$a$ divide a $b$'', y lo escribimos
  como $a \mid b$, si y solo si existe algún $d \in \zset$ con el que se
  cumpla

  $$ b = d \cdot a $$
\end{deffinition}

\noindent Es decir, si $b$ es múltiplo de $a$, tal y como dijimos.

Advierta que ese $d$ no tiene por qué ser único. Lo opuesto, es decir, si
para todo $d \in \zset$ se cumple que $b \neq a \cdot d$, se dice que $a$ no
divide a $b$, cosa que se suele designar como $a \nmid b$.

Existen otras formas de llamar a esto mismo. Si $a \mid b$, también se dice
que ``$a$ es un factor de $b$'' o que, tal y como ya hemos visto, que ``$b$
es un múltiplo de $a$''.

Advierta que el operador ``divide a'' es un operador buleano (en inglés,
\engm{Boolean}); el resultado será, por tanto, \m{verdadero} o \m{falso}. No
lo confunda con el valor de esa variable que aquí hemos designado por $d$.

\begin{properties}\label{propi-divide-1}
  Para cuaquier $a \in \zset$, se cumple:

  \begin{enumerate}
    \item $0 \mid 0$.
    \item $a \mid 0$. Es decir, todo número $a$ divide a 0.
    \item Dado $a \neq 0$, se cumple $0 \nmid a$. Es decir, ningún número
      distinto de 0 es múltiplo de 0.
    \item $1 \mid a$. Es decir, el 1 divide a todo número.
    \item $a \mid a$. Es decir, todo número se divide a sí mismo.
  \end{enumerate}
\end{properties}

En lo poco que llevamos de capítulo, he tomado dos decisiones que me harán,
si deseo mantener coherencia en el resto del texto, poner las demás
definiciones, teoremas, etc. algo distintas a las de textos clásicos como,
por ejemplo, \cite{burton}. Estoy siguiendo esta regla
imitando lo que se hace en \cite{weissman}. Concretamente, lo
que origina todo el cambio es no poner, en la definición del operador
``divide a'', una excepción excluyendo el caso $a = 0$ y, en estas
propiedades, admitir que $0 \mid 0$.

Lo que debe tener en cuenta es que, en ciertas situaciones en las que se
tiene una proposición (en forma de proposición, teorema, lema, etc.) con un
operador lógico condicional, algunos casos serán vacuamente ciertos
(\engm{vacuously true}). TKTK.

Independientemente de lo que se diga aquí para este operador, seguramente le
habrían enseñado en cursos previos que el 0 no divide a ningún número,
incluido al 0. TKTK. Si está más interesado en la posibilidad de que 0 pueda
dividir a 0, puede ver el vídeo del siguiente enlace web:
\url{https://www.youtube.com/watch?v=5WbK9O9JNDk}.

\begin{proof}
  -

  \begin{enumerate}
    \item Para todo $q \in \zset$ se cumple $0 = 0q$, con lo que será
      cierto que $0 \mid 0$.

    \item Se tiene que cumplir $0 = aq$ para algún $q \in \zset$. Si $a =
      0$, es cierto por ser como el punto anterior. Si $a \neq 0$, tomamos
      $q = 0$ y se cumple.

    \item Es fácil demostrar que no existe ningún $q \in \zset$ para el que
      se cumpla $a = 0q$, pues $0q = 0$ para cualquier $q$ y nos queda $a =
      0$, pero esa posibilidad la hemos excluido.

    \item Se tiene que cumplir $a = q1$. Tomando $q = a$ se cumple.

    \item Se tiene que cumplir $a = qa$. Tomando $q = 1$ se
      cumple.\qedhere
    \end{enumerate}
\end{proof}

Es fácil resolver un ejercicio cuando nos piden comprobar, para un caso que
sea cierto, si un número divide a otro. Sin embargo, si es un ejericio en el
que resulta que no es cierto, ¿cómo se comprueba? ¿Se prueba para todos los
casos? Es imposible, pues habría que probar con infinitos números. Bueno, sí
que existe una forma. Veámoslo con un ejemplo.

Queremos saber si 2 divide a 5. Por un lado, vamos a ver si se cumple para
4, en lugar de 5. Se cumple ya que $2 \cdot 2 = 4$. De hecho, en general, se
tiene siempre que

$$ \text{Para todo $m \in \zset$ con $m \leq 2$}. \quad 2m \leq 4 $$

\noindent Esto se debe a que la función $f(m) = 2m$ con el dominio en
$\zset$ es una función monótona estrictamente creciente. TKTK.

Por otro lado, se tiene siempre que

$$ \text{Para todo $m \in \zset$ con $m \geq 3$}. \quad 2m \geq 6 $$

\noindent por una razón análoga a la del otro caso.

Así, puesto que $m$ no puede tomar un valor en $\zset$ entre 2 y 3, todos
los valores de la función $2m$ se irán por encima de 6 o por debajo de 4,
por lo que nunca podrá ser 5. Por tanto, $2 \nmid 5$.

En lo que respecta a las reglas de prioridad en la notación, por lo que veo,
parece que en los libros siguen como regla de estilo dar más prioridad al
operador ``divide a'' que a la multiplicación, pero no así con la suma. Por
lo que puede ver sin necesidad de usar paréntesis, cosas como $p \mid a
\cdot b$ o $p \mid a \, b$, para designar que $p$ divide al producto de $a$
y $b$, pero, si en lugar del producto es la suma, sí se usarán paréntesis,
como en la expresión $p \mid (a + b)$.

Veamos algunas propiedades más del operador ``divide a''.

\begin{properties}\label{propi-divide-2}
  Dados $a, b, c \in \zset$, se cumple:

  \begin{enumerate}
    \item Si $a \mid b$, entonces ${-a} \mid b$. También, $a \mid ({-b})$ y
      $({-a}) \mid ({-b})$. Debido a esto, todos los factores de un número
      $b$ arbitrario vienen en pares de la forma ${\pm a}$.
    \item Si $a \mid b$ y $b \mid c$, entonces $a \mid c$. Esta es la
      propiedad transitiva del operador ``divide a''.
    \item Si $a \mid b$ y $b \mid a$, entonces o bien $a = b$ o bien $a =
      {-b}$.
  \end{enumerate}
\end{properties}

La penúltima de estas propiedades es la transitiva, y, la última, es algo
parecido a la antisimétrica, aunque no llega a serlo, pues se admite que
también se puede dar la posibilidad $a = {-b}$. Para que sí se dé la
antisimétrica, podríamos restringirnos a los números naturales, en lugar de
los enteros. Entonces, con estas dos propiedades, junto con la reflexiva,
que se vio antes que también se cumple para todo número, se tiene que el
operador ``divide a'' es una relación de orden, al igual que sucede con el
operador ``menor o igual que'' (`$\leq$'). Eso sí, al contrario de lo que
sucede con el operador ``menor o igual que'', el operador ``divide a'' es un
orden parcial, no total.

\begin{proof}
  -

  \begin{enumerate}
    \item Como se da que $a \mid b$, existe un $q \in \zset$ para el que

      $$ b = aq $$

      Entonces,

      $$ b = aq = (a \cdot 1)q = \left[({-a})({-1})\right]q =
      ({-a})\left[({-1})q\right] = ({-a})({-q}) $$

      \noindent Como ${-q} \in \zset$, se da entonces que ${-a} \mid b$.

      Esto engloba como casos particulares los casos $a = b = 0$, por un
      lado, y $a = 0$ y $b \neq 0$, por el otro. Se comprueba fácilmente
      que, pare el primero, se cumple, ya que ${-0} = 0$, y, para el
      segundo, será vacuamente cierta, ya que $0 \nmid b$.

      Los demás casos de este apartado se hacen de forma similar.

    \item Veamos el caso más general, en el que $a \neq 0$, $b \neq 0$ y $c
      \neq 0$.

      Si se da $a \mid b$, se tiene un $q_1 \in \zset$ tal que

      $$ b = a q_1 $$

      Del mismo modo, si $b \mid c$, se tiene un $q_2 \in \zset$ tal que

      $$ c = b q_2 $$

      Entonces, sustituyendo,

      $$ c = (a q_1) q_2 = a (q_1 q_2) $$

      Advierta que, al igual que antes, $q_1 q_2 \in \zset$, por lo que se
      cumplirá necesariamente que $a \mid c$.

      Ahora, habría que analizar los casos en los que aparece algún 0.
      Serían varios, pero en realidad no entorpecen la versión generalista
      de esta propiedad ya que son vacuamente verdaderas. Por ejemplo, para
      el caso con $a \neq 0$, $b = 0$ y $c \neq 0$, el antecedente es falso,
      pues, si se da $a \mid b$, entonces no puede darse $b \mid c$. No voy
      a demostrarlo aquí para todos esos casos.

    \item Veamos primero el caso en el que $a \neq 0$ y $b \neq 0$.

      Como $a \mid b$, se tendrá un $q_1 \in \zset$ tal que

      $$ b = a q_1 $$

      Por el otro lado, como $b \mid a$, se tendrá un $q_2 \in \zset$ tal
      que

      $$ a = b q_2 $$

      Manipulando las expresiones, llegamos a que

      $$ a = a(q_1 q_2) $$

      Por tanto, $q_1 q_2 = 1$. Para que se dé esto, existen solo dos
      posibilidades: que $q_1 = q_2 = 1$ o bien que $q_1 = q_2 = {-1}$.
      (Esto quizás requiere de una demostrción. TKTK).

      Luego, simplemente con sustituir el valor de $q_1$ en la ecuación
      anterior $b = a q_1$, se demuestra que tiene que darse $a = b$ o que
      $a = {-b}$.

      Para los casos $a \neq 0$, $b = 0$ y $a = 0$, $b \neq 0$, sería
      vacuamente verdadera. Para el caso $a = b = 0$, se demostraría como
      los otros casos, siempre y cuando se tenga en cuenta que ${-0} =
      0$.\qedhere
  \end{enumerate}
\end{proof}

Existen muchas otras proposiciones de este tipo que son triviales de
demostrar y no nos molestaremos en presentar aquí. Si las necesita para
algún ejercicio, puede deducirlas sobre la marcha. También, si quiere tener
un listado más amplio de este tipo de proposiciones, puede consultar
\cite{burton}.

A continuación vamos a dar dos proposiciones que, en combinación, nos dan la
posibilidad de abarcar un espectro muy amplio de casos para este operador.

Según la primera de estas, si un número divide a otro, entonces también
dividirá a todos los múltiplos de este.

\begin{proposition}[del Factor del Múltiplo]\label{prop-divide-al-multiplo}
  Dados $a, b, k \in \zset$. Si $a \mid b$, entonces se cumple $a \mid bk$.
\end{proposition}

Se demuestra simplemente dándose uno cuenta de que estamos en una situación
en la que se está haciendo uso de la propiedad transitiva, ya que, por
definición, $bk$ será múltiplo de $b$.

  \iffalse
  La demostración es igual de trivial que las anteriores. De hecho, se
  podría deducir directamente de la propiedad transitiva que acabamos de
  ver, ya que, por definición, $bk$ será múltiplo de $b$. En cualquier caso,
  vamos a poner una demostración completa.

  Puesto que se cumple que $a \mid b$, existirá un $d \in \zset$ tal que

  $$ b = ad $$

  Ahora, tomamos un $k \in \zset$ y lo multiplicamos en las dos partes de la
  igualdad anterior.

  $$ bk = (ad)k = a(dk) $$

  Por la estructura algebraica que manejamos, el producto es una operación
  interna, con lo que $dk \in \zset$. Por tanto, queda demostrado que $a
  \mid bk$.

  Se debe comprobar también si se cumple en los casos en los que interviene
  el 0. Para el caso en que $a = b = 0$, la expersión $a \mid bk$ se
  reduciría a $0 \mid 0$, que, como vimos, es cierta.

  Si $a \neq 0$, $b \neq 0$ y $k = 0$, la expresión se reduce a $a \mid 0$,
  que también vimos que era cierta. Lo mismo para el caso $a \neq 0$, $k
  \neq 0$ y $b = 0$.

  Si $a = 0$ y $b$ y $k$ son distintos de 0, será vacuamente cierta.
  \fi

Otra proposición importante y con mucho uso de este operador viene a decir
que, si tenemos dos múltiplos de un número, la suma y la resta de estos
también serán múltiplos de dicho número. Es decir:

\begin{proposition}
  Dados $d, x, y \in \zset$. Si $d \mid x$ y $d \mid y$, entonces $d \mid (x
  + y)$ y $d \mid (x - y)$.
\end{proposition}

\begin{proof}
  Se debe a que, en los números enteros, se cumple la propiedad distributiva
  del producto respecto a la suma.

  La hipótesis implica que $x = md$ e $y = nd$ para algún par de enteros $m,
  n \in \zset$. Sumando y restando $x$ e $y$, tenemos

  $$ x \pm y = md \pm nd = (m \pm n)d $$

  \noindent con lo que $d \mid (x \pm y)$.

  Habría que estudiar también los casos en los que alguna de las variables
  vale 0. En todos, sigue siendo cierta la proposición. Sería destacable el
  caso en que $d = 0$ pero alguno de $x$ e $y$ es distinto de 0. En dicho
  caso, la proposición sería vacuamente cierta.
\end{proof}

La proposición anterior se puede exponer de otra forma, que será la que
usemos con más frecuencia, aunque sigue siendo lo mismo.

\begin{proposition}[del Dos de Tres Para la
  Divisibilidad]\label{princ-dos-de-tres}
  Dados $a, b, c \in \zset$ que satisfacen la ecuación $a + b = c$ y dado $d
  \in \zset$. Si dos de los números del conjunto $\{a, b, c\}$ son múltiplos
  de $d$, entonces el otro también lo será.
\end{proposition}

Tal y como expliqué antes, uniendo esta proposición con la
\ref{prop-divide-al-multiplo} podemos sacar muchas conclusiones sobre la
operación ``divide a'', en muchos casos. Concretamente, uniendo ambas
tendríamos que si un número divide, de forma independiente, a los
coeficientes de una combinación lineal entera (una combinación lineal como
las que conoce solo que con los coeficientes y las variables en el dominio
de $\zset$), entonces también dividirá a esta. Pero en realidad no se queda
ahí, pues también nos permiten explicar el divisor para expresiones más
complicadas.

Ejercicio. ¿Tiene soluciones enteras la ecuación $7x^2 + 11 = 21y$?

Partamos de un hecho que vimos antes que se cumplía siempre. Si un número
$a$ divide a otro $b$, entonces también divide a cualquier múltiplo de $b$,
por ejemplo, expresado como $k \cdot b$.

Volviendo ahora a los datos de este ejercicio, tenemos que, aplicando lo
anterior, sabemos que tanto $7x^2$ como $21y$ son múltiplos de 7, ya que $21
= 7 \cdot 3$. Aplicando aquí la Proposición \ref{princ-dos-de-tres}, se
tiene que 11 ha de ser múltiplo de 7, pero esto sabemos que no es cierto.
Por tanto, aplicando el condicional contrarrecíproco, se tendrá que la
ecuación no tiene solución.

Este ejercicio viene en \cite{weissman}, problema 0.34, pág. 17.

Aunque más adelante veremos ecuaciones de este tipo, que las suelen llamar
ecuaciones diofánticas, en este caso simplemente hemos usado el ingenio con
nuestros pocos conocimientos hasta ahora de teoría de números para deducir
que no tiene solución.

La proposición siguiente también se usa en muchas demostraciones.

\begin{proposition}[del Factor y el Menor o Igual]\label{prop-factor-men}
  Dados $a, b \in \zset$, siempre que no se dé que $a \neq 0$ y $b = 0$. Si
  $a \mid b$, entonces $|a| \leq |b|$.
\end{proposition}

\begin{proof}
  Para el caso con $a = b = 0$, la demostración es trivial, pues $0 \mid 0$
  y $0 = |0| \geq |0| = 0$. Para el caso con $a = 0$ y $b \neq 0$ es
  vacuamente verdadera pues $0 \nmid b$, es decir, el antecedente es falso.

  Para el caso con $a \neq 0$ y $b \neq 0$, tenemos que existe un $q \in
  \zset$ tal que

  $$ b = qa $$

  Además, por las propiedades del valor absoluto, tenemos que

  $$ |b| = |qa| = |q| \cdot |a| $$

  Como $b \neq 0$, también será $q \neq 0$, pues, de lo contrario, se
  tendría $|b| = |q| \cdot |a| = 0$. Por tanto, $|q| \geq 1$, ya que el
  valor absoluto es siempre positivo. Uniendo ambas condiciones, tenemos

  $$ |b| = |q| \cdot |a| \geq 1 \cdot |a| = |a| $$

  \iffalse
  % Desarrollo de la explicación de $|b| = |q| \cdot |a| \geq |a|$.

  Tomamos una variable $m$ a partir de $q$ tal que $m = q - 1$. Entonces,
  tenemos que

  $$ |b| = |m + 1| \cdot |a| $$

  Por la desigualdad triangular, se tiene que $|m + 1| \leq |m| + |1|$. Por
  tanto,

  $$ |b| = |m + 1| \cdot |a| \leq (|m| + 1) \, |a| = (|m| \cdot |a|) + |a|
  $$

  Como $|q| \geq 1$, tal y como dijimos antes, podemos ahora deducir que
  $|m| \geq 0$, aunque esto sería innecesario. Entonces, $|m| \cdot |a| \geq
  0$. Y esto es la definición de ``mayor o igual que''. Por tanto, se
  cumplirá $|b| \geq |a|$.
  \fi

  Si tiene curiosidad de por qué se excluye el caso para $a \neq 0$ y $b =
  0$, se debe fijar en que se cumple $a \mid 0$, pero también se cumple $|a|
  > 0 = |0| = |b|$.
\end{proof}

Como caso particular, se cumple también que, si $a > 0$ y $b > 0$, entonces,
si $a \mid b$, se cumple que $a \leq b$.












% -----------------------------------------------------------------------


\iffalse
El concepto de combinación lineal, que definimos a continuación, sería como
el de múltiplo pero elevado a un grado mayor de complejidad.

\begin{deffinition}[Combinación lineal entera]
  Dados $a, b \in \zset$. Se llama combinación lineal entera de $a$ y $b$ a
  cualquier expresión como la siguiente

  $$ ax + by $$

  \noindent o resultado de esta, para cualesquiera $x, y \in \zset$.
\end{deffinition}

A las \m{combinaciones lineales enteras} (\engm{integer linear
combinations}), normalmente, en los textos de teoría de números, se las
suele llamar simplemente \m{combinaciones lineales}, pues se sobrentiende
que se trata de enteras.

La proposición siguiente sería una generalización de la propiedad 2 de las
propiedades \ref{propi-divide-2}, puesto que, tal y como acabamos de
explicar, el concepto de combinación lineal es como el de múltiplo pero
elevado a un grado mayor de complejidad. Es una proposición muy importante y
que tiene mucho uso en todo lo que sigue. Se usará en muchas de las
demostraciones posteriores de la teoría de números.

Esta viene a decir que si un número divide por separado a otros dos,
entonces también dividirá a cualquier combinación lineal de ambos.

\begin{proposition}[del Factor de las Combinaciones
  Lineales]\label{prop-comb-lin}
  Dados $a, b, c \in \zset$. Si $a \mid b$ y $a \mid c$, entonces $a \mid bx
  + cy$ para cualesquiera $x, y \in \zset$.
\end{proposition}

\begin{proof}
  Como se dan $a \mid b$ y $a \mid c$, existirán $q_1, q_2 \in \zset$ tales
  que

  \begin{alignat*}{2}
    b   &= a q_1 \\
    c   &= a q_2
  \end{alignat*}

  Entonces,

  $$ bx + cy = a q_1 x + a q_2 y = a(q_1 x + q_2 y) = aq $$

  \noindent siendo $q = q_1 x + q_2 y$. Así, se ve claramente que se cumple
  entonces que $a \mid bx + cy$.

  En particular, para el caso cuando $a = 0$ será vacuamente verdadera. Para
  cuando $a \neq 0$ y $b = c = 0$, como la combinación lineal valdrá 0 y
  todo número divide al 0, también será verdad. Para el caso cuando $a = b =
  c = 0$, se reducirá a $0 \mid 0$, que, como sabemos es verdad. Todos los
  casos en los que interviene el 0 son verdad, pero los demás no los
  demostraré.
\end{proof}
\fi




\iffalse
La siguiente es un caso particular de la proposición anterior.

\begin{proposition}
  Dados $a, b \in \zset$ con $a > 0$ y $b > 0$. Si $a \mid b$, entonces $a
  \leq b$.
\end{proposition}

\begin{proof}
  Ya que $a > 0$ y $b > 0$, se cumple, por la definición del valor absoluto,
  que $|a| = a$ y $|b| = b$. Y, como se cumple, por la proposición
  \ref{prop-factor-men}, $|a| \leq |b|$, se tiene entonces que

  $$ a = |a| \leq |b| = b $$
\end{proof}
\fi
