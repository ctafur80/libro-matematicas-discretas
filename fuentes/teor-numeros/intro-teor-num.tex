




Hay que aclarar que, en lo que respecta a la teoría de números, la que se
toca en esta asignatura es la \m{elemental} (\engm{elementary}), adjetivo
que se da para especificar que puede seguirla casi cualquiera, al casi no
tener requisitos; basta con conocimientos de matemáticas de educación
primaria. Curiosamente, en la educación secundaria no se toca nada de esto.
La no elemental puede hacer uso de nociones de álgebra abstracta, de
análisis de variable compleja, o de otras áreas. Como puede imaginar, son
asignaturas avanzadas. Cada una de estas suele tener su propio adjetivo. Por
ejemplo, teoría de números algebraica, entre otras.

Debido a esto, no abundan los libros sobre la teoría de números elemental.
Uno que se puede considerar un clásico es \cite{burton}. Es un libro muy
antiguo, aunque se ha seguido editando y han ido actualizando algunos de sus
contenidos. Es también bastante críptico, pero puede servir para comprender
bien la asignatura. Otro que le gusta mucho a la gente es \cite{rosen}, y me
parece que explica mejor las cosas, aunque es un libro que se centra mucho
en las aplicaciones de esta materia; principalmente, en la criptografía.
Rosen es autor también de un libro muy popular sobre matemáticas discretas.
Otro que es un clásico, pero que creo que es algo más avanzado, es
\cite{theory-of-numbers-niven}, pero es difícil de encontrar actualmente.

Un libro de reciente aparición que también toca la teoría de números
elemental es \cite{weissman}. Toca la mayoría de los temas de
teoría de números de la asignatura, pero es algo distinto y creo que toca
menos cosas que los anteriores. El punto fuerte de este es que tiene
muchísimos gráficos muy bien hechos y que resultan de gran ayuda para
comprender las explicaciones. También, he notado que la notación que usa es
algo innovadora, es decir, difiere algo de las que se suelen ver en esta
materia, pero creo que es bastante acertada.

La mayoría de los libros de esta materia están constantemente, en todas las
definiciones, proposiciones, propiedades, etc., haciendo excepciones sobre
el rango de validez de las variables, es decir, excluyendo de estas algunos
casos particulares. Concretamente, en casos que involucran al 0. Se debe a
la decisión de excluir al 0 en el operando de la izquierda en la definición
del operador ``divide a'' (sección \ref{op-divide-a}). Es decir, encontrará
que casi siempre dicen algo así como ``$a \neq 0$ tal que $a \mid b$''. En
cualquier caso, esto tampoco evita que en muchas ocasiones aparezcan estas
``verrugas''.

Esta decisión es la que se ha tomado tradicionalmente en la teoría de
números, pero últimamente hay quien opta por no seguirla. Es el caso de
\cite{weissman} pág. 15. También, quizás influenciado por el
anterior, se sigue esta regla en \cite{comb-num-th-mileti} sección 1.5
Divisivility. En estos, las definiciones son más amplias y, por ejemplo, en
lugar de poner ``$a \neq 0$ en $a \mid b$'' dicen simplemente ``$a \mid
b$''. Si lo piensa, no hay ningún problema en que sea así. Simplemente, para
un $b \neq 0$ sucederá siempre que $0 \nmid b$, lo cual no afectaría a la
validez de la proposición.

Estas definiciones, proposiciones, etc. amplias, sin especificar los casos
conflictivos que involucran al 0, hacen uso de lo que suelo llamar un caso
\m{vacuamente verdadero} (\engm{vacuously true}), que no es más que, en una
proposición lógica que tiene un condicional (normalmente, representado por
`$\longrightarrow$' o `$\Longrightarrow$'), si el antecedente es falso, el
consecuente será automáticamente cierto, independientemente del valor de
verdad de la otra proposición. Es lo que sucede, por ejemplo, en la
proposición \ref{prop-divide-al-multiplo}, así como en muchas otras del
texto.

Pero, ¿qué sucede para $0 \mid 0$? Quizás le ``chirríe'' ver que admitimos
que $0 \mid 0$ es cierto, pues a usted le enseñaron en sus cursos anteriores
de matemáticas que el 0 no divide a ningún número, o que, como dicen en las
asignaturas de cálculo, 0 dividido entre 0 es una indeterminación. Sin
embargo, si nos ceñimos a la definición del operador anterior, sí sería
cierto. TKTK.

Otra razón que me hace considerar que $0 \mid 0$ es que, como se verá en la
sección donde se presenta el máximo común divisor, se ha de admitir que el
máximo común divisor de 0 y 0 es 0. Si se excluye la posibilidad de $0 \mid
0$, esto lo veo como un ``parche'' o ``verruga'' a la teoría. Sin embargo,
en caso de admitirlo, es algo que se deduce fácilmente.

En esta asignatura, al principio, parece imposible hacer muchos de los
ejercicios. En realidad, se suelen basar en tener un bagaje de conocimientos
de hechos sobre los números enteros. Por ejemplo, cosas como que un número
par por uno impar da uno par, propiedades sobre el operador ``divide a'', y
cosas así. Es conveniente, antes de ponerse a hacer ejercicios difíciles,
hacer por uno mismo una gran cantidad de esas demostraciones de propiedades
sencillas. Muchas de estas aparecen en los libros sobre demostraciones,
como, por ejemplo en \cite{proofs-cummings}.



