



Tal y como se verá más adelante, los números primos son como los ``átomos''
de los que están hechos los números.

\begin{deffinition}[Número primo]
  Dado un $p \in \nset$ con $p > 1$. Decimos que $p$ es un \emph{número primo}
  (o, simplemente, un \emph{primo}) si sus únicos divisores positivos son 1 y
  $p$. Si no es primo, se dice que es \emph{compuesto}.
\end{deffinition}

Si se fija, nos hemos limitado a números mayores que 1. Esto es lo que se hace
con los números primos, pues TKTK. Así, pues, como particularidad se tiene que
el 1 no es un número primo.

En muchas definiciones, teoremas, etc., si se impone la condición de que una
variable representa a un número primo, no se dirá que $p > 1$, pues sería
redundante. Pero debe recordar que los números primos son siempre mayores que 1.

Advierta también que, como $p > 1$, los divisores no pueden ser mayores que
$p$, según la Proposición \ref{prop-factor-men}.

Se podría dar la misma definición pero más simbólica.

$$ p \in \nset \ \text{con} \ p > 1 \ \text{es primo} \quad \iff \quad
\forall m \in \nset \ \text{con} \ 1 < m < p. \quad \nexists q \in \zset.
\quad p = m \,q $$

Podemos hacer otra definición equivalente que en realidad no dista tanto de
esta. Simplemente, pasamos a llamar $a$ a $m$ y $b$ a $q$. Es la siguiente.

\begin{proposition}[Definición Alternativa de Número Primo]
  Un número $p \in \nset$ con $p > 1$ es primo si y solo si no existen $a, b
  \in \nset$ con $1 < a < p$ y $1 < b < p$ para los que se cumple $p = a \,
  b$.
\end{proposition}

Verá que también se usa la negación de esta para definir a los números
compuestos, pues es útil hacerlo así en algunas ocasiones.

\begin{proposition}[Definición Alternativa de Número Compuesto]
  Un número $p \in \nset$ con $p > 1$ es compuesto si y solo si existen $a,
  b \in \nset$ con $1 < a < p$ y $1 < b < p$ para los que se cumple $p = a
  \, b$.
\end{proposition}

Es facil comprobar que los primeros números primos son $2, 3, 5, 7\ldots$.
(Recuerde que el 1 no es número primo.) Como particularidad, es evidente que
el 2 es el único número primo par, ya que, cualquier número natural mayor
que 2 que sea par será compuesto al ser múltiplo de 2.

\begin{deffinition}[Primos relativos]
  Dados dos números $a, b \in \zset$. Si $\gcd(a, b) = 1$ diremos que $a$ y
  $b$ son \emph{primos relativos}.
\end{deffinition}

Puede encontrar otras formas de llamar a los primos relativos. Por ejemplo,
que $a$ y $b$ son primos entre sí o que son coprimos.

La definición de primos relativos se puede generalizar a un número mayor y
arbitrario de argumentos. Advierta que, dados $a, b, c \in \zset$, $\gcd(a,
b, c) = 1$ no implica necesariamente que $\gcd(a, b)$, que $\gcd(a, c)$ o
que $\gcd(b, c)$.

\begin{lemma}[de Euclides]\label{th-lema-euclides}
  Sean $a, b, k \in \zset$. Si $\gcd(a, b) = 1$ y $a \mid bk$, entonces $a
  \mid k$.
\end{lemma}

\begin{proof}
  Por el Corolario \ref{cor-comb-lin-primos-rel}, como $\gcd(a, b) = 1$, se
  tendrá que existen $x, y \in \zset$ para los que $1 = ax + by$.
  Multiplicando por un $k \in \zset$, se tiene

  $$ k = 1 \cdot k = (ax + by)k = akx + bky = (ak)x + (bk)y $$

  Además, es evidente que $a \mid ak$, al ser $ak$ un múltiplo de $a$. Con
  esto, y la hipótesis $a \mid bk$, se sigue que $a$ divide a esa
  combinación lineal de $x$ e $y$, es decir, $a \mid [(ak)x + (bk)y]$. Pero
  esta es lo mismo, según la igualdad anterior, que $k$, con lo que se tiene
  que $a \mid k$.
\end{proof}

Por lo que veo, tanto en \cite{burton} como en \cite{weissman} ponen una versión
menos general del siguiente corolario que la que aparece en \cite{texto-uned}.
Concretamente, es un condicional, en lugar de un bicondicional. Además, la
última parte de la demostración de este último no la entiendo, aunque tampoco le
he dedicado mucho tiempo. En cualquier caso, me quedo con la versión del
condicional, que es la que presento a continuación.

\begin{corollary}[Lema de Euclides Para un Número
  Primo]\label{cor-lema-euclides-num-primos}
  Dados $b, c, p \in \zset$ siendo $p$ un número primo. Si $p \mid bc$, entonces
  $p \mid b$ o $p \mid c$ (o ambas).
\end{corollary}

\begin{proof}
  Consideremos a $\gcd(p, b)$. Este es un divisor positivo de $p$ y, puesto que
  $p$ es primo, se tiene que o bien $\gcd(p, b) = 1$ o bien $\gcd(p, b) = p$.

  Si $\gcd(p, b) = p$, entonces, como $\gcd(p, b) \mid b$, cosa que es
  evidente, se cumple que $p \mid b$ y se cumpliría el corolario. Si
  $\gcd(p, b) = 1$, ya que $p \mid bc$, aplicando el Lema
  \ref{th-lema-euclides}, se tiene que $p \mid c$.

  Otra forma muy sencilla de abordar la demostración sería dividiéndolo en dos
  casos: si $p \mid b$ y si $p \nmid b$. Esta demostración haría también uso del
  Lema \ref{th-lema-euclides}.
\end{proof}

Ahora, se puede hacer una generalización del corolario anterior. Sería lo
siguiente.

\begin{corollary}\label{lema-euclides-cor-2}
  Dados un número primo $p$ y una lista de números $a_1, a_2, a_3, \ldots,
  a_r \in \zset$. Si $p \mid a_1 \, a_2 \, a_3 \, \cdots \, a_r$, entonces
  se tiene $p \mid a_i$ para algún $a_i$ con $i \in \nset$ y $1 \leq i \leq
  r$.
\end{corollary}

\begin{proof}
  Por la propiedad asociativa, $a_1 \, a_2 \, a_3 \, \cdots \, a_r = (a_1 \,
  a_2 \, a_3 \, \cdots \, a_{r-1}) a_r $. Por el Corolario
  \ref{cor-lema-euclides-num-primos}, se tendrá que serán ciertas $p \mid
  a_1 \, a_2 \, a_3 \, \cdots \, a_{r-1}$ o $p \mid a_r$, pudiendo serlo
  ambas. En caso de que sea cierto $p \mid a_r$, ya estaría demostrado.

  En caso contrario, tenemos que $p \mid a_1 \, a_2 \, a_3 \, \cdots \,
  a_{r-1}$ y operamos sobre este del mismo modo. Llegamos a que $p \mid a_1
  \, a_2 \, a_3 \, \cdots \, a_{r-2}$ o $p \mid a_{r-1}$.

  Continuando sucesivamente con este mismo razonamiento, si no se ha
  encontrado ninguno que detenga el procedimiento, se llegaría a la
  situación en que nos quedan únicamente dos números. Tenemos que $p \mid
  a_1 \, a_2$, con lo que $p \mid a_2$ o $p \mid a_1$. En cualquier caso, se
  cumplirá $p \mid a_i$ para algún $a_i$ siendo $i \in \nset$ y $1 \leq i
  \leq r$.
\end{proof}

\begin{corollary}\label{lema-euclides-cor-3}
  Sean $p, q_1, q_2, q_3, \ldots, q_r \in \zset$ todos y cada uno números
  primos (por separado; no me refiero a relativos). Si $p \mid q_1 \, q_2 \,
  q_3 \, \cdots \, q_r$, entonces se tiene $p = q_i$ para algún $i \in
  \nset$ con $1 \leq i \leq r$.
\end{corollary}

\begin{proof}
  Aplicando el Corolario \ref{lema-euclides-cor-2}, tenemos que para algún
  $i$ se da $p \mid q_i$. Por otro lado, al ser $q_i$ primo, sus únicos
  divisores posibles serán 1 y $q_i$. Por tanto, $p$ solo puede valer 1 o
  $q_i$. Pero, $p$ no puede valer 1 ya que el 1 no es un número primo. Por
  tanto, $p$ solo puede valer $q_i$.
\end{proof}

Ahora, vamos a ver al que se conoce como Teorema Fundamental de la
Aritmética. Con tal nombre, parece evicente que se trata de un teorema muy
importante. En el fondo no es más que algo que ya conoce: la descomposición
de un número natural mayor que 1 en factores primos.

No me gusta nada cómo se enuncia en \cite{texto-uned}, por lo que
voy a dar la definición de \cite{burton}.

\begin{theorem}[Fundamental de la
  Aritmética]\label{th-fundamental-aritmetica}
  Todo número natural mayor que 1 es primo o un producto de primos. Además,
  esta representación es única, siempre que se ignore el orden de
  presentación de los factores.
\end{theorem}

\iffalse
\begin{theorem}[Fundamental de la
  Aritmética]\label{th-fundamental-aritmetica}
  Dado $n \in \zset$ con $n > 1$. Existe un conjunto de números primos $p_1,
  p_2, p_3, \ldots, p_r$ tales que

  $$ n = p_1 \cdot p_2 \cdot p_3 \cdot \cdots \cdot p_r $$

  \noindent donde $p_1 \leq p_2 \leq p_3 \leq \cdots \leq p_r$.

  Además, la factorización es única en el sentido siguiente. Sean $q_1, q_2,
  q_3, \ldots, q_s$ números primos con $q_1 \leq q_2 \leq q_3 \leq \cdots
  \leq q_s$ y tales que $n = q_1 \cdot q_2 \cdot q_3 \cdot \cdots \cdot
  q_s$, entonces $r = s$ y para todo $i \in \zset$ con $1 \leq i \leq r$ se
  tiene $p_i = q_i$.
\end{theorem}
\fi

\begin{proof}
  Como se trata de un teorema de existencia y unicidad, estas serán las dos
  fases que se seguirán en la demostración.

  Fase 1. Todo número se puede representar como un producto de primos.

  Para un número $n \in \zset$ siendo $n > 1$ solo se tienen dos
  posibilidades: que sea primo o que sea compuesto. En el primer caso, ya
  tendríamos la representación en factores primos y aquí terminaría la
  demostración. En caso de que sea compuesto, entonces existe un $d \in
  \zset$ con $1 < d < n$ tal que $d \mid n$.

  Ahora, tomamos al menor de estos enteros $d$, es decir, al menor de los
  divisores de $n$ en el intervalo $1 < d < n$, cosa que es perfectamente
  factible debido a la Propiedad \ref{princ-buena-ord}. A este valor lo
  representaremos por $p_1$.

  Si $p_1$ fuese compuesto, entonces tendría algún divisor $q$ con $1 < q <
  p_1$, y entonces se cumpliría $q \mid p_1$ y $p_1 \mid n$, y aplicando la
  propiedad transitiva del operador ``divide a'', se tendría $q \mid n$, con
  lo que entonces $p_1$ ya no sería el menor de los divisores de $n$ en el
  rango $1 < p_1 < n$. Esta contradicción nos conduce a que $p_1$ no sea
  compuesto, con lo que solo queda que sea primo.

  Por tanto, ya que estamos en el caso de que $n$ sea compuesto, podemos
  descomponerlo en una multiplicación de un primo $p_1$ y un número $n_1$
  que necesariamente cumpla $1 < n_1 < n$.

  $$ n = p_1 \, n_1 \quad \text{con} \ 1 < n_1 < n $$

  Si $n_1$ es primo, entonces ya tenemos nuestra representación y aquí
  terminaría la demostración. Si es compuesto, repetiríamos el argumento,
  con otros datos actualizados, para producir un segundo primo $p_2$ tal que
  $n_1 = p_2 n_2$ siendo $1 < n_2 < n_1$, y, por tanto,

  $$ n = p_1 \, p_2 \, n_2 \quad \text{con} \ 1 < n_2 < n_1 < n $$

  Si $n_2$ es primo, entonces ya tenemos nuestra representación y aquí
  terminaría la demostración. Si es compuesto, repetiríamos el argumento,
  con otros datos actualizados, para producir un tercer primo $p_3$ tal que

  $$ n = p_1 \, p_2 \, p_3 \, n_3 \quad \text{con} \ 1 < n_3 < n_2 < n_1 < n
  $$

  Se seguiría así sucesivamente. La sucesión decreciente

  $$ 1 < \cdots < n_3 < n_2 < n_1 < n $$

  \noindent tendrá que terminar en un número finito de pasos, al ser
  desigualdades estrictas. Por tanto, se llegará a un $n_{k-1}$ que sea
  primo, al que pasaremos a denotar por $p_k$. Así, hemos llegado a tener la
  descomposición en factores primos siguiente

  $$ n = p_1 \, p_2 \, p_3 \, \cdots \, p_{k-1} \, n_{k-1} = p_1 \, p_2 \,
  p_3 \, \cdots \, p_{k-1} \, p_k = \prod_{i=1}^k p_i $$

  \noindent siendo $k \in \zset$ con $1 < k < n$ un índice cuyo valor se
  determinará en cada caso, y siendo primos todos los factores $p_i$.

  Fase 2. La representación de la fase anterior será única, si se obvia el
  orden de los factores.

  La forma en que se demostrará será por contradicción. Supondremos que
  existen dos representaciones distintas de factores primos para el número
  $n$ de antes. Evidentemente, se sigue suponiendo que $n$ es compuesto. Se
  tiene, por tanto,

  $$ n = p_1 \, p_2 \, p_3 \, \cdots \, p_r = q_1 \, q_2 \, q_3 \, \cdots \,
  q_s $$

  \noindent suponiendo, sin pérdida de generalidad, que $r \leq s$, y siendo
  primos (individualmente) los $p_i$ y los $q_j$, para todos los $i, j \in
  \zset$ con $1 \leq i \leq r$ y $1 \leq j \leq s$.

  Supongamos, también sin pérdida de generalidad, que esas dos
  representaciones están escritas en orden creciente, es decir,

  $$ p_1 \leq p_2 \leq p_3 \leq \cdots \leq p_r \quad \text{y} \quad q_1
  \leq q_2 \leq q_3 \leq \cdots \leq q_s $$

  Ya que todos los $p_i$ y $q_j$ son primos y $p_1 \mid n$, se tendrá, por
  el Corolario \ref{lema-euclides-cor-3}, que para algún $k \in \zset$ con
  $1 \leq k \leq s$, $p_1 = q_k$, pero entonces se tiene que $p_1 \geq q_1$,
  ya que los $q_i$ estaban ordenados. Análogamente, se tiene que $q_1 \geq
  p_1$, con lo que de ambas desigualdades deducimos que $p_1 = q_1$. Por
  tanto, podemos cancelar estos dos factores en la igualdad anterior, con lo
  que quedaría

  $$ p_2 \, p_3 \, \cdots \, p_r = q_2 \, q_3 \, \cdots \, q_s $$

  Se repetiría ahora el proceso para $p_2$ y llegaríamos a la conclusión de
  que $p_2 = q_2$, con lo que quedaría

  $$ p_3 \, p_4 \, \cdots \, p_r = q_3 \, q_4 \, \cdots \, q_s $$

  Continuando de esta forma, si se diera que $r = s$, se cancelarían todos y
  terminaríamos con la expresión $1 = 1$. Así habríamos demostrado que las
  dos representaciones en factores primos eran en ralidad la misma.

  Si no se diera $r = s$, se supondría, sin pérdida de generalidad, como
  dijimos antes, que $r < s$, y se llegaría, tras cancelar $r$ valores, a
  algo como lo siguiente:

  $$ 1 = q_{r+1} \, q_{r+2} \, \cdots \, q_s $$

  \noindent lo cual no tiene sentido ya que para todo $j$ se tiene $q_j >
  1$, cosa que rompería la igualdad $n = n$. Por tanto, se da
  obligatoriamente que $r = s$.

  En lo que respecta al orden de presentación de los factores, como es
  evidente, ya que en la estructura algebraica en la que nos encontramos se
  cumplen las propiedades asociativa y conmutativa, este es irrelevante.

  Alternativamente, existe una demostración empleando el Principio de
  Inducción, pero aún no hemos llegado a ese tema. Como verá, esa es muy
  breve y sencilla comparada con esta.
\end{proof}

En realidad, la descomposición en factores primos de un número no se suele
presentar tal y como acabamos de mostrar, sino que los factores repetidos se
agrupan y la expresión aparece entonces con exponentes. Esta representación,
es también única, evidentemente, como consecuencia del teorema anterior.

En \cite{texto-uned} lo expresan generalizándolo para representar a
cualquier número entero distinto de 0, pero en realidad no le veo el
sentido. Es mejor definirlo para los números naturales mayores que 1 y luego
mencionar como caso particular que se puede obtener su equivalente en
negativo (es decir, su inverso aditivo) simplemente multiplicando ambas
partes por ${-1}$.

\begin{corollary}[de la Factorización Canónica en Números Primos]
  Todo número $n$ natural mayor que 1 tiene una factorización única, llamada
  \emph{factorización canónica}, de la forma

  $$ n = p_1^{r_1} \, p_2^{r_2} \, p_3^{r_3} \, \cdots \, p_t^{r_t} $$

  \noindent donde $t \in \nset$ con $t \geq 1$, todos los $p_i$ son números
  primos distintos entre sí siendo $p_1 < p_2 < p_3 < \cdots < p_t$, y $r_i
  \geq 1$ para $1 \leq i \leq t$.
\end{corollary}

\begin{proof}
  Algo que hay que tener en cuenta es que tenemos, para las dos
  representaciones, dos índices distintos. En la del Teorema
  \ref{th-fundamental-aritmetica}, el índice de los factores es $r$, pero
  ahí pueden entrar elementos repetidos. En esta, a los elementos repetidos
  los agrupamos y elevamos el exponente en una unidad. Por tanto, en la
  representación canónica (la de este corolario), se tendrán $t$ elementos,
  que serán $t \leq r$.

  Si $n > 1$, hacemos la factorización como se hace en el Teorema
  \ref{th-fundamental-aritmetica}.

  En la representación que obtenemos, para todos los $i$ con $1 \leq i \leq
  r$, agrupamos todos los $p_i$ repetidos. Supongamos que hay $k_i$
  repeticiones de $p_i$. Así, al comprimir la representación, nos quedará

  $$ n = p_1^{k_1} \, p_2^{k_2} \, p_3^{k_3} \, \cdots \, p_t^{k_t} $$

  \noindent que será la representación canónica.

  Ahora, analicemos el caso cuando $n$ sea negativo. En este caso, se tiene
  que $n = {-|n|}$. Entonces, podemos aplicar a $|n|$ lo que hicimos en el
  caso anterior.
\end{proof}

En \cite{weissman} directamente usan la representación
factorial; se saltan el Teorema \ref{th-fundamental-aritmetica}. También, la
primera parte, en la que se demuestra la existencia de la factorización,
está algo escondida, al comienzo del capítulo 2.

Esta factorización nos permite calcular el máximo común divisor de dos
números. Suponga que tiene dos números $a, b \in \zset$, y, sin pérdida de
generalidad, supongamos que $a > b$. Sus representaciones canónicas
supongamos que son:

\begin{alignat*}{2}
  a &= p_1^{k_1} \, p_2^{k_2} \, p_3^{k_3} \, \cdots \, p_n^{k_n} \\
  b &= q_1^{j_1} \, q_2^{j_2} \, q_3^{j_3} \, \cdots \, q_n^{j_n}
\end{alignat*}

\noindent Hemos puesto ambas con $n$ factores, que serán los que tenga el
mayor. En algunos de los factores, puede que el exponente del factor sea 0.
Entonces, se tiene que

$$ \gcd(a, b) = p_1^{r_1} \, p_2^{r_2} \, p_3^{r_3} \, \cdots \, p_n^{r_n}
$$

\noindent donde $r_i = \text{min}(k_i, j_i)$.

Aunque existen muchos misterios sobre los números primos, hay algunos
métodos para comprobar si un número es primo más eficientes que ir probando
con todos los números menores que este si son son divisores suyos. Uno muy
popular es la criba (\emph{sieve}) de Eratóstenes. Con este método, para
decir si un número es primo, nos basta con comprobarlo con los primos
menores que su raíz cuadrada.

El teorema siguiente sirve para hacer esta criba. TKTK. Este viene a decir
que un número compuesto $a$ siempre poseerá un divisor primo menor que
$\sqrt{a}$ que será divisor de $a$.

\begin{theorem}[de la Criba de Eratóstenes]
  Para todo número compuesto $a \in \nset$ con $a > 1$, existe un número
  primo $p$ tal que $p \leq \sqrt{a}$ para el que se cumple $p \mid a$.
\end{theorem}

También puede verse desde el otro punto de vista, es decir, si no existe un
número primo $p$ entre 2 y $\sqrt{a}$ que tenga al número $a$ como múltiplo,
entonces $a$ será primo.

Viene también en \cite{burton} sección 3.2, pág. 44.

\begin{proof}
  Sea $a \in \nset$ con $a > 1$ un número compuesto. Al ser compuesto, este
  se podrá escribir como $a = b \cdot c$ para $b, c \in \nset$ y $1 < b < a$
  y $1 < c < a$.

  Ahora, suponemos, sin pérdida de generalidad, que $b \leq c$.
  Multiplicando esta expesión por $b$ en ambas partes de la desigualdad,
  tenemos

  \begin{alignat*}{2}
    b           &\leq c \\
    b \cdot b   &\leq c \cdot b \\
    b^2         &\leq a \\
    b           &\leq \sqrt{a}
  \end{alignat*}

  Al ser $b < 1$, por el Teorema \ref{th-fundamental-aritmetica} se tendrá
  que $b$ tiene al menos un factor primo, que designaremos por $p$. Se
  cumple, entonces, $p \leq b \leq \sqrt{a}$.

  Además, como $p \mid b$ y $b \mid a$, aplicando la propiedad transitiva
  del operador ``divide a'', tenemos que $p \mid a$. Esto último es la
  evidencia de que $a$ posee un divisor primo $p$ tal que $p \leq \sqrt{a}$.

  \iffalse
  Suponemos que $a$ es un número natural positivo que además es compuesto.
  Entonces, por la definición de número compuesto, existen $b, c \in \zset$
  con $1 < b < a$ y $1 < c < a$ tales que $a = bc$.

  Suponemos, sin pérdida de generalidad, que $b \leq c$. Entonces,

  $$ b^2 \leq bc = a $$

  \noindent y, por tanto,

  \begin{alignat*}{2}
    \sqrt{b^2}  &\leq \sqrt{a}   \\
    |b|         &\leq \sqrt{a}   \\
    b           &\leq \sqrt{a}
  \end{alignat*}

  \noindent Aquí, $|b| = b$ porque $b > 1$, como hemos dicho.

  Si $b$ es primo, hemos llegado a una contradicción, ya que por hipótesis
  $a$ no es múltiplo de ningún primo $p \leq \sqrt{a}$.

  Si, por el contrario, $b$ es compuesto, entonces $b$ tiene factores
  primos. Sea $p$ uno de ellos. Entonces, $p < b$ y, así,

  $$ p < b \leq \sqrt{a} $$

  \noindent lo cual es otra contradicción.
  \fi
\end{proof}

Se puede ver inmediatamente dos tipos de ejercicios muy sencillos que surgen
como consecuencia directa de la criba de Eratóstenes.

Por un lado, podemos demostrar que un número $a$ es primo comprobando que no
es múltiplo de ninguno de los números primos entre 2 y $\sqrt{a}$. Vea
\cite{burton} pág. 44. Otro tipo de ejercicio es obtener los
números primos entre 2 y $a$. Para esto, escribiríamos todos los números
entre 2 y $\sqrt{a}$ y después iríamos tachando todos los múltiplos, entre 2
y $a$, de los primos entre 2 y $\sqrt{a}$. Vea \cite{burton}
pág. 45.

Haciendo uso del teorema anterior, se puede encontrar todos los números
primos no mayores de uno que deseemos, $n \in \nset$ con $n > 1$. Es un
método que consiste en hacer una ``criba'' (\emph{sieve}). Al método se le
conoce como La Criba de Eratóstenes (\eng{The Sieve of Eratosthenes}). Lo
primero que se hace es una lista ordenada con todos los números entre 2 y
$n$, ambos incluidos. Luego, comenzaremos por el primero, el 2, y tacharemos
a todos sus múltiplos en la lista, pero este se dejaría sin tachar. Luego,
se haría con el siguiente mayor que el 2 que no esté tachado, el 3, y se
haría lo mismo. El proceso se repetiría hasta el mayor primo que sea menor o
igual que $\sqrt{n}$. Tras terminar el cribado, los números de la lista que
hayan quedado sin marcar podemos afirmar que son números primos.

Por ejemplo, hallar los números primos no mayores que 100. Lo primero sería
escribir la lista 2, 3, 4, 5, 6, 7, \ldots, 100. Se comenzaría el cribado
por el 2, que es el primero de la lista sin marcar. Tachamos a todos los
múltiplos de 2, pero no al 2. Por ejemplo, 4, 6, 8, 10, 12, \ldots.

Pasamos a otra iteración del proceso. Seleccionamos el primer elemento de la
lista sin marcar, el 3. Tachamos todos los múltiplos del 3 que no hayan sido
ya tachados, dejando al 3 sin tachar. Se tacharían entonces 9, 15, 21,
\ldots.

El siguiente no sería el 4, ya que se tachó en el cribado del 2, sino el 5.
Tachamos de la lista a todos los múltiplos de 5 que no se hayan tachado
antes, y, además exceptuando al 5. Serían 25, 35, \ldots.

Si se fija, aunque en cada iteración lo primero que hacemos es simplemente
comenzar por el menor número sin tachar, estamos seleccionando en el fondo
el menor número primo de la lista en su estado actual, ya que sabemos que
este no tiene divisores entre 1 y él mismo, sin incluir ambos extremos.

El último número que seleccionaríamos de la lista para hacer una criba sería
el 7, pues es el mayor primo menor o igual que $\sqrt{100}$, que es 10.

Una vez finalizada esta última criba, la del 7, se habría terminado el
proceso. Entonces, el estado de la lista mostrará tachados a todos los
números compuestos entre 2 y 100, ambos incluidos, y sin tachar a los primos
en ese mismo rango.

\begin{theorem}
  $\sqrt{2}$ no es un número racional.
\end{theorem}

% TKTK Poner en la teoría qué es una fracción irreducible.

\begin{proof}
  Vamos a abordarlo por contradicción. Suponemos que $\sqrt{2}$ es racional.
  Por tanto, existen $m, n \in \zset$ para los que $m/n = \sqrt{2}$. Podemos
  suponer, además, sin pérdida de generalidad, que es una fracción
  irreducible, es decir, $m$ y $n$, en sus respectivas descomposiciones en
  factores primos, no tienen ningún factor en común, cosa que equivale a
  decir que $m$ y $n$ son primos relativos.

  Ahora, si elevamos al cuadrado en ambas partes de la ecuación, tenemos

  \begin{alignat*}{2}
    \frac{m}{n} &= \sqrt{2} \\
    m           &= n \sqrt{2} \\
    m^2
      &= \left(n \sqrt{2}\right)^2 = n^2 \left(\sqrt{2}\right)^2 = 2n^2 \\
  \end{alignat*}

  Por tanto, $2 \mid m^2$. Como sabemos por TKTK, si $2 \mid m^2$, entonces
  $2 \mid m$. Por tanto, existe un $k \in \zset$ tal que $m = 2k$. Así,
  pues,

  \begin{alignat*}{2}
    m     &= 2k \\
    m^2   &= (2k)^2 \\
    2n^2  &= 4k^2 \\
    n^2   &= 2k^2
  \end{alignat*}

  De esto se deduce que también se da $2 \mid n^2$. Entonces, al igual que
  antes, se tiene que $2 \mid n$. Hemos llegado entonces a la conclusión de
  que el 2 es un divisor común de $m$ y $n$, es decir, $2 \mid \{m, n\}$, lo
  cual se contradice con que ambos números sean primos relativos.

  Por tanto, la premisa de la que partimos será falsa. Esta era que
  $\sqrt{2}$ era un número racional, con lo que hemos demostrado que no lo
  es.
\end{proof}

\begin{theorem}[de la Infinitud de los Números Primos]
  Existen infinitos números primos.
\end{theorem}

A continuación damos la misma demostración que incluyó Euclides en su libro.

\begin{proof}
  Se va a abordar por el método de contradicción.

  Suponemos que existe una cantidad finita $n \in \nset$ con $n \geq 1$ de
  números primos. Ese límite inferior es fácil demostrarlo puesto que
  sabemos que, por ejemplo, el 2 es un número primo. Nos centraremos en los
  números naturales mayores que 1, que son los únicos que pueden ser primos.
  Los números primos son, en orden ascendente, $p_1 = 2$, $p_2 = 3$, $p_3 =
  5$, $p_4 = 7$, \ldots. Al último lo designamos por $p_n$.

  Considere ahora al número que es la multiplicación de todos los primos.

  $$ N = p_1 \, p_2 \, p_3 \, \cdots \, p_n $$

  $N$ tiene que ser un entero positivo, ya que todos los primos son enteros
  mayores que 1. Así que $N+1$ será un entero mayor que 1. Por el Teorema
  \ref{th-fundamental-aritmetica}, se tiene que $N+1$, puesto que ha de ser
  necesariamente compuesto al no estar en esa lista de números primos, ha de
  tener un factor primo $p$. Ese $p$ tiene que estar en la lista de factores
  de $N$, ya que estos son todos los números primos. Por tanto, se da que $p
  \mid N$.

  Por otro lado, antes hemos dicho que $p$ es un factor de $N+1$, o lo que
  es lo mismo, $p \mid (N+1)$. Entonces, por la Proposición
  \ref{princ-dos-de-tres}, tenemos que

  $$ p \mid ((N+1) - N) $$

  \noindent o, lo que es lo mismo, $p \mid 1$. Pero ningún número primo es
  un factor de 1. Por tanto, la hipótesis, es decir, que $p$ se encontraba
  en la lista de los números primos, será falsa. Así, pues, $p$ será un
  ``nuevo'' número primo que no habíamos tenido en cuenta.

  Esto mismo se puede hacer continuamente y así ir obteniendo
  sistemáticamente nuevos números primos. Por tanto, la cantidad de números
  primos es ilimitada.

  \iffalse
  siendo $n \geq 1$, de números primos. A estos los representaremos por el
  conjunto siguiente:

  $$ P = \{p_1, p_2, p_3, \ldots, p_n\} $$

  Sea $m = (p_1 \, p_2 \, p_3 \, \cdots \, p_n) + 1$. Por un lado, es
  evidente que $m > 1$, ya que $m$ es 1 más un producto de números mayores
  que 1. Además, podemos factorizar a $m$, con lo que obtenemos

  $$ m = q_1 \, q_2 \, q_3 \, \cdots \, q_r $$

  \noindent siendo todos los $q_i$ primos, y $r \in \zset$ con $r \geq 1$.

  Por otro lado, para todo $j \in \zset$ con $1 \leq j \leq n$, se tiene $m
  \opmod p_j = 1$, por lo que ninguno de los $p_j$ será factor primo de $m$.
  Es decir, ningún elemento de $P$ será igual a $q_1$. En realidad,
  sucedería para todos los $q_i$. Por tanto, $q_1$ (o cualquier $q_i$) será
  un número primo que no pertenece a $P$, cosa que contradice la hipótesis
  de partida y nos lleva a concluir que existen infinitos números primos.
  \fi
\end{proof}

Que se puedan obtener de forma ilimitada números primos no quiere decir que
sea fácil obtenerlos o que exista un algoritmo con el que obtenerlos con
poco esfuerzo. De hecho, aunque existen algoritmos para obtener números
primos de algunos tipos de forma algo eficiente, no existe uno que sea muy
eficiente para ir obteniendo todos los números primos uno a uno. En eso se
basa gran parte de la seguridad en las comunicaciones actualmente.

Me gusta más, al parecerme más elegante, la demostración que aparece en
\cite{weissman} en la pág. 50.

Ejercicio. Demuestre que dos númeos consecutivos son primos relativos.

Suponemos que tenemos los números $n, n+1 \in \zset$. Supongamos que ambos
tienen un divisor común que designaremos por $d \in \zset$.

Como $d \mid n$ y $d \mid (n+1)$, se tiene que $d \mid [(n+1) - n]$, que es
lo mismo que decir que $d \mid 1$. Pero el único número positivo que divide
a 1 es el 1, con lo que $d = 1$ y, por tanto, $\gcd(n, n+1) = 1$.

Si se fija, es lo mismo que la última demostración que hemos hecho.

Se podría demostrar tabién por contradicción, que es una demostración algo
curiosa, al hacer uso de las propiedades del anillo conmutativo $(\zset, +,
\times)$. Vea \url{https://www.youtube.com/watch?v=qFj7V769sY8}.

\begin{lemma}
  El producto de dos o más enteros de la forma $4n + 1$, para un $n \in
  \zset$, es un número con esa misma forma.
\end{lemma}

\begin{proof}
  Bastaría con considerar el producto de dos enteros, pues la multiplicación
  de varios se puede hacer agrupándolos gracias a la propiedad asociativa,
  con lo que, de cumplirse para dos, siempre tendríamos productos de dos
  enteros de esa forma.

  Tomemos dos números genéricos con esa forma. Los designaremos por

  \begin{alignat*}{2}
    k &= 4n + 1 \\
    k' &= 4m + 1
  \end{alignat*}

  \noindent para $m, n \in \zset$. Si los multiplicamos, tenemos

  $$ kk' = (4n + 1)(4m + 1) = 16nm + 4n + 4m + 1 = 4(4nm + n + m) + 1 $$
\end{proof}

\begin{theorem}
  Existen infinitos números primos de la forma $4n + 3$ siendo $n \in
  \zset$.
\end{theorem}

Viene en \cite{burton} pág. 54.

\begin{proof}
  La demostración la haremos por contradicción.

  Suponemos que existe una cantidad finita de núeros primos de la forma $4n
  + 3$, con $n \in \zset$. La lista será la siguiente:

  $$ q_1, q_2, q_3, \ldots, q_s $$

  \noindent Consideremos el número entero siguiente:

  $$ N = 4 q_1 \, q_2 \, q_3 \, \cdots \, q_s - 1 = 4 q_1 \, q_2 \, q_3 \,
  \cdots \, q_s - 4 + 3 = 4(q_1 \, q_2 \, q_3 \, \cdots \, q_s - 1) + 3 $$

  Por el Teorema \ref{th-fundamental-aritmetica}, el número $N$ tendrá una
  factorización prima. Supongamos que esta es

  $$ N = r_1 \, r_2 \, r_3 \, \ldots \, r_t $$

  Puesto que $N$ es un entero impar, ya que tiene la forma $4n + 3$,
  tenemos, para todo $k \in \nset^{+}$ con $1 \leq k \leq t$, $r_k \neq 2$.
  Por tanto, descartamos las posibilidades de que $r_k$ tenga la forma $4n$
  o $4n + 2$, quedando entonces $4n + 1$ o $4n + 3$ como las únicas
  posibilidades. Pero sabemos, por el Lema anterior TKTK, que el producto de
  varios enteros de la forma $4n + 1$ será otro entero con la misma forma,
  es decir $4n + 1$. Por tanto, por el contrarrecíproco, como $N$ tiene la
  forma $4n + 3$, habrá algún $r_i$ que no tenga la forma $4n + 1$, siendo
  $i \in \nset^{+}$ con $1 \leq i \leq t$. Ese $r_i$ tendrá entonces la
  forma $4n + 3$, que es la única posibilidad que queda.

  Pero si ese $r_i$ se encuentra entre los $q_1, q_2, q_3, \ldots, q_s$, se
  tiene que $r_i \mid 1$. Esta parte no la entiendo TKTK. La única
  conclusión posible es que haya más números primos de la forma $4n + 3$ que
  los de la lista $q_1, q_2, q_3, \ldots, q_s$. Por tanto, habrá infinitos
  números primos de la forma $4n + 3$.
\end{proof}

\begin{proposition}
  Para cada $n \in \nset$. Existen al menos $n$ enteros compuestos
  positivos.
\end{proposition}

\begin{proof}
  Dado un $n \in \nset$, consideremos la suma siguiente:

  $$ (n+1)! + 2, (n+1)! + 3, (n+1)! + 4, \ldots, (n+1)! + (n+1) $$

  Para cada $2 \leq k \leq n+1$ se tiene que $k \mid (n+1)!$, ya que en ese
  factorial aparecerá $k$ como uno de los factores. Con esto y, como $k \mid
  k$, se tiene que $k \mid (n+1)! + k$.

  Por tanto, en esta lista de $n$ números consecutivos todos son compuestos.
\end{proof}

El teorema siguiente nos da otro método para calcular el máximo común
divisor de dos números.

\begin{theorem}
  Dados $a, b \in \zset$, con descomposiciones canónicas en factores primos,
  hasta el mayor índice de ambas, el $t$-ésimo:

  $$ a = \pm p_1^{r_1} \, p_2^{r_2} \, p_3^{r_3} \, \cdots \, p_t^{r_t}
  \quad \text{y} \quad b = \pm p_1^{s_1} \, p_2^{s_2} \, p_3^{s_3} \, \cdots
  \, p_t^{s_t} $$

  \noindent Entonces se cumple

  $$ \gcd(a, b) = p_1^{\mathrm{min}(r_1, s_1)} \, p_2^{\mathrm{min}(r_2,
  s_2)} \, p_3^{\mathrm{min}(r_3, s_3)} \, \cdots \, p_t^{\mathrm{min}(r_t,
  s_t)} $$
\end{theorem}

Advierta que, en las representaciones canónicas, algunos de los exponentes
pueden ser 0.

\begin{proof}
  Sea

  $$ d = p_1^{\mathrm{min}(r_1, s_1)} \, p_2^{\mathrm{min}(r_2, s_2)} \,
  p_3^{\mathrm{min}(r_3, s_3)} \, \cdots \, p_t^{\mathrm{min}(r_t, s_t)} $$

  Es evidente que $d$ es divisor común de $a$ y $b$, pues, por ejemplo, $a$
  tiene en su representación canónica a todos esos factores, y, seguramente,
  algunos más. Lo mismo con $b$.

  Supongamos ahora que existe un $c \in \zset$ que sea también divisor común
  de $a$ y $b$. Su descomposición canónica en factores primos sería

  $$ c = p_1^{u_1} \, p_2^{u_2} \, p_3^{u_3} \, \cdots \, p_t^{u_t} $$

  \noindent donde $r_i \leq r_i$ y $r_i \leq s_i$, para todo $i = 1, 2, 3,
  \ldots, t$. Por tanto, $\u_i \leq \mathrm{min}(r_i, s_i)$ para todo $i =
  1, 2, 3, \ldots, t$, con lo que tendremos que $c \mid d$. Por tanto, $d =
  \gcd(a, b)$.
\end{proof}


\begin{deffinition}[Mínimo Común Múltiplo]
  Dados $a, b \in \zset$. Llamamos \emph{mínimo común múltiplo} al menor entero
  positivo que sea múltiplo de ambos. Lo designaremos por $\lcm(a, b)$.
\end{deffinition}

\begin{theorem}
  Dados $a, b \in \zset$, con descomposiciones canónicas en factores primos,
  hasta el mayor índice de ambas, el $t$-ésimo:

  $$ a = \pm p_1^{r_1} \, p_2^{r_2} \, p_3^{r_3} \, \cdots \, p_t^{r_t}
  \quad \text{y} \quad b = \pm p_1^{s_1} \, p_2^{s_2} \, p_3^{s_3} \, \cdots
  \, p_t^{s_t} $$

  \noindent Entonces se cumple

  $$ \lcm(a, b) = p_1^{\mathrm{max}(r_1, s_1)} \, p_2^{\mathrm{max}(r_2,
  s_2)} \, p_3^{\mathrm{max}(r_3, s_3)} \, \cdots \, p_t^{\mathrm{max}(r_t,
  s_t)} $$
\end{theorem}

\begin{proof}
  Si denominamos

  $$ q = p_1^{\mathrm{max}(r_1, s_1)} \, p_2^{\mathrm{max}(r_2, s_2)} \,
  p_3^{\mathrm{max}(r_3, s_3)} \, \ldots, p_t^{\mathrm{max}(r_t, s_t)} $$

  Entonces, $a \mid q$ y $b \mid q$.

  Por otro lado, supongamos $c \in \zset$ con $c > 0$ tal que $c$ es
  múltiplo tanto de $a$ como de $b$. Entonces, su factorización canónica
  será

  $$ c = p_1^{u_1} \, p_2^{u_2} \, p_3^{u_3} \, \ldots, p_t^{u_t} $$

  \noindent donde $u_i \geq r_i$ y $u_i \geq s_i$. Así, $u_i \geq
  \mathrm{max}(r_i, s_i)$, con lo que $q \mid c$, y, por tanto, $q \leq c$.
\end{proof}

También, se tiene el mínimo común múltiplo de más de dos números.

\begin{proposition}
  Un número es un cuadrado perfecto si y solo si su descomposición canónica
  en factores primos tiene todos los exponentes pares.
\end{proposition}

\begin{proof}
  Lo demostraremos en dos fases. En la primera, demostraremos que, si un
  número tiene todos los exponentes pares en su descomposición canónica en
  factores primos, entonces dicho número será un cuadrado perfecto.

  Supongamos que tenemos un número $a \in \nset$ con la descomposición
  canónica en factores primos siguiente:

  $$ a = p_1^{r_1} \, p_2^{r_2} \ p_3^{r_3} \, \cdots \, p_n^{r_n}
  = \prod_{i=1}^n p_i^{r_i} $$

  \noindent siendo el índice $i \in \nset$ con $1 \leq i \leq n$, $p_i$
  número primo y $r_i \in \nset$ la multiplicidad del número primo en la
  descomposición canónica. Si todas las multiplicidades son pares, es decir,
  para todo $i$ se da que $r_i$ es par, entonces existirá un número $b \in
  \nset$ con la siguiente descomposición canónica en factores primos

  \begin{alignat*}{2}
    b &= p_1^{r_1/2} \, p_2^{r_2/2} \ p_3^{r_3/2} \, \cdots \, p_n^{r_n/2} =
        \sqrt{p_1^{r_1}} \, \sqrt{p_2^{r_2}} \, \sqrt{p_3^{r_3}} \, \cdots
        \, \sqrt{p_n^{r_n}} \\
      &= \sqrt{p_1^{r_1/2} \, p_2^{r_2/2} \ p_3^{r_3/2} \, \cdots \,
        p_n^{r_n/2}} = \sqrt{a}
  \end{alignat*}

  \noindent con lo quee se tiene que $a = b^2$, es decir, $a$ es un cuadrado
  perfecto.

  Ahora, pasemos a la segunda fase. Vamos a demostrar que todo número que
  sea un cuadrado perfecto tendrá su descomposición canónica en factores
  primos con todos los exponentes pares.

  Tenemos $a, b \in \nset$ tales que $a = b^2$. Sea la descomposición
  canónica en factores primos de $b$ la siguiente:

  $$ b = q_1^{s_1} \, q_2^{s_2} \, q_3^{s_3} \, \cdots \, q_m^{s_m} $$

  \noindent Entonces, se tiene que

  $$ a = b^2 = \left( q_1^{s_1} \, q_2^{s_2} \, q_3^{s_3} \, \cdots \,
  q_m^{s_m} \right)^2 = q_1^{2s_1} \, q_2^{2s_2} \, q_3^{2s_3} \, \cdots \,
  q_m^{2s_m} $$

  \noindent Como esta expresión tiene la forma de una descomposición
  canónica en factores primos de $a$, y sabemos que esta tiene que ser
  necesariamente única, esta será la que tiene por descomposición canónica
  en factores primos, y, si se fija, tiene todos los exponentes pares, como
  queríamos demostrar.
\end{proof}

\begin{proposition}\label{prod-cuadrado-perfecto}
  Dados dos números enteros positivos. Si son primos relativos y su
  multiplicación es un cuadrado perfecto, entonces estos, de forma
  independiente, serán también cuadrados perfectos.
\end{proposition}

\begin{proof}
  Supongamos que los números son $r, s \in \zset$ y a su producto, $r \cdot
  s$, lo designaremos por $t^2$.

  Si se da que $r = 1$ o $s = 1$, será cierto y la demostración será
  trivial, pues 1 es un cuadrado perfecto.

  Veamos ahora el caso en el que $r > 1$ y $s > 1$. Estos tendrán las
  siguientes descomposiciones canónicas en factores primos:

  \begin{alignat*}{2}
    r &= p_1^{a_1} \, p_2^{a_2} \, p_3^{a_3} \, \cdots \, p_u^{a_u} \\
    s &= p_1^{a_{u+1}} \, p_2^{a_{u+2}} \, p_3^{a_{u+3}} \, \cdots \, p_v^{a_v}
  \end{alignat*}

  \noindent y la de $t$ será

  $$ t = q_1^{b_1} \, q_2^{b_2} \, q_3^{b_3} \, \cdots \, q_k^{b_k} $$

  Ya que $r$ y $s$ son primos relativos, necesariamente se dará que los
  factores en estas factorizaciones de $r$ y $s$ son distintos, es decir, no
  puede haber ninguna coincidencia, pues esto entraría en contradicción con
  que sean primos relativos.

  Por otro lado, como $rs = t^2$, tendremos

  \begin{alignat*}{2}
    p_1^{a_1} \, p_2^{a_2} \, p_3^{a_3} \, \cdots \, p_u^{a_u} \,
      p_1^{a_{u+1}} \, p_2^{a_{u+2}} \, p_3^{a_{u+3}} \, \cdots \,
      p_v^{a_{u+v}}
      &= \left( q_1^{b_1} \, q_2^{b_2} \, q_3^{b_3} \, \cdots \, q_k^{b_k}
        \right)^2 \\
      &= q_1^{2b_1} \, q_2^{2b_2} \, q_3^{2b_3} \, \cdots \, q_k^{2b_k}
  \end{alignat*}

  Del Teorema de la Factorización Canónica TKTK, al ser única la
  descomposición canónica en factores primos, las potencias que aparecen en
  la expresión anterior han de ser las mismas, aunque no necesariamente en
  el orden en el que se presentan aquí. Es decir, cada $p_i$ será igual que
  algún $q_j$, siendo $i, j \in \nset$ los índices de las factorizaciones,
  con $1 \leq i \leq u$ y $1 \leq j \leq v - u$. Además, los exponentes
  correspondientes serán tales que $a_i = 2b_j$. Por tanto, todo exponente
  $a_i$ será par y, por tanto, $a_i/2$ será un entero. Es decir, existirán
  $m, n \in \zset$ tales que $r = m^2$ y $s = n^2$ siendo

  \begin{alignat*}{2}
    m &= p_1^{a_1/2} \, p_2^{a_2/2} \, p_3^{a_3/2} \, \cdots \, p_u^{a_u/2} \\
    n &= p_1^{a_{u+1/2}} \, p_2^{a_{u+2/2}} \, p_3^{a_{u+3/2}} \, \cdots \,
      p_v^{a_{u+v/2}}
  \end{alignat*}

  \iffalse
  \noindent se tendrá la descomposición, no necesariamente canónica, de $a
  \cdot b$ siguiente:

  $$ c^2 = a \cdot b = p_1^{l_1} \, p_2^{l_2} \, p_3^{l_3} \, \cdots \,
  p_n^{l_n} \, q_1^{k_1} \, q_2^{k_2} \, q_3^{k_3} \, \cdots \, q_m^{k_m} $$

  Si se diese para algún par de esos factores $p_i, q_j$ que $p_i = q_j$,
  esto podría provocar que algún exponente en la descomposición canónica de
  $a \cdot b$ fuese impar, pero esto es imposible ya que $a$ y $b$ son
  primos relativos, con lo que para todos los índices $i, j \in \nset$ con
  $1 \leq i \leq n$ y $1 \leq j \leq m$ se tiene $p_i \neq q_j$. Por tanto,
  esa será su descomposición canónica.

  Por otro lado, por la proposición anterior TKTK, que $a \cdot b$ sea un
  cuadrado perfecto, una de las condiciones de esta proposición, es lo mismo
  que decir que su descomposición canónica en factores primos tenga todos
  sus exponentes pares. Por tanto, en la anterior expresión de $a \cdot b$
  todos los exponentes que aparecen serán pares. Concretamente, serán pares
  todos los exponentes de la descomposición de $a$ y los de la de $b$. Así,
  pues, aplicando otra vez el teorema anterior TKTK, tendremos que tanto $a$
  como $b$ serán cuadrados perfectos.
  \fi
\end{proof}

Ahora, vamos a ver la generalización de esto último para una potencia
cualquiera $n \in \nset$.

\begin{proposition}
  Si $a \cdot b = c^n$ para un $n \in \nset$ y $\gcd(a, b) = 1$, entonces
  tanto $a$ como $b$ son potencias $n$-ésimnas.
\end{proposition}

Está en \cite{burton} pág. 247.

\begin{proof}
  Supondremos que $a > 1$ y $b > 1$. Las descomposiciones canónicas en
  factores primos de $a$ y $b$ suponemos que son

  \begin{alignat*}{2}
    a = p_1^{r_1} \, p_2^{r_2} \, p_3^{r_3} \, \cdots \, p_w^{r_w} \\
    b = q_1^{s_1} \, q_2^{s_2} \, q_3^{s_3} \, \cdots \, q_m^{s_m}
  \end{alignat*}

  \noindent siendo $w, m \in \nset$. Teniendo en cuenta que $\gcd(a, b) =
  1$, no puede haber ningún $p_i$ entre los $q_j$, siendo los índices $i, j
  \in \nset$ con $1 \leq i \leq w$ y $1 \leq j \leq m$. Como consecuencia de
  esto, tenemos que la descomposición canónica en factores primos de $a
  \cdot b$ es

  $$ a \cdot b = p_1^{r_1} \, p_2^{r_2} \, p_3^{r_3} \, \cdots \, p_w^{r_w}
  \, q_1^{s_1} \, q_2^{s_2} \, q_3^{s_3} \, \cdots \, q_m^{s_m} $$

  Por otro lado, supongamos que la descomposición en factores primos de $c$
  es la siguiente:

  $$ c = u_1^{l_1} \, u_2^{l_2} \, u_3^{l_3} \, \cdots \, u_t^{l_t} $$

  \noindent con $t \in \nset$. Entonces, la condición $a \cdot b = c^n$ hace
  que tengamos

  \begin{alignat*}{2}
    p_1^{r_1} \, p_2^{r_2} \, p_3^{r_3} \, \cdots \, p_w^{r_w} \, q_1^{s_1}
      \, q_2^{s_2} \, q_3^{s_3} \, \cdots \, q_m^{s_m}
      &= \left( u_1^{l_1} \, u_2^{l_2} \, u_3^{l_3} \, \cdots \, u_t^{l_t} \right)^n \\
      &= u_1^{nl_1} \, u_2^{nl_2} \, u_3^{nl_3} \, \cdots \, u_t^{nl_t}
  \end{alignat*}

  \noindent y necesariamente se tiene que dar que los factores $u_1, u_2,
  u_3, \ldots, u_t$ sean los mismos que $p_1, p_2\, p_3, \ldots, p_w, q_1,
  q_2, q_3, \ldots, q_m$, en algún orden, así como que los exponentes $nl1,
  nl_2, nl_3, \ldots, nl_t$ sean los mismos que $r_1, r_2, r_3, \ldots, r_w,
  s_1, s_2, s_3, \ldots, s_m$. Y, evidentemente, $t = w + m$. Por tanto,
  todos y cada uno de los enteros $r_1, r_2, r_3, \ldots, r_w, s_1, s_2,
  s_3, \ldots, s_m$ tiene que ser divisible entre $n$. Si ahora ponemos

  \begin{alignat*}{2}
    a_1 &= p_1^{r_1/n} \, p_2^{r_2/n} \, p_3^{r_3/n} \, \cdots \,
      p_w^{r_w/n} \\
    b_1 &= q_1^{s_1/n} \, q_2^{s_2/n} \, q_3^{s_3/n} \, \cdots \,
      q_m^{s_m/n}
  \end{alignat*}

  \noindent entonces $a_1^n = a$ y $b_1^n = b$, como deseábamos demostrar.
\end{proof}










































% ----------------------------

Números de Mersenne.

Son los números de la forma $2^n - 1$. No quiere decir que todos los números
de esa forma sean primos, sino que existen algunos números de esa forma que
son primos.

Una observación elemental es que, si en un número de Mersenne, se tiene que
$n$ es par siendo $n \neq 2$, entonces este no es primo.

Este tipo de preguntas ha dicho el profesor que es probable que caiga en el
examen. No esta exactamente, pero este tipo de preguntas dice que le gustan
mucho.
