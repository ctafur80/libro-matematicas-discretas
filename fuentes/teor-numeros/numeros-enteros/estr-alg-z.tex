

Esta consta, además de los propios números de $\zset$, de dos operaciones:
la suma, `$+$', y la multiplicación, `$\cdot$' (a veces se representa por
otra simbología, como `$\times$' o por aposición); me refiero a las
versiones estándar que seguro que conoce\footnotemark. A esta estructura se
la denota por $(\zset, +, \cdot)$ (es decir, por una terna ordenada) o,
también, por $(\zset, +, \times)$; como prefiera.

\footnotetext{aunque en otras asignaturas se usan otras operaciones que aun
siendo distintas a estas se las denota por esos mismos símbolos}

Por un lado, tenemos que $(\zset, +)$ es un grupo conmutativo, puesto que la
operación `$+$' será interna para dicho conjunto y, además se cumplirán las
propiedades asociativa, elemento identidad (que se suele representar como
`0'), elemento inverso y conmutativa. Cuando tenemos en cuenta los elementos
inversos por la suma, tenemos ya los números negativos, con los que aparece
la operación resta.

Por el otro, tenemos que $(\zset, \cdot)$, aunque la operación es interna
para dicho conjunto, y se satisfacen las propiedades asociativa, elemento
identidad (que se suele representar por `1') y conmutativa, no llega a ser
un grupo, pues no cumple la propiedad del elemento inverso. Es decir, si nos
dan un $a \in \zset$, es posible que en $\zset$ no encontremos un elemento
$b$ tal que $a \cdot b = e$, siendo este $e$ el elemento identidad del
producto, o sea, 1. De hecho, los únicos elementos de $\zset$ que cuentan
con elemento inverso son el ${-1}$ y el 1. Evidentemente, se puede
``ampliar'' esta estructura y obtener otra en la que todos los elementos
tengan su inverso. El ejemplo clásico sería el de los números racionales,
pero en esta asignatura apenas se hablará de ellos.

En $(\zset, +, \cdot)$, también se satisface la propiedad distributiva del
producto respecto a la suma. Es decir, dados $a, b, c \in \zset$, se cumplen

\begin{alignat*}{2}
  a(b + c) &= ab + ac \\
  (b + c)a &= ba + ca \\
\end{alignat*}

\noindent Ojo, la distribución de la suma respecto al producto no se cumple.
Es decir, no podemos poner, en general, algo como $a + (b \cdot c) =
(a+b)(a+c)$.

Todo esto nos da, tal y como dijimos, para $(\zset, +, \cdot)$, la
estructura algebraica de anillo conmutativo unitario. Más concretamente, el
elemento unidad del producto es el 1, tal y como dijimos, y no tiene
divisores de 0. Esto último hace que, dados $a, b \in \zset$, si $ab = 0$,
entonces se da necesariamente que $a = 0$, $b = 0$ o ambos. Esta última
propiedad la usaremos en algunos ejercicios.






