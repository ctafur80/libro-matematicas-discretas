



\begin{deffinition}[Valor absoluto]
  Dado un número $n \in \zset$, el \m{valor absoluto} de $n$, que designamos
  como $|n|$, es la aplicación $|\ |: \zset \longrightarrow \nset$
  siguiente:

  \begin{equation*}
    |n| =
    \left\{
    \begin{array}{rl}
      n     & \text{si} \ n \geq 0 \\
      {-n}  & \text{si} \ n < 0
    \end{array}
    \right.
  \end{equation*}
\end{deffinition}

Esta sería la definición estándar, pero existen otras definiciones
alternativas que quizás sean más interesantes e inspiradoras. Por ejemplo,
una de estas sería decir que el valor absoluto mide la distancia entre esos
dos números. ¿La distancia dónde? En la representación que se suele hacer en
de los números en una recta. Esta definición serviría también para intuir
otros conceptos análogos al de valor absoluto solo que para más de una
dimensión.

Algunas propiedades interesantes del valor absoluto son las siguientes.

\begin{properties}
  Dados $a, b \in \zset$.

  \begin{enumerate}
    \item $|a| = 0$ si y solo si $a = 0$.
    \item $|a \cdot b| = |a| \cdot |b|$.
    \item $|a + b| \leq |a| + |b|$. (Desigualdad triangular.)
  \end{enumerate}
\end{properties}

La última de estas tres propiedades es importante y se usa en muchas áreas
de las matemáticas.

\begin{proof}
  La primera de las propiedades es trivial demostrarlo a partir de la
  definición. La segunda, se puede comprobar para los distintos casos,
  dependiendo de los signos que tomen $a$ y $b$; son 4 posibilidades. Para
  la tercera, se puede aportar intuición gráfica simplemente comprobando que
  la suma de dos de los lados de un triángulo siempre será mayor o igual que
  la del otro, aunque eso no es realmente una demostración pues estaríamos
  probándolo para un caso particular. Una forma de demostrarlo sería
  analizando los posibles casos con los distintos signos de $a$ y $b$ y
  también teniendo en cuenta si $a \leq b$ o al contrario.
\end{proof}

\iffalse
Ejercicio. Demuestre que, dado un $a \in \zset$ tal que $a < 0$, se cumple,
tal y como demostramos en el ejercicio de la sección de la resta, que $a <
{-a}$.

Si $a < 0$, entonces, ${-a} > 0$, y, también, $c = 2({-a}) > 0$.

Tenemos

$$ a = a \cdot 1 = a(2 - 1) = 2a - a = {-a} + 2a $$

Restando $2a$ a ambas partes, tenemos

$$ {-a} = a - 2a = a + 2({-a}) = a + c $$

\noindent siendo $c > 0$, tal y como dijimos.

Entonces, de la definición de valor absoluto aplicado aquí se deduce que $a
< {-a}$.
\fi





