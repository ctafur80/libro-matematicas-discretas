

\begin{deffinition}[Mayor que]
  Dados $a, b \in \zset$, diremos que $b$ \m{es mayor que} $a$, y lo
  expresamos como $b > a$, si existe un $d \in \nset$ siendo $d > 0$ que
  hace que se cumpla $b = a + d$.
\end{deffinition}

\noindent O, lo que es lo mismo, ya que ya sabemos qué es la resta, si $b -
a \in \nset^{+}$.

Existe también la operación opuesta, llamada ``menor que'', que se
representa con el símbolo `<'. Es decir, $a < b$ es lo mismo que $b > a$.

\begin{deffinition}[Mayor o igual que]
  Dados $a, b \in \nset$, diremos que $b$ \m{es mayor o igual que} $a$, y lo
  escribiremos como $b \geq a$, si $b > a$ o $b = a$.
\end{deffinition}

Se podría dar otra definición alternativa. Dados $a, b \in \nset$, diremos
que $b$ es mayor o igual que $a$, y lo expresamos como $b \geq a$, si existe
un $d \in \nset$ tal que se cumpla $b = a + d$ o, lo que es lo mismo, $d = b
- a$. O, dicho de otra forma, si $b - a \in \nset$.

Análogamente al operador `$>$', se tiene el operador opuesto a `$\geq$', que
se representará por el símbolo `$\leq$' y se puede definir diciendo que las
expresiones $a \geq b$ y $b \leq a$ representan lo mismo.

A veces, cuando se indica algo como $a > b$, si se desea resaltar que no se
trata de $a \geq b$, se dice que $a$ \m{es estrictamente mayor que} $b$.
Existen otras expresiones similares, como, por ejemplo, que $a$ \m{es mayor
estricto} que $b$.

Como quizás sepa de sus conocimientos de álgebra, estos los operadores
`$\leq$' y `$\geq$' indican, cada uno, para los números enteros
$\mathbb{Z}$, una relación binaria de orden; concretamente, un orden total.
Sin embargo, esto mismo no se puede decir para sus versiones estrictas, ya
que en estas no se cumple, por ejemplo, la propiedad reflexiva. La forma que
tienen es de operadores infijos, aunque podrían aparecer con otras
notaciones; por ejemplo, como función. Se me ocurre la notación `$\geq(a,
b)$'.

Ejercicio. Demuestre que, dado un $a \in \zset$, si $a < 0$ entonces ${-a} >
0$.

\begin{alignat*}{2}
  a         &< 0 \\
  a - a     &< 0 - a = 0 + ({-a}) = {-a} \\
  0         &< {-a}
\end{alignat*}

Las operaciones algebraicas que se pueden hacer con las inecuaciones tienen
algunas sutilezas, pero en la estructura algebraica que nos moveremos no
suele tenerlas TKTK.





