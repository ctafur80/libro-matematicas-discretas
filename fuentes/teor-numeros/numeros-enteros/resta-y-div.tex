

Es interesante también preguntarse por las operaciones de suma y producto
pero con los inversos respectivos para estas operaciones. Estas son,
respectivamente, la resta y la división.

Al tratarse de números enteros, conjunto que incluye, además de los
positivos, al 0 y los negativos, se tiene la operación resta, que es en
realidad una consecuencia de la suma en $\zset$.

\begin{deffinition}[Resta de números en $\zset$]
  Dados $a, b \in \zset$. La operación \m{resta} $a$ menos $b$, que
  simbólicamente denotamos como $a - b$, es el número $d \in \zset$ con el
  que se cumple $a = b + d$.
\end{deffinition}

A la operación resta también se la conoce como \m{diferencia}.

Advierta que la operación resta no es conmutativa, aunque si consideramos a
dicha operación como la suma del inverso aditivo para un número, sí lo sería
TKTK.

Sin embargo, la división se expresará en esta asignatura de un modo algo
distinto a lo que vendría usando en la educación secundaria. Vamos a retomar
la división que usaba en la educación primaria, la división en $\zset$ con
resto, operación que definiremos en el capítulo \ref{ch-div-entera}.

Basta con que de momento sepa que nos autolimitamos a los números enteros;
por tanto, es como si no supiéramos de la existencia de los racionales,
$\qset$, ni de los reales, $\rset$.

\iffalse
, pues, salvo para el número 1, para
nosotros no existirán los inversos multiplicativo en $\zset$.
\fi

Como ve, en esta asignatura nos estamos restringiendo a nosotros mismos
respecto a la estructura que manejamos. Venimos de usar estructuras más
permisivas, como la del cuerpo de los números reales, que es lo que se suele
manejar en las matemáticas de la educación secundaria. Lo hacemos porque los
números enteros, aunque sean más simples, guardan secretos que es
interesante conocer, además de que se pueden aplicar en el mundo real.

\iffalse
Tal y como dijimos, esta estructura es cerrada para la suma, con lo que la
resta TKTK, mientras que no lo es para la división.
\fi

Prioridad de operaciones. TKTK


