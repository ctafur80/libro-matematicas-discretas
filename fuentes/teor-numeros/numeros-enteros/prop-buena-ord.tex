


\begin{theorem}[Propiedad de Buena
  Ordenación\footnotemark]\label{princ-buena-ord}
  Todo subconjunto no vacío de números enteros no negativos tiene un primer
  elemento, es decir, tiene un elemento que es menor o igual que todos los
  demás.
\end{theorem}

\footnotetext{En realidad, es más conocido como Principio de Buena
Ordenación, pero a mí me gusta más calificarlo de propiedad. En inglés, se
le llama \eng{Well-Ordering Principle} o \eng{Well-Ordering Property}.}

Dicho de forma más simbólica:

$$ \forall A \subseteq \zset^{+} \cup \{0\}, A \neq \emptyset. \ \exists m
\in A \st \forall k \in A. \quad m \leq k. $$

En realidad, existen otros enunciados más completos de esta propiedad. TKTK.

Se puede demostrar, pero aquí no lo haremos.

\iffalse
Para todo conjunto $A \subseteq \zset^{+} \cup \{0\}$, con $A \neq
\emptyset$, existe un $m \in A$ tal que para todo $k \in A$, $m \leq k$.
\fi




