


\begin{deffinition}[Grafo]
  Se entiende por \m{grafo} (\engm{graph}) a todo par ordenado $(V, E)$
  siendo los elementos de $E$ conjuntos de dos elementos de $V$.
\end{deffinition}

A los elementos de $V$ se les llama \m{vértices} (\engm{vertices}; singular,
\emph{vertex}) y, a los de $E$, \m{aristas} (\engm{edges}). Muchas veces los
llaman de otras formas. Así, por ejemplo, a los vértices también se les
llama \m{nodos} (\engm{nodes}), a las aristas, \m{enlaces} (\engm{links}) o
\m{ramas} (\engm{branches}), y más que no merece la pena mencionar. Muchas
veces, la terminología depende de la rama de la ciencia o la tecnología a la
que se aplique. Así sucede, por ejemplo, con la teoría de circuitos, en la
electrotecnia. Aun así, la teoría de grafos tiene aplicaciones en ámbitos
muy diversos.

Debe advertir dos propiedades en las que quizás no haya reparado pero que se
encuentran de forma implícita en la definición anterior. En ambas, lo que es
determinante es que el concepto de \m{conjunto} tiene ciertas implicaciones
en oposición a otros como \m{par ordenado} (o, más generalmente, $n$-tupla)
y \m{multiconjunto}. Por un lado, ya que $E$ es un conjunto, en este no
puede haber elementos repetidos. Por tanto, entre dos vértices habrá una
arista o ninguna. Por otro lado, al estar $E$ constituido por conjuntos de
dos elementos, en estos no pueden aparecer elementos del tipo $\{v_i,
v_i\}$, pues, al tratarse de conjuntos, colapsarían a $v_i \in V$ y sería un
conjunto de un elemento, con lo que no puede ser un elemento de $E$.

Si denotamos por $G$ a un grafo, se denotará por $V(G)$ al conjunto de
vértices del mismo, y $E(G)$, al de aristas.

Dado un grafo $G = (V, E)$. Sean $u$ y $v$ vértices de $G$, es decir, $u, v
\in V$. Si el par de vértices $u$ y $v$ forman una arista de $G$, es decir,
$\{u, v\} \in E$, se dice que dichos vértices son \m{adyacentes} y la arista
se designará por $uv$. En ese caso, a los vértices $u$ y $v$ se les llama
\m{extremos} de la arista $uv$. A veces puede ver también que a una arista
$uv \in E$ la designan como $(u, v) \in E$, pero es menos común. Además, en
todo caso tendría más sentido que fuese $\{u, v\} \in E$, que es la notación
que se usa para conjuntos de dos elementos.

Los grafos se suelen representar mediante esquemas, pero en realidad, desde
un punto de vista abstracto y riguroso, TKTK. La representación no es única,
pues el concepto de medida es distinto al de la geometría. Creo que la
teoría de grafos pertenece a la topología. Para hacer la representación,
primero se dibujan los vértices, como puntos, y, luego, se unirán mediante
segmentos de recta los vértices adyacentes. Se debe hacer en ese orden. Esos
segmentos serán las aristas.

Ejemplo. El grafo $G$ representado en la figura \ref{grafo-ejemplo-01} tiene
por conjuntos de vértices y de aristas los siguientes:

\begin{alignat*}{2}
  V &= \{v_1, v_2, v_3, \ldots, v_7\} \\
  E &= \{v_1 v_2, v_2 v_3, v_1 v_3, v_1 v_4, v_4 v_7, v_7 v_6, v_6 v_5, v_5
    v_4, v_5 v_7\}
\end{alignat*}

\begin{figure}\label{grafo-ejemplo-01}\caption{Grafo}
  \centering
  \begin{tikzpicture}[nodes={circle, draw}]

    % Tratar de pasar estas opciones a las del entorno tikzpicture.
    \node [circle, radius=1pt, fill, label=$v_1$] (v1) at (2,4) {};
    \node [circle, radius=1pt, fill, label=$v_2$] (v2) at (0,4) {};
    \node [circle, radius=1pt, fill, label=$v_3$] (v3) at (1,2) {};
    \node [circle, radius=1pt, fill, label=$v_4$] (v4) at (5,3) {};
    \node [circle, radius=1pt, fill, label=$v_5$] (v5) at (4,2) {};
    \node [circle, radius=1pt, fill, label=$v_6$] (v6) at (5,1) {};
    \node [circle, radius=1pt, fill, label=$v_7$] (v7) at (6,2) {};

    \graph [] {
      (v1) -- {(v2), (v3), (v4)};
      (v2) -- (v3);
      (v4) -- {(v5), (v7)};
      (v5) -- {(v6), (v7)};
      (v6) -- (v7);
    };
  \end{tikzpicture}
\end{figure}

Para un grafo $(V, E)$, se denota por $\#V$ al número de vértices y por
$\#E$ al de aristas.

Se dice que un grafo es finito si el número de vértices que tiene es finito,
ya que son los vértices los que marcan inicialmente al grafo; las aristas se
toman sobre los vértices del mismo. En esta asignatura solo se estudiarán
grafos finitos.

Existen algunas variaciones del concepto de grafo, en las que la definición
es algo más laxa. Un \m{multigrafo} (\engm{multigraph}) es un grafo que
puede tener varias aristas entre dos vértices. Esto se logra haciendo que
$E$ sea un multiconjunto, en lugar de un conjunto.

Otra variación laxa del concepto de \m{grafo} es la de \m{pseudografo}. En
estos, se permiten aristas cuyos extremos coinciden. En dicho caso, los
elementos de $E$ serían multiconjuntos de dos elementos.

Advierta la diferencia cuando hemos hablado de multiconjuntos: en los
multigrafos se permite que $E$ sea un multiconjunto, mientras que, en los
pseudografos, se permite que los elementos de $E$ sean multiconjuntos.

Otra variación es el \m{grafo dirigido} (\engm{directed graph}; también
conocido por su contracción \m{digrafo}, \engm{digraph}), que es un grafo
pero los elementos de $E$ son pares ordenados, en lugar de conjuntos de dos
elementos. En el esquema del digrafo, ese orden se indica colocando puntas
de flecha sobre las aristas. En este caso, se llama \m{origen} al primer
vértice de la arista y \m{destino} al segundo.

Como es evidente, pueden aparecer muchas variaciones que incluyan a varias
de estas. Por ejemplo, multidigrafos o pseudomultidigrafos.

\begin{deffinition}[Grafos Isomorfos]
  Sean $(V, E)$ y $(V', E')$ dos grafos y sea $f: V \longrightarrow V'$ una
  biyyección entre los conjuntos de vértices tal que $uv \in E$ si y solo si
  $\{f(u), f(v)\} \in E'$. La biyección $f$ es un \m{isomorfismo} de $(E,
  V)$ a $(E', V')$. Se dice entonces que $(E, V)$ y $(E', V')$ son
  isomorfos.
\end{deffinition}

Es decir, para dos grafos isomorfos se podría decir que, al pasar de uno al
otro, aunque se transformen los vértices no se transforman las aristas de
estos, o, lo que es lo mismo, la adyacencia de vértices. TKTK.

Pueden existir varios isomorfismos diferentes entre dos grafos isomorfos,
pero, si dos grafos no son isomorfos, no encontrará un isomorfismo entre
ellos.

Algo que se puede deducir de la definición anterior es que una condición
necesaria para que dos grafos sean isomorfos es que tengan el mismo número
de vértices, pues, si no se da esta condición, no podrá existir una
biyección entre ambos. TKTK.

\begin{deffinition}[Grado de un Vértice]
  Se llama \m{grado} (\engm{degree}) de un vértice de un grafo al número de
  aristas del grafo que tienen a dicho vértice por extremo.
\end{deffinition}

Quizás, podría pensar que una definición alternativa de \m{grado} de un
vértice de un grafo sería el número de vértices adyacentes que tiene. El
problema es que esto no es cierto en los multigrafos.

El grado de un vértice $u$ se suele denotar por $\text{gr}(u)$. También,
cuando se habla de más de un grafo, se suele poner al designador de este
como subíndice, para así distinguir a cuál nos referimos. Por ejemplo,
$\text{gr}_G(u)$. Aunque no es lo usual, hay quien llama al grado de un
vértice su \m{valencia} (\engm{valence} TKTK).

Los grados de los vértices se conservan por isomorfismo. Demostrar TKTK.

\begin{proposition}
  Sean $G$ y $G'$ dos grafos y $f$ un isomorfismo entre ambos. Si $u$ es un
  vértice de $G$, entonces $\text{gr}_G(u) = \text{gr}_{G'}(f(u))$.
\end{proposition}

\begin{proof}
  Se debe a que $f$ preserva la adyacencia. Por tanto, para cada $u$ y su
  vértice correpondiente en la transformación, es decir, $f(u)$, se conserva
  el grado de los mismos.
\end{proof}

\begin{theorem}[Primer Teorema de la Teoría de Grafos]
  Sea $G = (V, E)$ un grafo con $V = \{v_1, v_2, v_3, \ldots, v_p\}$ el
  conjunto de vértices y $\#E$ su cantidad de aristas. Entonces,

  $$ \sum_{i=1}^p \text{gr}(v_i) = 2 \cdot \#E $$

  \noindent Y, además, todo grafo contiene 0 o un número par de vértices de
  grado impar.
\end{theorem}

A este teorema también se le conoce como el lema de los apretones de mano
(\engm{Handshake lemma} o \engm{Handshaking lemma}).

\begin{proof}
  Al realizar la suma de los grados de cada vértice, o lo que es lo mismo,
  la suma de las aristas de las que este es un extremo, cuando estamos
  contando las de un vértice en concreto, esa misma arista volverá a
  contarse en algún momento, cuando toque contar la del otro extremo de la
  misma. Por lo tanto, teniendo en cuenta el cómputo global, toda arista se
  contará por duplicado, por lo que se tiene

  $$ \sum_{i=1}^p \text{gr}(v_i) = 2 \cdot \#E $$

  Ya hemos demostrado la primera parte. Ahora, vamos a demostrar la segunda.

  Sean, en el grafo de antes, $v_1, v_2, v_3, \ldots, v_t$ los vértices con
  grado par, y sean $v_{t+1}, v_{t+2}, v_{t+3}, \ldots, v_p$ los de impar.
  Tenemos entonces, por lo anterior, que

  $$ 2 \cdot \#E = \sum_{i=1}^p \text{gr}(v_i) = \sum_{i=1}^t \text{gr}(v_i)
  + \sum_{i=t+1}^p \text{gr}(v_i) $$

  Teniendo en cuenta que la suma de los vértices de grado par tiene que ser
  necesariamente par ya que es una suma de números pares, se tiene que la de
  los vértices de grado impar tiene que ser necesariamente par, ya que es
  una resta de dos números pares:

  $$ \sum_{i=t+1}^p \text{gr}(v_i) = 2 \cdot \#E - \sum_{i=1}^t
  \text{gr}(v_i) $$
\end{proof}

Quizás, esto se pueda separar en un teorema y un corolario. TKTK.

\begin{deffinition}[Subgrafo]
  Sea $G = (V, E)$ un grafo. Un \m{subgrafo} (\engm{subgraph}) de $G$ es
  cualquier grafo $H = (V(H), E(H))$ tal que $V(H) \subseteq V$ y $E(H)
  \subseteq E$.
\end{deffinition}

Se podría decir que un subgrafo se genera al tomar un grafo y ``borrar'' o
eliminar algunas aristas y vértices de este. Lo que hay que tener en cuenta
es que, si se elimina un vértice, se deben eliminar todas las aristas de las
que este es un extremo, pues no puede haber aristas algún vértice. Si es
posible la existencia de vértices aislados, es decir, sin ser extremo de
ninguna arista.

Como consecuencia de la definición de \m{subgrafo}, se tiene que, si $H$ es
un subgrafo de $G$, para todo $v_i \in V(H) \subseteq V(G)$ se tiene

$$ \text{gr}_H(v_i) \leq \text{gr}_G(v_i) $$

Creo que de esto podría haber una demostración. TKTK.

\begin{deffinition}[Grafo Regular]
  Un grafo se dice \m{regular} (\engm{regular}) si todos sus vértices tienen
  el mismo grado. Si dicho grado es $k \in \nset$, se dice que dicho grafo
  es $k$-regular.
\end{deffinition}

Como es evidente, la $k$-regularidad de un grafo se conserva por
isomorfismo.

\begin{deffinition}[Grafo Completo]
  Un grafo en el que todo par de vértices son extremos de alguna arista se
  llama \m{grafo completo} (\engm{complete graph} TKTK).
\end{deffinition}

Es decir, un grafo completo es, a partir del conjunto de vértices, el grafo
con mayor número de aristas que podemos formar. O, dicho de otra forma: para
cualquier vértice, todos los demás son sus adyacentes.

\begin{proposition}
  Dos grafos completos con el mismo número de vértices son isomorfos.
\end{proposition}

\begin{proof}
  Al tener ambos el mismo número de vértices, se tendrá una biyección. Es
  fácil entonces establecer una biyección entre ambos. TKTK.
\end{proof}

Al grafo completo con $n$ vértices se le designa por $K_n$. O quizás se
refiere al conjunto de estos TKTK.





% TODO Ver si explica algo el libro sobre los grafos regulares.









