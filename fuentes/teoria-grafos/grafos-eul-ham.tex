



\begin{deffinition}[Camino]
  Un \m{camino} (\engm{path}) en un grafo o un multigrafo es una sucesión
  finita de vértices y aristas del mismo alternándose, siendo, además, los
  extremos de cada arista los vértices adyacentes en dicha sucesión.
\end{deffinition}

Es decir, sería algo así:

$$ v_0, v_0v_1, v_1, v_1v_2, \ldots, v_{n-1}, v_{n-1}v_n, v_n $$

A los vértices primero y último de la sucesión se les llama \m{extremos} del
camino. Es decir, a $v_0$ y $v_n$, respectivamente. También se dice que un
camino \m{conecta} o \m{va de} $v_0$ a $v_n$ (en realidad, también de $v_n$
a $v_0$, ya que no se trata de un grafo dirigido).

La \m{longitud} (\engm{length}) de un camino es el número de aristas que
contiene en la secuencia.

Un camino se dice que es \m{cerrado} (\engm{closed}) si sus extremos
coinciden, es decir, $v_0 = v_n$.

En un grafo, pero no así en un multigrafo, un camino puede ser especificado
de una forma abreviada que consiste únicamente en la sucesión de los
vértices. Es decir, $v_0, v_1, v_2, \ldots, v_{n-1}, v_n$.

Un \m{circuito} (\engm{circuit}) es un camino cerrado que no repite aristas.

Un camino será \m{simple} si en la sucesión de vértices no hay ninguno
repetido.

Un \m{ciclo} es un camino cerrado donde el único vértice repetido es el
primero, que es igual al último. Además, en el caso de caminos de longitud
2, solo se consideran ciclos a aquellos caminos cerrados en multigrafos que
no repiten aristas. Como es evidente, en un camino en el que no se repitan
vértices, tampoco se repetirán las aristas. Por tanto, todo ciclo es también
un circuito.

\begin{theorem}
  Si en un grafo existe un camino que conecta dos vértices distintos,
  entonces existe un camino simple cuyos extremos son esos dos vértices.
\end{theorem}

\begin{proof}
  TKTK.








\end{proof}

Existen grafos donde para cada par de vértices existe un camino cuyos
extremos son estos vértices y otros donde es imposible encontrar un camino
así.

\begin{deffinition}[Grafo conexo]
  Un grafo es \m{conexo} si para cada par de vértices existe un camino para
  el que estos son sus extremos. En caso contrario, se dice que el grafo es
  \m{desconexo}.
\end{deffinition}

\begin{deffinition}[Camino euleriano]
  Un camino euleriano de un grafo es un camino de este que contiene todas
  las aristas del grafo apareciendo cada una de ellas exactamente una vez.
\end{deffinition}

Si ese camino es, además, cerrado, será entonces un circuito, ya que no se
pueden repetir aristas. En ese caso se tendría un \m{circuito euleriano}.

Advierta que lo que sí se permite es que se repitan vértices en el camino,
por lo que no tiene por qué ser un camino simple ni un ciclo.

La definición de \m{camino euleriano} viene del famoso problema de los
puentes de Königsberg, que abordó Euler y dio lugar al nacimiento de la
teoría de grafos. El problema consiste en tratar de encontrar una ruta (un
camino) en dicha ciudad que recorriera los siete puentes (aristas) cruzando
cada uno de ellos una única vez. Hay otras formas alternativas de enunciar
el problema. Por ejemplo, averigüe cómo dibujar, sin levantar el lapiz ni
repasar líneas, una ruta que atraviese todos los puentes.

A un grafo que admite un circuito euleriano se le califica de \m{grafo
euleriano}. En realidad, en otras fuentes dicen que será el que admita un
camino euleriano. TKTK.

Euler encontró una caracterización de los grafos eulerianos en términos de
los grados de los vértices. Resolvió el problema más general que la pregunta
sobre los caminos sobre los puentes de Königsberg.

\begin{lemma}[de los Grados de los Vértices en un Grafo
  Euleriano]\label{lema-g-eul-grado-par}
  En un grafo euleriano todos sus vértices tienen grado par.
\end{lemma}

\begin{proof}
  Que sea un grafo euleriano quiere decir que tiene un circuito euleriano.

  Fijémonos en un vértice cualquiera de dicho circuito que no sea un extremo
  del mismo. Cada arista adyacente a dicho vértice conecta a este con otros
  dos vértices, ya que no es un multigrafo, con lo que aporta 2 unidades al
  grado de dicho vértice.

  Ahora, toca analizar los extremos del circuito. Al tratarse de un
  circuito, ambos extremos serán el mismo vértice, por lo que se comportará,
  a este respecto, como los demás vértices. Por tanto, se aportarán dos
  unidades al grado de dicho vértice.

  En todos los casos, todas las aportaciones son números pares, como hemos
  demostrado, con lo que todo vértice tendrá grado par.
\end{proof}

\begin{lemma}[de los Grados de los Vértices en un Grafo
  Euleriano]\label{lema-camino-eul-grado-par}
  En un grafo que contenga un camino euleriano se da que o bien todo vértice
  tiene grado par o bien dos de los vértices tienen gado impar.

  % TODO Creo que también tiene que ser conexo. Mirarlo en el libro.
\end{lemma}

\begin{proof}
  TKTK.
\end{proof}

El Lema \ref{lema-camino-eul-grado-par} es una condición necesaria para que
un grafo sea euleriano. Puesto que el grafo de los puentes de Königsberg
tiene más de dos vértices con grado impar, se deduce que es un problema que
no tiene solución.

\begin{theorem}
  Un gafo conexo es euleriano si y solo si cada vértice tiene grado par.
\end{theorem}

Antes de demostrar el teorema, vamos a ver un lema.

\begin{lemma}
  Sea $H$ un grafo tal que todo vértice de $H$ tiene grado par. Si $u$ y $v$
  son dos vértices de $H$ que son adyacentes, entonces existe un circuito
  $g$ que contiene a la arista $uv$.
\end{lemma}

\begin{proof}
  





\end{proof}


















