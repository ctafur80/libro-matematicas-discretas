



Una cuestión de las matemáticas sobre la que no existe un consenso
actualmente es si incluir al 0 en el conjunto de los números naturales
(representado normalmente por $\mathbb{N}$). En estos apuntes, se seguirá la
regla opuesta a la del libro oficial, \cite{texto-uned}. La que yo sigo es
también la que prefiere el profesor del centro asociado que me corresponde.

Estuve tentado a hacer que, a lo largo de todo el texto, las variables
pertenecen a $\zset$ de forma implícita, a menos que se especifique otra
cosa, pero en realidad ahora no lo considero una buena idea, pues puede ser
problemático cuando el libro se usa simplemente para consultar algo. De
entre las fuentes que he mencionado, la única que sigue esa regla de estilo
es \cite{leighton}.







En lo que respecta a la combinatoria, me gustaría aclarar que remplazo y
repetición serían sinónimos. Puede ver que se usa reemplazo, forma que
también está aceptada, pero creo que es más conveniente usar remplazo.











