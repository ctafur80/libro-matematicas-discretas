


Esta asignatura consta de tres partes, cada una de las cuales se podría
estudiar en mayor profundidad en una asignatura propia. Son: teoría de
números, teoría de grafos y combinatoria.

Es una asignatura que, aunque los tutores dicen que no es difícil, a mí me
está costando. Al menos, a la vista de los ejercicios que aparecen en el
libro oficial y otros que he consultado, parece que muchos de los ejercicios
requieren de bastante ingenio para resolverlos. Quizás, con el tiempo uno se
vaya acostumbrando y le terminen por parecer fáciles.

El libro que se sigue en la asignatura es \cite{texto-uned}, creado
por profesores de la UNED. Personalmente, es un libro que no me gusta, por
varias razones. Es un libro que da bastante poco conexto y a veces cuesta
entender partes de demostraciones. Además, este libro es, en lo que respecta
a la parte de teoría de números, un plagio de \cite{burton},
un clásico en esta materia. Es fácil comprobar que tiene partes ``calcadas''
de este. Por tanto, creo que, si se puede, es mejor tratar de seguir las
distintas secciones de esta parte por este último. Aunque se dispongan en
otro orden, todo lo que aparece en \cite{texto-uned} se encuentra
explicado tambíen en \cite{burton}. Además, en este último hay
muchas otras cosas que no se explican en el otro, con lo que quizás sea un
libro que merezca la pena comprar, pues es posible que sirva para cursos más
avanzados de esta materia.

Encontré una web en la que tienen colgadas las soluciones a todos los
ejercicios de este. Es la siguiente:
\url{http://www.americanriver.com/mathematics/elementary_number_theory.html}

Alternativamente, se podría optar por usar \cite{rosen} como
sustituto o complemento del libro oficial. Está bastante bien y es algo más
moderno que \cite{burton}. También, puede encontrar el libro
con las soluciones a los ejercicios de \cite{rosen}.

Otra cosa que no me gusta del libro oficial es que no está maquetado con la
tecnología TeX, al contrario de lo que sucede hoy en día con la abrumadora
mayoría de textos de matemáticas.

Un libro que toca casi todos estos temas, pero de un modo algo más moderno,
es \cite{weissman}, que está muy bien, aunque existen algunas
partes del temario que no se encuentran en este.

En lo que respecta a la parte de la teoría de grafos, TKTK.

En lo que respecta a la parte de combinatoria, está muy bien
\cite{chuan-chong}. TKTK.




% -----------------------------------------------


También, he estado haciéndome mis propios apuntes, pues hay cosas que las
hago de un modo más moderno, como se explican en
\cite{weissman}, que es un libro que me encanta pero que difiere
en ciertas partes a la forma en que se presentan los contenidos en el libro
oficial que seguimos. En cualquier caso, la elaboración de unos apuntes de
gran calidad consume mucho tiempo y ese es tiempo que se pierde
innecesariamente en detrimento de otras opciones como hacer más ejercicios
sobre esta misma materia. Por tanto, siempre que pueda, usaré alguna fuente
que me sirva como alternativa a \cite{texto-uned}. Espero encontrar
alguna para la parte de la teoría de grafos. Para la parte de combinatoria,
creo que podría usar \cite{chuan-chong}, pues me parece muy bueno.



