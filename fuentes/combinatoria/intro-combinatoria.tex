



La parte de combinatoria de esta asignatura puede seguirse en gran medida
por \cite{brualdi}. Es un gran complemento para \cite{bujalance-costa}, el
libro oficial, ya que contiene bastantes más explicaciones y ejemplos
ilustrativos de todos los conceptos que presenta.

En cualquier caso, debe ser consciente de que en el texto oficial se
presentan algunos conceptos con mayor rigurosidad y abstracción, pero no
creo que esto suponga ningún problema para hacer los ejercicios, que es la
meta principal de esta asignatura.

Muchas veces, se dice que la combinatoria es ``el arte de contar''. Aunque
piense que se trata de una exageración, esta frase es bastante cierta TKTK.

En cualquier caso, existen otras definiciones más informativas, como la que
aparece en \cite{brualdi}:

\begin{quote}
  La combinatoria tiene que ver con disposiciones (\emph{arrangements}) de
  los objetos de un conjunto en patrones que satisfagan ciertas reglas.
\end{quote}








En combinatoria, es bastante frecuente que cambiando el punto de vista se
llegue fácilmente a una demostración. A una demostración de este tipo se la
suele llamar \emph{demostración combinatoria}. También he visto que hay quien
lo llama \engm{story proof}, como hace el profesor de la Universidad de
Harvard Joseph K.~Blitzstein en su libro de probabilidad. Esto se define de
un modo que me gusta bastante en \emph{Discrete Mathematics} de Rosen, pág.
412:

\begin{quote}
  Una demostración combinatoria de una identidad es una demostración que usa
  argumentos de contar para demostrar que ambos lados de la identidad
  cuentan el mismo número de objetos pero de formas distintas o una
  demostración que se basa en mostrar que existe una biyección entre los
  conjuntos de objetos contados por los dos lados de la identidad. A estos
  dos tipos de demostraciones se las llama, respectivamente,
  \emph{demostraciones por conteo doble} y \emph{demostraciónes biyectivas}.
\end{quote}

\noindent Creo que lo que se suele llamar \emph{demostraciones combinatorias}
es lo que en la cita anterior llaman \emph{demostraciones por conteo doble}.

A este respecto, algo que ha comentado varias veces el profesor Mario Pineda
es que, por ejemplo, cuando se nos plantea un problema y hablamos de que
``importa el orden'', en realidad eso es una interpretación que hacemos
nosotros. El problema en realidad, de forma intrínseca, no es que sea un
problema de cosas con orden. De hecho, muchos problemas se pueden resolver
tanto teniendo en cuenta el orden como sin tenerlo. Por ejemplo, esto sucede
con la demostración de que el cardinal de las partes de un conjunto es 2
elevado al cardinal de dicho conjunto. Se puede resolver tanto por
permutaciones con repetición como por sumas de combinaciones.

De hecho, cuando toma cierta soltura en combinatoria, puede optar por
resolver los problemas sin siquiera pensar en las fórmulas de permutaciones,
combinaciones, etc., sino que esas fórmulas las puede construir uno mismo a
partir de los principios que se presentan en el Capítulo
\ref{ch-comb-tecnicas-basicas}, de los que se deducen esas otras fórmulas.
Esto es lo más recomendable, pues le sirve también para estar preparado para
los casos en los que no pueda hacer uso de esas fórmulas ``enlatadas''.

Por tanto, tal y como verá, para resolver los ejercicios y problemas influye
mucho la subjetividad. Analizar las cosas desde cierto punto de vista
facilita mucho el trabajo en combinatoria. Pero no es que sea algo subjetivo
en realidad, pues el resultado de cada situación es objetivo y, por tanto,
siempre el mismo. Esto se explica en \cite{brualdi} al final de la página
43. La regla a seguir es comenzar por la regla más restrictiva.

También, debe saber que la combinatoria es un área de las matemáticas que
tiene aplicación en casi todas las demás. Lo más natural es pensar en su
aplicación a la probabilidad, pero realmente tiene aplicación en muchas
otras áreas.







