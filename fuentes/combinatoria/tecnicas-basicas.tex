




\section{Introducción}

Aquí, se presentan varios principios o reglas básicas de las que se deducen
luego varias fórmulas para casos más particulares; también las llamo
``técnicas {`enlatadas'}''. de combinatoria, que se presentan en el capítulo
siguiente. En cualquier caso, los principios de este capítulo son la base de
todo y conviene recordarlos. De hecho, son sencillos de recordar y de
demostrar. Si los recuerda, podrá deducir por usted mismo esas técnicas
``enlatadas'' del capítulo siguiente. Además, habrá veces en las que se
encuentre con un ejercicio que requiera del uso de estas técnicas básicas;
es decir, cuya resolución no tiene una forma abreviada.

Por mi parte, he incluido algunos principios que no se presentan en
\cite{bujalance-costa} pero sí aparecen en \cite{brualdi}, como el del
complementario, por ejemplo. Al fin y al cabo son sencillos y vienen muy
bien para resolver ejercicios. Incluso se usan en algunas demostraciones
posteriores.

Antes de comenzar a enunciar estos principios, me gustaría recordar un
concepto que seguramente ya conozca pero que no está de más presentar
formalmente.

\begin{deffinition}[Partición de un Conjunto]
  Dado un conjunto $S$, una \emph{partición} (\emph{partition}) de $S$ es una
  colección $S_1, S_2, S_3, \ldots, S_m$ de subconjuntos de $S$ tales que
  cada elemento de $S$ se encuentre en uno y solo uno de esos subconjuntos.
\end{deffinition}

Puede encontrar una definición de \emph{partición} que me gusta más en
\cite{newstead} pág. 200.

Se podría dar una definición alternativa, más simbólica, que es más probable
que sea la que conozca.

\begin{deffinition}[Partición de un Conjunto (definición alternativa)]
  Una partición de un conjunto $S$ es una colección de conjuntos $S_1, S_2,
  S_3, \ldots, S_m$ para los que se cumplen las propiedades

  \begin{alignat*}{2}
    S &= S_1 \cup S_2 \cup S_3 \cup \cdots \cup S_m \\
    S_i \cap S_j &= \emptyset \ \text{para todos} \ i, j \in \nset \
      \text{siendo} \ 1 \leq i \leq m, 1 \leq j \leq m, i \neq j
  \end{alignat*}
\end{deffinition}









\section{Algunos principios}

El principio siguiente viene a decir que, dada una partición de un conjunto,
el cardinal de dicho conjunto es igual a la suma de todas las partes de
esta.

\begin{theorem}[Principio de la Suma]\label{princ-suma}
  Dados $A_1, A_2, A_3, \ldots, A_n$ conjuntos finitos tales que para todo
  $i, j \in \nset$ con $1 \leq i \leq n$, $1 \leq j \leq n$ e $i \neq j$ se
  da $A_i \cap A_j = \emptyset$, se cumple

  $$ \card(\bigcup_{i=1}^n A_i) = \sum_{i=1}^n \card(A_i) $$
\end{theorem}

\noindent Por razones evidentes, también lo llaman Principio de la Adición.

Aunque algunos de esos conjuntos pueden ser vacíos, en la práctica, a esos
no los consideramos, tal y como verá en los ejercicios. Es decir, no sería
un impedimento que algunas de las partes sean conjuntos vacíos, aunque
también es cierto que, si se trata de un conjunto vacío, no se considera en
la partición.

¿Pero qué sucedería en el caso general, es decir, si esos conjuntos no son
disjuntos dos a dos? Eso se estudia en el capítulo \ref{ch-princ-incl-excl}.

Tal y como veremos con los ejercicios y problemas que hagamos luego, estos
conjuntos pueden representar a agrupaciones de cosas, pero, a veces, a una
misma cosa en un instante de tiempo diferente.

Algo que se debe tener en cuenta es que, tal y como se ve en el enunciado
del principio, se debe hacer una partición de un conjunto. Esa partición no
puede ser cualquiera. Aquí es donde entra en juego la astucia de quien
resuelve el ejercicio o problema. Se podría hacer una partición en la que
cada parte contuviese únicamente a un elemento del conjunto original, pero
eso no aportaría nada ya que contar esas particiones sería igual que contar
los elementos del conjunto original uno a uno. Tal y como se explica en
\cite{brualdi} en la pág. 28, hay que particionar al conjunto original en un
número no demasiado grande de particiones.

\begin{proof}
  La demostración la haremos por inducción.

  Primero, veamos el caso base, que lo haremos para $n = 2$. Suponemos los
  dos conjuntos siguientes:

  \begin{alignat*}{2}
    A_1 &= \{a_1, a_2, a_3, \ldots, a_m\} \\
    A_2 &= \{b_1, b_2, b_3, \ldots, b_p\}
  \end{alignat*}

  \noindent que, evidentemente, tienen los tamaños $\card(A_1) = m$ y
  $\card(A_2) = p$. Si son disjuntos, es decir, $A_1 \cap A_2 = \emptyset$,
  el conjunto unión de estos dos será

  $$ A_1 \cup A_2 = \{a_1, a_2, a_3, \ldots, a_m, b_1, b_2, b_3, \ldots,
  b_p\}$$

  \noindent ya que no había ningún elemento repetido. Por tanto, se tiene
  que

  $$ \card(A_1 \cup A_2) = m + p = \card(A_1) + \card(A_2) $$

  \noindent Por tanto, para este caso se cumpliría.

  Ahora, vamos al paso inductivo. Por hipótesis de inducción, se cumple para
  el caso del conjunto de $n$ elementos:

  $$ \card(\bigcup_{i=1}^n A_i) = \sum_{i=1}^n \card(A_i) $$

  \noindent y debemos comprobar que se cumple también para el caso de $n+1$
  elementos.

  Por la propiedad asociativa de la unión de conjuntos se tiene:

  $$ \bigcup_{i=1}^{n+1} A_i = \left(\bigcup_{i=1}^{n} A_i \right) \cup
  A_{n+1} $$

  Como, por hipótesis, se cumple para todo $i \in \nset$ siendo $1 \leq i
  \leq n$ que $A_i \cap A_{n+1} = \emptyset$, se tiene también, como sabemos
  por la teoría de conjuntos, que

  $$ \emptyset = \bigcup_{i=1}^n \emptyset = \bigcup_{i=1}^n \left(A_{n+1}
  \cap A_i\right) = A_{n+1} \cap\left( \bigcup_{i=1}^n A_i \right) $$

  \noindent Gracias a esto, podemos aplicar a la penúltima identidad lo que
  obtuvimos para el caso base, es decir, para $n = 2$. Tenemos

  $$ \card(\bigcup_{i=1}^{n+1}) = \card(\bigcup_{i=1}^n A_i) +
  \card(A_{n+1}) = \card(A_{n+1}) + \sum_{i=1}^n \card(A_i) =
  \sum_{i=1}^{n+1} \card(A_i) $$

  \noindent La penúltima igualdad se justifica por la hipótesis de
  inducción.
\end{proof}

Ahora, vamos a ver el Principio del Producto. Lo analizaremos primero para
el caso de dos conjuntos únicamente. Después, lo generalizaremos.

\begin{theorem}[Principio del Producto para 2 Conjuntos]
  Dados dos conjuntos finitos $A$ y $B$, el cardinal del producto cartesiano
  $A \times B$ es el producto de sus cardinales respectivos. Es decir,

  $$ \card(A \times B) = \card(A) \cdot \card(B) $$
\end{theorem}

Si se fija, esto no es más que una consecuencia del Principio de la Suma.
Supongamos que el conjunto $A$ está constituido por los elementos $a_1, a_2,
a_3, \ldots, a_p$ y, $B$, por $b_1, b_2, b_3, \ldots, b_q$. Como es
evidente, se tiene que $\card(A) = p$ y $\card(B) = q$. Particionamos
entonces a $A \times B$ en $S_1, S_2, S_3, \ldots, S_p$, siendo $S_i$ el
conjunto de pares ordenados con primer elemento $a_i$, para un $i \in \nset$
siendo $1 \leq i \leq p$. Por tanto, el tamaño de cada $S_i$ es $q$.

Aplicando entonces el Principio de la Suma, se tiene

\begin{alignat*}{2}
  \card(A \times B)
    &= \card(S_1) + \card(S_2) + \card(S_3) + \cdots + \card(S_p) \\
    &= q + q + q + \cdots + q = p \cdot q
\end{alignat*}

\noindent La última igualdad se justifica por el hecho de que, para los
números naturales, la multiplicación por un número $p \in \nset$ es lo mismo
que la suma repetida $p$ veces.

Conviene aquí consultar esta sección por \cite{brualdi} ya que pone varios
ejercicios muy básicos pero que vienen muy bien para comprender el tipo de
razonamiento que se debe seguir en los ejercicios en los que se aplican
estos principios. Sin embargo, en \cite{bujalance-costa} avanzan de un modo
bastante más rápido y puede resultar complicado para quien no tenga cierta
base.

La diferencia principal entre el Principio de la Suma y el del Producto está
en que, en el primero, se hace una sola selección o forma de aparecer,
mientras que en el otro se hacen varias selecciones independientes. Esas
selecciones podrían hacerse en secuencia o simultáneamente, dependiendo del
ejercicio o problema en particular.

Ahora, vamos a ver el principio anterior pero para el caso generalizado
finito.

\begin{theorem}[Principio del Producto]
  Sea $A_1, A_2, A_3, \ldots, A_n$ una colección de conjuntos finitos. Se
  cumple

  $$ \card(A_1 \times A_2 \times A_3 \times \cdots \times A_n) =
  \prod_{i=1}^n \card(A_i) $$
\end{theorem}

Advierta que este principio, al contrario de lo que sucede en el Teorema
\ref{princ-suma}, no se requiere que los conjuntos sean disjuntos dos a dos.

\begin{proof}
  Se demostrará por inducción.

  El caso base será $n = 2$. Veamos primero la definición del producto
  cartesiano de dos conjuntos.

  $$ A_1 \times A_2 = \{(a, b) \st a \in A_1 \ \text{y} \ b \in A_2\} $$

  Tenemos los conjuntos

  \begin{alignat*}{2}
    A_1 &= \{a_1, a_2, a_3, \ldots, a_m\} \\
    A_2 &= \{b_1, b_2, b_3, \ldots, b_p\}
  \end{alignat*}

  \noindent con $\card(A_1) = m$ y $\card(A_2) = p$. Para cada $i \in \nset$
  siendo $1 \leq i \leq m$ se tienen $p$ pares ordenados:

  $$ (a_i, b_1), (a_i, b_2), (a_i, b_3), \ldots, (a_i, b_p) $$

  Esto se tiene para cada uno de los $m$ elementos $a_i$. Por tanto, en
  total existirán $m \cdot p$ pares ordenados, que es lo mismo que
  $\card(A_1) \cdot \card(A_2)$. Por tanto, se tiene que

  $$ \card(A_1 \times A_2) = \card(A_1) \cdot \card(A_2) $$

  \noindent con lo que queda demostrado que se cumple para el caso base $n =
  2$.

  Veamos ahora el paso inductivo. Suponemos como hipótesis de inducción que
  se cumple para un caso genérico $n > 2$:

  $$ \card(A_1 \times A_2 \times A_3 \times \cdots \times A_n) =
  \prod_{i=1}^n \card(A_i)$$

  Veamos ahora si se cumple para $n+1$. Por la propiedad asociativa del
  producto cartesiano, se tiene

  $$ A_1 \times A_2 \times A_3 \times \cdots \times A_n \times A_{n+1} =
  (A_1 \times A_2 \times A_3 \times \cdots \times A_n) \times A_{n+1} $$

  Por tanto, aplicando a esto lo que demostramos en el caso base, es decir,
  como si fuese $n = 2$, tenemos

  \begin{alignat*}{2}
    \card(A_1 \times A_2 \times A_3 \times \cdots \times A_n \times A_{n+1})
      &= \card( (A_1 \times A_2 \times A_3 \times \cdots \times A_n) \times
        A_{n+1}) \\
      &= \card(A_1 \times A_2 \times A_3 \times \cdots \times A_n) \cdot
        \card(A_{n+1}) \\
      &= \card(A_{n+1}) \cdot \prod_{i=1}^n \card(A_i) \\
      &= \prod_{i=1}^{n+1} \card(A_i)
  \end{alignat*}

  \noindent La penúltima igualdad se justifica por la hipótesis de inducción
  y la última por la propiedad asociativa del producto de números naturales.
\end{proof}

Ejercicio. ¿Cuántos números naturales $n$ existen entre 5000 y 10000, ambos
incluidos, teniendo todos sus dígitos diferentes?

El número 10000 nos entorpece en cierta medida la forma de abordar el
problema por combinatoria, ya que tiene 5 dígitos, mientras que todos los
demás son números de 4 dígitos. La solución es muy sencilla: contarlo a
parte. Como dicho número no cumple la condición pues 10000 tiene dígitos
repetidos, ni siquiera tenemos que contar nada aparte. Por tanto, el
problema lo podemos reformular diciendo que se debe hallar el número de
números naturales sin dígitos repetidos entre 5000 y 9999. Ahora, todos son
de 4 dígitos, lo cual nos resulta mucho más cómodo.

Ahora, hay que usar la astucia para seleccionar un orden a la hora de
abordar por combinatoria el ejercicio. Esto nos hará que la resolución sea
más breve. Vamos a usar el Principio del Producto, comenzando por las
unidades de millar. Estas pueden tomar un número entre 5 y 9, ambos
incluidos. Se tienen 5 posibilidades. Una vez seleccionado ese dígito,
seleccionaríamos las centenas. Como no se puede repetir el dígito que hemos
usado, habrá 9 posibilidades, que es una menos que las 10 que hay entre 0 y
9 incluyendo a ambos. Para las decenas, habría 8 y, para las unidades, 7.
Por tanto, por el Principio del Producto, se tendría que el resultado al
ejercicio es

$$ 5 \cdot 9 \cdot 8 \cdot 7 = 2520 $$

De haber elegido otro orden de los dígitos para aplicar el Principio del
Producto, habría que dividor el problema en dos casos, lo cual es más
engorroso que lo que acabamos de hacer. Concretamente, suponga que comienza
por las unidades. En ese caso, el dígito que se selecciona puede afectar a
la cuenta de las unidades de millar, pues, si es un número entre 0 y 4, no
le ``quita'' una posibilidad, pero, si está entre 5 y 9, sí. Por tanto, lo
que habría que hacer es dividir el conteo en dos casos. TKTK.

En este última forma de resolver el problema, al contrario que en la
primera, se debe recurrir al uso del Principio de la Suma. Si es posible, es
mejor ahorrárselo y así evitar recurrir a analizar la casuistica.

En cualquier caso, advierta que ambas resoluciones dan lugar a una misma
solución, por lo que, aunque la forma de abordar el problema contenga un
componente de subjetividad, el resultado es objetivo.

Esto se explica en \cite{brualdi} al final de la pág. 33. Eso, si es que se
puede, pues hay veces en las que no se puede. En cualquier caso, lo que se
debe probar es a contar primero las elecciones más restrictivas.

\begin{theorem}[Principio de la Resta]
  Dado un conjunto $U$ y un subconjunto de este, $A \subseteq U$. Sea

  $$ \overline{A} = U \setminus A = \{x \in U \st x \not\in A\} $$

  \noindent el complementario de $A$ respecto de $U$. Se tiene que

  $$ \card(A) = \card(U) - \card(\overline{A}) $$
\end{theorem}

Ese conjunto $U$ suele ser un conjunto que surge de forma natural como el
conjunto de todos los objetos que se están considerando en el ejercicio o
problema. A un conjunto de este tipo se le suele llamar \emph{conjunto
universal} (\emph{universal set}).

En cuanto a la demostración, es muy sencilla. Basta con darse cuenta de que
esto es una aplicación directa del Principio de la Suma en el que estamos
contando el tamaño de $U$ a partir de la partición formada por un
subconjunto cualquiera $A$ de $U$ y del complementario de este respecto a
$U$, es decir, $\overline{A} = U \setminus A$. Una vez hecho esto, solo hay
que manipular la expresión algebraica para aislar a $\card(A)$.

Este principio es muy útil cuando es más fácil contar el número de objetos
en $U$ y $\overline{A}$ que en $A$.

\begin{theorem}[Principio de la División]
  Sea $S$ un conjunto finito sobre el que existe una partición tal que cada
  una de las partes tiene el mismo número de elementos. El número de partes
  $k \in \nset$ en la partición será, necesariamente,

  $$ k = \frac{\card(S)}{\text{número de objetos en cada parte}} $$
\end{theorem}

Es evidentemente una consecuencia directa del Principio de la Suma, para los
casos en los que la partición está formada por conjuntos del mismo tamaño.

Aunque sea evidente, seguramente sea más fácil de comprender haciendo la
multiplicación, en lugar que la división. Es lo mismo que sucede en teoría
de números con los conceptos de \emph{factor} y \emph{múltiplo}.

El teorema siguiente se usa muy a menudo también y puede usarse para
resolver algunos problemas muy creativos. Aunque lo llamamos Principio de
Distribución, es más conocido como Principio del Palomar (\eng{Pigeonhole
Principle}), porque se suele explicar mediante un problema sobre un palomar.
También, se le conoce como el Principio del Cajón de Dirichlet.

\begin{theorem}[Principio de la Distribución]\label{princ-distribucion}
  Dados $m, n, p \in \nset^{+}$. Para cualquier distribución de $n \cdot p +
  m$ elementos en $n$ conjuntos habrá al menos un conjunto con $p + 1$
  elementos.
\end{theorem}

Como se le suele conocer más como el Principio del Palomar, normalmente se
habla de cajas en lugar de conjuntos y de objetos en lugar de elementos.
Bueno, imagino que a estas alturas tendrá cierta práctica con el
razonamiento abstracto y no se liará porque yo lo llame de otro modo.

Aunque es bastante evidente, tiene una demostración.

\begin{proof}
  De entre todas las distribuciones posibles que se podrían hacer, con la
  que más nos acercaríamos a que no se cumpliera esta proposición sería en
  la que los primeros $np$ elementos que tomemos los repartamos de forma
  homogénea en los $n$ conjuntos. En ese caso, cada conjunto tendría $p$
  elementos.

  Pero hay que añadir $m \in \nset$ elementos siendo $m > 0$, entre todos
  esos $n$ conjuntos, con lo que al menos uno de estos $n$ conjuntos tendrá
  $p+1$ elementos.
\end{proof}

\iffalse
Forma alternativa de enunciarlo.

\begin{theorem}[Principio de la Distribución]\label{princ-distribucion}
  Dados $c, e \in \nset^{+}$ con $c \nmid e$. Para cualquier distribución de
  $e$ elementos en $c$ conjuntos se tendrá algún conjunto con mínimo un
  elemento más que el cociente de la división de $e$ entre $c$.
\end{theorem}

Sería parecido a lo que se explica en el Teorema de la División Entera con
Resto solo que el cociente y el resto no están tan restringidos aquí.
\fi

El corolario siguiente, aun siendo una consecuencia directa del Principio de
la Distribución, es muy útil para resolver ciertos problemas.

\begin{corollary}
  Dados $m_1, m_2, m_3, \ldots, m_i, \ldots m_n \in \nset^{+}$, siendo un
  índice $i \in \nset$ con $1 \leq i \leq n$. Si se cumple

  $$ \frac{1}{n} \sum_{i=1}^n m_i > p $$

  \noindent se tiene que $m_i > p$ para algún $i$.
\end{corollary}

\begin{proof}
  La primera de las desigualdades se puede escribir de este otro modo:

  $$ \sum_{i=1}^n m_i > np $$

  \noindent Por la definición de la relación ``mayor que'', esto es lo mismo
  que decir que, para un $k \in \nset$ siendo $k > 0$,

  $$ \sum_{i=1}^n m_i = np + k $$

  \noindent y, esto, según el Teorema \ref{princ-distribucion}, hace que
  entre los $n$ números $m_i$ exista al menos uno tal que $m_i > p$.
\end{proof}

Ejercicio. Dado el conjunto de números naturales $A = \{1, 2, 3, \ldots,
40\}$, demuestre que cualquier subconjunto sin remplazo de 21 elementos del
mismo posee 2 que son uno divisor del otro.

Vamos a crear 20 cajas imaginarias para luego distribuir una selección
cualquiera de 21 elementos de A en estas. Vamos a crear una regla para
introducir los números en las cajas según la cual cualquier par de elementos
en una misma caja serán necesariamente uno divisor del otro.

Una vez hecho esto, se podrá aplicar el Principio de la Distribución para
demostrar que para cualquier distribucion de esos 21 números en las 20 cajas
al menos habrá una caja con más de un elemento. No nos preocupa si los
elementos de una caja son divisibles por los de otra, pues basta con lo
anterior para la demostración que deseamos hacer.

¿Pero cómo es esa forma de asignar elementos a las cajas?

Vamos a llamar $n$ a un elemento genérico de los 21 seleccionados de $A$.
Por tanto, $n \in \mathbb{N}$ y, también, $1 \leq n \leq 40$.

Descomponemos a $n$ en factores primos y hacemos dos agrupaciones de su
descomposición. Por un lado, se tiene el factor $2^k$ para algún $k \in
\nset$, que se deja tal y como está en la descomposición canónica en
factores primos. Por el otro, se tiene al factor $m$, que es el producto de
todos los demás factores de dicha descomposición canónica. Al ser $m$ un
producto de números primos distintos de 2, será un producto de números
impares y, por tanto, será también un número impar.

Así, se tiene que, para cualquier $n \in A$,

$$ n = 2^k \cdot m $$

\noindent Advierta que, como el menor factor $2^k$ es $2^0 = 1$ y $1 \leq n
\leq 40$, se tiene que $1 \leq m \leq 39$. Es decir, $m$ son todos los
números impares entre 1 y 39.

Veamos algunos casos particulares. Si $n$ es una potencia de 2, irá entonces
en la caja 1, ya que tendríamos que $m = 1$. En caso contrario, irá en una
de las demás cajas, al ser $m \neq 1$. Por ejemplo,

\begin{equation*}
  \begin{array}{lll}
    24 &= 2^3 \cdot 3     & \text{Número} \ 24 \ \text{caja} \ 3 \\
    25 &= 2^0 \cdot 25    & \text{Número} \ 25 \ \text{caja} \ 25 \\
    39 &= 2^0 \cdot 39    & \text{Número} \ 39 \ \text{caja} \ 39 \\
    40 &= 2^3 \cdot 5     & \text{Número} \ 40 \ \text{caja} \ 5
  \end{array}
\end{equation*}

Lo que queda por demostrar es que, para cualquier caja, los elementos en su
interior son uno divisor del otro, tal y como mencionamos. Supongamos que,
en la caja $m$ se han introducido los números $n_1$ y $n_2$. Tendríamos que

$$ n_1 = 2^k \cdot m \quad \text{y} \quad n_2 = 2^l \cdot m $$

\noindent para algún par de números $k, l \in \mathbb{N}$. También, podemos
asumir que $k > l$, sin pérdida de generalidad. Advierta que, si $k$ y $l$
fuesen iguales, se trataría de $n_1 = n_2$, con lo que ese caso no se
contempla y, por tanto, han de ser distintos necesariamente. Entonces, se
cumple

$$ \frac{n_1}{n_2} = \frac{2^k \cdot m}{2^l \cdot m} = 2^{k-l} $$

\noindent siendo $k - l \in \mathbb{N}$ y $k - l > 0$, por lo que $n_2 \mid
n_1$.

Otro ejercicio que se usa luego como base de otros es el siguiente. Está
también en \cite{brualdi} pág. 71.

Demuestre que, dada una sucesión de números enteros $a_1, a_2, a_3, \ldots,
a_m$, existe un tramo de esta tal que la suma de los elementos del tramo sea
un múltiplo del número de elementos de la sucesión.

Se puede expresar también de un modo más simbólico, para evitar las posibles
ambigüedades. Dada una sucesión de $m$ números enteros $a_1, a_2, a_3,
\ldots, a_m$ existen dos enteros $k$ y $l$ con $0 \leq k < l \leq m$ tales
que $a_{k+1} + a_{k+1} + \cdots + a_l$ es múltiplo de $m$.

Consideremos las sumas de los $m$ tramos comenzando desde el 1.

$$ a_1, a_1 + a_2, a_1 + a_2 + a_3, \ldots, a_1 + a_2 + a_3 + \cdots + a_m
$$

Si alguna de estas sumas es múltiplo de $m$, entonces se cumple lo que
deseábamos.

Vayámonos al caso más desfavorable. Supongamos que cada una de estas sumas
tiene un resto distinto de 0, cuando se la divide entre $m$. Por tanto,
dicho resto será uno de los elementos siguientes: $1, 2, 3, \ldots, m-1$.
Como tenemos $m$ sumas, por aplicación del Principio del Palomar tendremos
entonces que dos de esas sumas tendrán un mismo resto en esa división entre
$m$.

Supongamos que esas dos sumas con el mismo resto se dan para los índices $k$
y $l$, con $k \neq l$, siendo, sin pérdida de generalidad, $k < l$. Se
tiene, entonces,

\begin{alignat*}{2}
  a_1 + a_2 + a_3 + \cdots + a_k = bm + r \\
  a_1 + a_2 + a_3 + \cdots + a_l = cm + r \\
\end{alignat*}

\noindent para algun par de $b, r \in \zset$. Si los restamos, tenemos

$$ a_{k+1} + a_{k+2} + a_{k-3} + \cdots + a_l = (c - b)m $$

\noindent con lo que queda demostrado que $a_{k+1} + a_{k+2} + a_{k-3} +
\cdots + a_l$ es múltiplo de $m$.

En los problemas del libro de problemas, en el 4 hace uso de algo que se ve
en la sección siguiente. Usa combinaciones de 8 elementos tomados de 2 en 2
para calcular el número de partidos que se juegan en el torneo. Son 28.






