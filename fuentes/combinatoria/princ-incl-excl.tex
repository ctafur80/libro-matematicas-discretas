



A pesar de su aparente simplicidad, el Principio de Inclusión-Exclusión es
una herramienta bastante potente del análisis combinatorio.

En su forma más simple, viene a decir que la relación entre el número de
elementos de la unión y la intersección de dos conjuntos es

$$ \card(A \cup B) = \card(A) + \card(B) - \card(A \cap B) $$

Seguramente se haya dado cuenta de que se trata de una generalización del
Principio de la Suma, para los casos en los que los conjuntos no son
disjuntos. De hecho, este resultado surge de forma directa del Principio de
la Suma y el Principio de la Resta. En realidad, el Principio de
Inclusión-Exclusión es un resultado de un caso más general.

Vamos a extender esto mismo a un número genérico de $m$ propiedades. Sean
$P_1, P_2, P_3, \ldots, P_m$ propiedades que hacen referencia a objetos de
$S$ (no quiere decir que todos cumplan todas estas propiedades), y sea

$$ A_i = \{x \st x \in S \ \text{y} \ x \ \text{cumple la propiedad} \ P_i\}
$$

\noindent para $i = 1, 2, 3, \ldots, m$, el subconjunto de objetos de $S$
que tienen la propiedad $P_i$, además de otras, posiblemente.

Entonces, $A_i \cap A_j$ es el subconjunto de objetos de $S$ que cumplen las
propiedades $P_i$ y $P_j$, y posiblemente otras más. $A_i \cap A_j \cap A_k$
será el subconjunto de objetos de $S$ que cumplen las propiedades $P_i$,
$P_j$ y $P_k$, además de otras posiblemente. Y así sucesivamente hasta
llegar a $A_1 \cap A_2 \cap A_3 \cap \cdots \cap A_m$, que sería el conjunto
de los objetos de $S$ que cumplen todas las propiedades. Por otra parte, el
subconjunto de objetos que no cumplen ninguna de las propiedades $P_1, P_2,
P_3, \ldots, P_m$ será

$$ \overline{A_1} \cap \overline{A_2} \cap \overline{A_3} \cap \cdots \cap
\overline{A_m} $$

El Principio de Inclusión-Exclusión muestra cómo contar el número de objetos
en este conjunto al ir contando conjuntos de objetos de acuerdo a
propiedades que sí cumplen. En cierto modo, se podría decir que este
principio invierte la forma de contar.

\begin{theorem}[Principio de Inclusión-Exclusión]
  El número de objetos del conjunto $S$ que no cumplen ninguna de las
  propiedades $P_1, P_2, P_3, \ldots, P_m$ viene dado por la expresión
  alternante siguiente:

  \begin{alignat*}{2}
    \card(\overline{A_1} \cap \overline{A_2} \cap \overline{A_3} \cap
      \cdots \cap \overline{A_m})
    &= \card(S) - \sum \card(A_i) + \sum \card(A_i \cap A_j) \\
    &- \sum \card(A_i \cap A_j \cap A_k) + \cdots \\
    &+ ({-1})^m \sum \card(A_1 \cap A_2 \cap A_3 \cap \cdots \cap A_m)
  \end{alignat*}

  \noindent donde la primera suma es sobre todos los subconjuntos de 1
  elemento, $\{i\}$, de $1, 2, 3, \ldots, m$, la segunda, de los de 2
  elementos, $\{i, j\}$, de $1, 2, 3, \ldots, m$, la tercera, de los de 3
  elementos, $\{i, j, k\}$, de $1, 2, 3, \ldots, m$, y así hasta la suma de
  todos los subconjuntos de $m$ elementos de $1, 2, 3, \ldots, m$, que en
  este último caso solo existe uno.
\end{theorem}

Hemos dejado los símbolos de sumatorio sin poner índices de forma
deliverada, ya que es algo complicado lo que sucede ahí. Para explicarlo,
vamos a mostrar algunos casos particulares.

Para $m = 3$, se tiene

\begin{alignat*}{2}
  \card(\overline{A_1} \cap \overline{A_2} \cap \overline{A_3})
    &= \card(S) - [\card(A_1) + \card(A_2) + \card(A_3)] + \\
    &+ [\card(A_1 \cap A_2) + \card(A_1 \cap A_3) + \card(A_2 \cap A_3)] + \\
    &- \card(A_1 \cap A_2 \cap A_3)
\end{alignat*}

Como ve, se toman los subconjuntos de 2 elementos sin tener en cuenta el
orden, ya que la intersección de conjuntos cumple la propiedad conmutativa.
Sucede lo mismo con todos los tipos de subconjuntos.

Si se fija en el número de términos a la derecha del igual, son 8, que es lo
mismo que $2^3$. Luego veremos por qué es así.

Veamos el caso para $m = 4$.

\begin{alignat*}{2}
  \card(\overline{A_1} \cap \overline{A_2} \cap \overline{A_3} \cap
    \overline{A_4})
    &= \card(S) - [\card(A_1) + \card(A_2) + \card(A_3) + \card(A_4)] + \\
    &+ [\card(A_1 \cap A_2) + \card(A_1 \cap A_3) + \card(A_1 \cap A_4) \\
    &+ \card(A_2 \cap A_3) + \card(A_2 \cap A_4) + \card(A_3 \cap A_4)] \\
    &- [\card(A_1 \cap A_2 \cap A_3) + \card(A_1 \cap A_2 \cap A_4) \\
    &+ \card(A_1 \cap A_3 \cap A_4) + \card(A_2 \cap A_3 \cap A_4)] \\
    &+ \card(A_1 \cap A_2 \cap A_3 \cap A_4)
\end{alignat*}

En este caso hay 16 términos a la derecha. En el caso general, se tienen

$$ \sum_{k=1}^m {m \choose k} = 2^m $$

\noindent tal y como explica el Teorema TKTK.

\begin{proof}
  El lado izquierdo de la ecuación cuenta el número de objetos de $S$ que no
  cumplen ninguna de las propiedades. Al igual que como se hizo en el caso
  especial $m = 2$, podemos establecer la validez del a ecuación mostrando
  que que un objeto que no cumple ninguna de las propiedades $P_1, P_2, P_3,
  \ldots, P_m$ hace una contribución neta de 1 al lado derecho, y uno que
  cumpla al menos una de las propiedades realiza una contribución neta de 0.

  Consideremos primero un objeto $x$ que no cumpla ninguna de las
  propiedades. Su contribución al lado derecho de la igualdad será

  $$ 1 - 0 + 0 - 0 + \cdots + ({-1})^m 0 = 1 $$

  \noindent puesto que está en $S$ pero no está en ninguno de los otros
  conjuntos.

  Ahora, consideremos un objeto $y$ que cumple exactamente con $n$ de las
  propiedades, siendo $1 \leq n \leq m$. La contribución de $y$ a $\card(S)$
  será 1, ya que $y \in S$. Esto es lo mismo que ${n \choose 0}$.

  Su contribución a $\sum \card(A_i)$ es $n$, que es lo mismo que ${n
  \choose 1}$ ya que cumple exactamente $n$ de las propiedades y, por tanto,
  será un miembro de $n$ de los subconjuntos $A_1, A_2, A_3, \ldots, A_m$.

  La contribución de $y$ a $\sum \card(A_i \cap A_j)$ es ${n \choose 2}$ ya
  que podemos seleccionar de ${n \choose 2}$ formas un par de las
  propiedades que cumple $y$. Debido a esto, $y$ será miembro únicamente de
  ${n \choose 2}$ de estos subconjuntos $A_i \cap A_j$.

  La contribución de $y$ a $\sum \card(A_i \cap A_j \cap A_k)$ es ${n
  \choose 3}$ y así sucesivamente.

  Por tanto, la contribución neta de $y$ al lado derecho de la igualdad será

  $$ {n \choose 0} - {n \choose 1} + {n \choose 2} - {n \choose 3} + \cdots
  + ({-1})^m {n \choose m} $$

  \noindent que es lo mismo que

  $$ {n \choose 0} - {n \choose 1} + {n \choose 2} - {n \choose 3} + \cdots
  + ({-1})^n {n \choose n} $$

  \noindent ya que $n \leq m$ y ${n \choose k} = 0$ si $k > n$. Debido a que
  esta última expresión es, según la identidad TKTK, igual a 0, la
  contribución neta de $y$ al lado derecho de la igualdad será 0 si $y$
  cumple al menos una de las propiedades.
\end{proof}

\begin{corollary}
  El número de objetos de un conjunto $S$ que cumplen al menos una de las
  propiedades $P_1, P_2, P_3, \ldots, P_m$ viene dado por

  \begin{alignat*}{2}
  \card(A_1 \cup A_2 \cup A_3 \cup \cdots \cup A_m)
    &= \sum \card(A_i) - \sum \card(A_i \cap A_j) \\
    &+ \sum \card(A_i \cap A_j \cap A_k) \\
    &+ ({-1})^{m+1} \sum \card(A_1 \cap A_2 \cap A_3 \cap \cdots \cap A_m)
  \end{alignat*}

  \noindent donde esas sumas son las mismas que las del teorema anterior.
\end{corollary}

\begin{proof}
  El conjunto $A_1 \cup A_2 \cup A_3 \cup \cdots \cup A_m$ está formado por
  todos los objetos en $S$ que poseen al menos una de las propiedades.
  También,

  $$ \card(A_1 \cup A_2 \cup A_3 \cup \cdots \cup A_m) = \card(S) -
  \card(\overline{A_1 \cup A_2 \cup A_3 \cup \cdots \cup A_m}) $$

  Por las Leyes de De Morgan, se tiene que

  $$ \overline{A_1 \cup A_2 \cup A_3 \cup \cdots \cup A_m} = \overline{A_1}
  \cap \overline{A_2} \cap \overline{A_3} \cap \cdots \cap \overline{A_m} $$

  \noindent y, por tanto,

  $$ \card(A_1 \cup A_2 \cup A_3 \cup \cdots \cup A_m) = \card(S) -
  \overline{A_1} \cap \overline{A_2} \cap \overline{A_3} \cap \cdots \cap
  \overline{A_m} $$

  Ahora, solo queda combinar esta ecuación con la del teorema anterior y
  tenemos

  \begin{alignat*}{2}
  \card(A_1 \cup A_2 \cup A_3 \cup \cdots \cup A_m)
    &= \sum \card(A_i) - \sum \card(A_i \cap A_j) \\
    &+ \sum \card(A_i \cap A_j \cap A_k) \\
    &+ ({-1})^{m+1} \sum \card(A_1 \cap A_2 \cap A_3 \cap \cdots \cap A_m)
  \end{alignat*}
\end{proof}














% ----------------------------------------------------




\iffalse
Sean $S$ un conjunto finito y $P_1, P_2, P_3, \ldots, P_n$ propiedades que
cada uno de los elementos de $S$ puede satisfacer o no. Para cada $i = 1, 2,
3, \ldots, n$, sea

$$ S_i = \{x \in S \st x \ \text{satisface} \ P_i\} $$

Entonces,

$$ \bigcup_{i=1}^n S_i = \{x \in S \st x \ \text{satisface al menos una de
las propiedades} \ P_i \ \text{para} \ i=1, 2, 3, \ldots, n\} $$

$$ \bigcap_{i=1}^n (S - S_i) = \bigcap_{i=1}^n S'_i = \{x \in S \st x \
\text{no satisface ninguna de las propiedades} \ P_i \ \text{para} \ i=1, 2,
3, \ldots, n\} $$

\noindent donde $S'_i$ es lel complementario de $S_i$ en $S$.




\begin{theorem}[Principio de Inclusión-Exclusión]
Con la notación anterior, se tiene que

$$ \card(\bigcap_{i=1}^n S'_i) = \card(S) + ({-1})^n \sum_{k=1}^n
\card(\bigcap_{i=1}^k S_i) $$

\noindent donde para $2 \leq k \leq n$ las sumas

$$ \sum_{i=1}^k \card(S_{i1} \cap S_{i2} \cap S_{i3} \cap \cdots \cap
S_{ik}) $$

\noindent se extienden a todas las combinaciones de orden $k$, $\{i_1, i_2,
i_3, \ldots, i_k\}$, de $\{1, 2, 3, \ldots, n\}$.
\end{theorem}

La ecuación del teorema sería, en forma más extendida,

\begin{alignat*}{2}
  \card(\bigcap_{i=1}^n S'_i)
    &= \card(S) - \sum_{i=1}^n \card(S_i) + \sum_{i=1}^n \card(S_{i1} \cap S_{i2}) \\
    &- \sum_{i=1}^n \card(S_{i1} \cap S_{i2} \cap S_{i3}) + \cdots \\
    &+ ({-1})^k \sum_{i=1}^n \card(S_{i1} \cap S_{i2} \cap S_{i3} \cap \cdots \cap S_{ik}) \\
    &+ \cdots + ({-1})^n \sum_{i=1}^n \card(S_1 \cap S_2 \cap S_3 \cap \cdots \cap S_n)
\end{alignat*}

\begin{proof}
  Se va a ver que cada $x \in S$ contribuye con la misma cantidad sobre los
  miembros de la igualdad del teorema.

  El proceso se puede separar en dos pasos:

  \begin{enumerate}
    \item $x$ no satisface ninguna de las propiedades $\{P_1, P_2, P_3,
    \ldots, P_n\}$, esto implica $x \in (S'_1 \cap S'_2 \cap S'_3 \cap
    \cdots \cap S'_n)$, luego $x$ contribuye con 1 al lado izquierdo de la
    igualdad. Puesto que $x \in S$ y $x \not\in S_i$, para todo $i = 1, 2,
    3, \ldots, n$, se tiene que $x$ contribuye con 1 al lado derecho.

    \item $x$ satisface $t$ de las propiedades $\{P_1, P_2, P_3, \ldots,
    P_n\}$ siendo $1 \leq t \leq n$. Entonces, $x \not\in (S'_1 \cap S'_2
    \cap S'_3 \cap \cdots \cap S'_n)$, luego la aportación de $x$ al lado
    izquierdo de la igualdad del teorema es cero. Puesto que $x \in S$, esto
    implica que la contribución de $x$ a $\card(S)$ es 1. Además, $x$
    pertenece a $t$ de los conjuntos $\{S_1, S_2, S_3, \ldots, S_n\}$ y, por
    lo tanto, su contribución a $\sum \card(S_i)$ es $t$. El número de veces
    que se cuenta $x$ en la suma

    $$ \sum \card(S_{i1} \cap S_{i2}) $$

    \noindent donde la suma está extendida a las combinaciones de dos
    elementos de $\{1, 2, 3, \ldots, t\}$ es $C(t, 2)$. En el caso genérico,
    la contribución de $x$ a

    $$ \sum \card(S_{i1} \cap S_{i2} \cap S_{i3} \cap \cdots \cap S_{ik}) $$

    \noindent siendo $3 \leq k \leq t$ donde la suma está extendida a las
    combinaciones de $k$ elementos, $\{i_1, i_2, i_3, ldots, i_k\}$ de $\{1,
    2, 3, \ldots, t\}$ es $C(t, k)$.

    Luego, la contribución total de $x$ a la expresión de la derecha es

    $$ C(t, 0) - C(t, 1) + C(t, 2) - C(t, 3) + \cdots + ({-1})^t C(t, t) $$

    Por el corolario 3.3.7, con $x = {-1}$, esta suma es nula. Por tanto,
    queda demostrada la igualdad.
  \end{enumerate}














\end{proof}
\fi









