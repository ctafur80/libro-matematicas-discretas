


Una palabra que aparece mucho en la combinatoria es \emph{disposiciones}
(\emph{arrangements}), así como \emph{situaciones}, \emph{posibilidades}, etc.
Puede que no sean cosas que veamos, sino posibilidades en distintos
instantes de tiempo. TKTK.

A lo largo de toda la parte de combinatoria, hablaremos de \emph{repetición}
(\emph{repetition}), pero esto sería también válido para las situaciones en
las que se incluya el \emph{remplazo} (\emph{replacement}) en las reglas TKTK.

En lo que respecta a la terminología del capítulo 2, sobre los conceptos de
permutaciones, combinaciones, etc., sobre la terminología hay algo de
discrepancia entre los distintos textos y entre el inglés y el español. Así,
por ejemplo, a las $k$-\emph{permutations} \emph{partial permutations} las
llamamos \emph{variaciones}, pero también hay gente que en español las llama
\emph{permutaciones}.

Algo que está bastante claro es que se sigue casi siempre la regla de, si no
se les acompaña de ``con repetición'', o algo similar, por ejemplo, ``con
remplazo'', se supone que se trata de permutaciones, combinaciones, etc. sin
repetición, o sin remplazo, si lo prefiere. 

Otra particularidad de la terminología es que muchas veces se deja al
contexto que termine por determinar al concepto al que nos referimos
exactamente. Así, si hablan de \emph{permutations}, si el número de
elementos y el orden son iguales, nos referiremos a \emph{permutations}
realmente, cosa que hay quien llama \emph{total permutations}. Sin embargo,
si por el contexto se tiene que el orden es un valor $k$ que es menor que el
tamaño del conjunto del que se toman, serán lo que se designa como
$k$-\emph{permutations}. A estas últimas también hay quien las llama
\emph{partial permutations}. Por cierto, en este caso, al no especificar,
serían sin repetición, tal y como dijimos antes.

En cuanto a las diferencias entre el español y el inglés, lo más común es
ver que a las \emph{partial permutations} las llamemos \emph{variaciones}, y, a
las \emph{total permutations}, \emph{permutaciones}.

Aunque esta es la terminología con más aceptación en español, me gusta más
hacerla más cercana al inglés, y llamo \emph{permutaciones} tanto a las
variaciones como a las permutaciones. De hecho, creo que existe cierta
incoherencia en la terminología en español, pues, por ejemplo, a las
\emph{circular permutations} de un tamaño $k$ menor que el tamaño del
conjunto del que se toman las llamamos \emph{permutaciones circulares} en lugar
de \emph{variaciones circulares}.









% -------------------------------------

En cualquier caso, creo que sería posible actualizar y simplificar la
terminología, en base a los objetos matemáticos de los que surgen estos
conceptos. Es decir, se podría usar una terminología muy minimalista
haciendo uso de conceptos que ya conocemos de otras áreas de las
matemáticas.

Se me ocurre la siguiente terminología alternativa, tal y como se muestra en
la tabla \ref{tbl:term-nueva}.

\begin{landscape}
\begin{figure}%
  \caption{Propuesta de terminología alternativa para combinatoria}%
  \begin{center}
    \begin{tabular}[t]{| p{16em} | p{16em} | p{16em} |}
      \hline
      \multicolumn{1}{|c}{\textbf{Terminología nueva}} &
      \multicolumn{1}{|c}{\textbf{Term. clásica inglés}} &
      \multicolumn{1}{|c|}{\textbf{Term. clásica español}} \\
      \hline\hline
      lista &
      \emph{permutation with repetition (o total permutation with repetition)} &
      permutación total con repetición \\
      \hline
      sublista sin repeticiones &
      \emph{$k$- permutation (o partial permutation)} &
      variaciones (o permutación parcial) \\
      \hline
      sublista sin restricciones &
      \emph{$k$-permutation with repetition (o partial permutation with repetition)} &
      variación con repetición (o permutación parcial con repetición) \\
      \hline
      sublista con un esquema de repeticiones &
      \emph{$k$-permutation with TKTK (TK)} &
      variación con repetición con esquema TKTK (o permutación parcial con repetición TKTK) \\
      \hline
      sublista circular sin repetición &
      \emph{circular $k$-permutation} &
      permutación circular \\
      \hline
      \hline
      subconjunto &
      \emph{combination} &
      combinación (sin repetición) \\
      \hline
      submulticonjunto (sin restricciones) &
      \emph{combination with repetition} &
      combinación con repetición \\
      \hline
      submulticonjunto con un esquema de repeticiones &
      \emph{combination with repetitions TKTK} &
      combinación con repetición TKTK \\
      \hline
    \end{tabular}%
  \end{center}
\end{figure}%
\label{tbl:term-nueva}
\end{landscape}

Al contrario de lo que sucede con el concepto de \emph{lista} (es decir,
permutaciones totales, en la terminología antigua), el concepto de
\emph{conjunto} no merece la pena analizarlo, pues solo hay una forma de
disponer un conjunto basándonos en este mismo.






% ------------------------------------------

Entre los ejercicios de combinatoria, es bastante común que le pregunten por
el número de números distintos que se pueden formar siguiendo ciertas
reglas. A veces, la terminología que usan es equivocada. Por ejemplo, he
encontrado que llaman \emph{cifra} (\emph{figure}) a lo que deberían llamar \emph{dígito}
(\emph{digit}).

Me gustaría aclarar la terminología a este respecto. Un \emph{número}
(\emph{number}) es un concepto que TKTK.

Una representación de un número ---que, por cierto, no tiene por qué ser
única, tal y como sucede, por ejemplo, con los números racionales--- es un
\emph{numeral} (\emph{numeral}) o \emph{cifra} (\emph{figure}). Los sistemas
numerales son códigos en los que se tienen ``palabras'' que serían los
numerales. Los símbolos que constituyen esas palabras son los \emph{dígitos}
(\emph{digits}). Así, por ejemplo, en el número 15 se tienen los dígitos 1 y
5.

El sistema numeral estándar que solemos usar y que le han enseñado en la
educación primaria es el TKTK. Alcanzó popularidad en Occidente cuando lo
introdujo Leonardo de Pisa, más conocido como Fibonacci, TKTK.

Se trata de un sistema de numeración posicional ponderado. Más
concretamente, usa una ponderación polinomial.

Aunque se suele decir que es un sistema en base 10, en realidad con esto no
estoy diciendo nada, ya que para cualquier otro sistema numeral también
escribiría que es ``en base 10'', al no quedar claro a qué base nos
referimos. Así, por ejemplo, si la base sería el 6 en nuestro sistema
numeral estándar, al decirlo en un sistema en dicha base, también diríamos
que se trataría de un sistema en base 10. La única forma que se me ocurre de
comunicar esa base es pronunciándola en un sistema que no sea posicional.
Por ejemplo, en el que se conoce como sistema unario. Haciendo uso de este,
podríamos decir que la base del sistema de numeración estándar en el mundo
es

$$ 1111111111 $$




% ------------------------------------------



Una regla que usamos por comodidad es asumir que, cuando hablamos de
intervalos o números comprendidos entre otros dos dados, si no se especifica
lo contrario, se asumirá que los de los extremos se incluyen.








