



\section{Introducción}

En esta sección, se presentan varias fórmulas para ciertos tipos de
disposiciones que suelen aparecer. Sería como ciertas reglas ``enlatadas''
para ahorrarnos el uso de los principios del capítulo anterior; para usar un
atajo en la resolución. En cualquier caso, hay situaciones en las que no se
puede usar ninguna de estas reglas, con lo que conviene no perder de vista
el contenido de la sección anterior. De hecho, hay gente con bastante
soltura en combinatoria que deducen por sí mismos estas reglas cada vez que
las van a usar. Esto tampoco es una heroicidad, ya que se deducen fácilmente
a partir de los principios de la sección anterior.

Sobre los conceptos presntados en esta sección, existe cierta confusión en
la terminología. Personalmente, me adaptaré a la terminología en inglés y
seguiré la de \cite{brualdi}. En esta, llaman permutaciones
(\emph{permutations}) a las disposiciones en las que el orden es relevante,
incluyendo aquí a las permutaciones totales y a las parciales, es decir,
también llamaré \emph{permutaciones} a lo que en español suelen llamar
\emph{variaciones}, tal y como explico en la sección sobre la terminología de
esta parte.

Cuando el orden no es relevante, se tienen las combinaciones
(\emph{combinations}).

Además, se tienen algunas variantes dentro de estas, como, por ejemplo, las
permutaciones circulares. TKTK.

Todas estas en realidad no son más que las fórmulas que aparecerán según el
tipo de objeto matemático con el que representamos la situación. Tal y como
se explica en \cite{brualdi}, las permutaciones se usan cuando usamos listas
(también llamadas tuplas o sucesiones finitas), mientras que las
combinaciones se usan para conjuntos y multiconjuntos.






\section{Permutaciones y combinaciones}

\begin{deffinition}[Permutación Total]
  Sea $A$ un conjunto no vacío. Una \emph{permutación} (\emph{permutation}) de
  $A$ es una biyección de $A$ en $A$.
\end{deffinition}

Aclaramos que se dirá que dos permutaciones son diferentes cuando esas
biyecciones sean distintas.

Normalmente, se las llama \emph{permutaciones} (\emph{permutations}), sin
especificar más, pero, como puede haber confusión con el otro tipo de
permutaciones, a veces se las llama \emph{permutaciones totales} (\emph{total
permutations}).

Normalmente, la forma de presentar esa biyección es en la forma siguiente.
Suponga que tiene el conjunto

$$ A = \{a_1, a_2, a_3, \ldots, a_n\} $$

\noindent La permutación sería la aplicación biyectiva

\begin{equation*}
  \begin{array}{ll}
    \sigma: & A     \longrightarrow   A \\
            & a_i   \longmapsto       \sigma(a_i)
  \end{array}
\end{equation*}

\noindent siendo el índice $i \in \nset$ tal que $1 \leq i \leq n$. La
premutación se suele representar como

\begin{equation*}
  \sigma =
  \left(
  \begin{array}{ccccc}
      a_1         & a_2         & a_3         & \cdots  & a_n \\
      \sigma(a_1) & \sigma(a_2) & \sigma(a_3) & \cdots  & \sigma(a_n)
  \end{array}
  \right)
\end{equation*}

\noindent Se presentaría como una matriz de dos filas, estando en la primera
los elementos originales y, en la de debajo, sus elementos transformados
correspondientes.

Puesto que las permutaciones de $A$ son biyecciones de $A$ en $A$, tiene
sentido efectuar la composición de dos permutaciones consideradas como
aplicaciones. Llamaremos a esta operación \emph{producto} (\emph{product}) de
permutaciones.

\begin{deffinition}[Producto de Permutaciones]
  Sean $\sigma_1$ y $\sigma_2$ dos permutaciones del conjunto $A$.
  Llamaremos \emph{producto} (\emph{product}) de $\sigma_1$ y $\sigma_2$ y lo
  designaremos como $\sigma_2 \sigma_1$ a la permutación $\sigma$ de $A$ tal
  que

  \begin{equation*}
    \begin{array}{ll}
      \sigma = \sigma_2 \sigma_1:
        & A   \longrightarrow A \\
        & a   \longmapsto     \sigma(a) = \sigma_2\sigma_1(a) =
          \sigma_2(\sigma_1(a))
    \end{array}
  \end{equation*}
\end{deffinition}

El conjunto de todas las permutaciones de un conjunto $A$ con la operación
producto (es decir, esta operación, que es la composición) tiene estructura
algebraica de grupo y se denomina \emph{grupo simétrico} de $A$. Se designa
habitualmente por $S_A$. Se estudia bastante en el álgebra abstracta este
grupo.

Como particularidad, si $A = \{1, 2, 3, \ldots, n\}$  para $n \in
\nset^{+}$, al grupo simétrico se le designará normalmente por $S_n$.

\begin{theorem}
  Sean dos conjuntos $A$ y $B$ con el mismo número $n \in \nset^{+}$ de
  elementos. El número de biyecciones distintas de $A$ a $B$ es $n \cdot
  (n-1) \cdot (n-2) \cdots \cdot 2 \cdot 1$.
\end{theorem}

\begin{proof}
  Sean $A = \{a_1, a_2, a_3, \ldots, a_n\}$ y $B = \{b_1, b_2, b_3, \ldots,
  b_n\}$. Para construir una biyección $f: A \longrightarrow B$, debemos
  indicar la imagen de cada $a_i \in A$. Recuerde que, para que la relación
  sea una aplicación, cada elemento de $A$ ha de estar relacionado con
  exactamente uno de $B$.

  Consideremos el primer elemento $a_1$ de $A$, aunque podríamos haber
  seguido cualquier otro orden. Podemos tomar como $f(a_1)$ a cualquiera de
  los $n$ elementos de $B$. Una vez fijado $f(a_1)$, como la aplicación es
  una biyección y, por tanto, es inyectiva, disponemos de de $n-1$ elementos
  distintos en $B$ para la imagen del siguiente elemento de $A$, que, con el
  orden que estamos siguiendo, se trata de $a_2$. Continuando el proceso,
  tendremos dos elementos para la imagen de $a_{n-1}$ y uno solo para la
  imagen de $a_n$. Por el Principio del Producto, el número total de
  biyecciones distintas entre $A$ y $B$ será

  $$ n \cdot (n-1) \cdot (n-2) \cdot \cdots \cdot 2 \cdot 1 $$
\end{proof}

A la función que produce el producto de los $n \in \nset^{+}$ primeros
números naturales positivos se le conoce como el \emph{factorial}
(\emph{factorial}) de $n$, y se designa por $n!$. Es algo extraño que se use
un signo de exclamación para representar a una función, pero esa es la
costumbre en este caso.

Por convenio, se toma $0! = 1$. Esto facilita algunos casos extraños. No
tiene sentido tratar de hallar el factorial de un número negativo ni de uno
con parte fraccionaria. A veces, el factorial se define en forma recursiva.
Dicha definición sería la siguiente:

\begin{equation*}
  n! = \left\{
  \begin{array}{ll}
    1               & \text{si} \ n = 0 \\
    (n - 1)! \, n   & \text{si} \ n \neq 0
  \end{array}
  \right.
\end{equation*}

\noindent cosa que se deduce muy fácilmente a partir de su forma explícita.
Como es evidente por la definición de \emph{permutación}, para el teorema
anterior, si se da que $A = B$, se tiene que las biyecciones son
permutaciones de $A$. Se tiene, por tanto, el corolario siguiente:

\begin{corollary}
  El número de permutaciones de un conjunto de $n \in \nset^{+}$ elementos,
  que designaremos por $P(n)$, es $n!$. Así, $\card(S_n) = n!$.
\end{corollary}

Ahora, nos podríamos plantear generalizar el concepto de permutación. A esto
aquí lo llamaremos \emph{permutación} de orden $k$ de un conjunto o una
$k$-permutación del conjunto $A$. En español, tal y como dijimos, es más
común llamarlo \emph{variación}. En inglés, lo llaman $k$\emph{-permutation}.
Se podría decir, si se quiere ser explícito, \emph{permutación parcial}
(\emph{partial permutation}). En ese caso, las que vimos antes serían,
entonces, \emph{permutaciones totales} (\emph{total permutations}). Si lo
piensa, representan simplemente a \emph{listas} (\emph{lists}), también
llamadas \emph{tuplas} (\emph{tuples}) o, si lo prefiere, sucesiones finitas
(\emph{finite sequences}), por lo que no encuentro una razón por la que no
pudiesen recibir estos últimos nombres.

A este respecto, conviene recordar una frase célebre del matemático Henri
Poincare:

\begin{quote}
  Las matemáticas consisten en llamar con el mismo nombre a cosas
  diferentes.
\end{quote}

\begin{deffinition}[Permutación Parcial]
  Sea $A$ un conjunto finito con $n \in \nset^{+}$ elementos y un número $r
  \in \nset^{+}$ tal que $r \leq n$. Una \emph{permutacion parcial}
  (\emph{partial permutation}) de orden $r$ de $A$ es cualquier lista
  ordenada $(a_1, a_2, a_3, \ldots, a_r)$ de $r$ elementos de $A$ distintos
  entre sí.
\end{deffinition}

Como consecuencia de esta definición, diremos que dos permutaciones son
diferentes si algún elemento de una de las dos listas no se encuentra en la
otra o bien si las dos listas contienen los mismos elementos pero no
exactamente en el mismo orden.

Normalmente, lo de \emph{parcial} o \emph{total} se deja sin especificar y se
sobrendiende por el orden $r$ y el número $n$. Será total si $r = n$ y
parcial en caso contrario.

Advierta que se trata de elementos distintos. Existen también permutaciones
en las que se permite la repetición de elementos, como veremos luego.

Si se fija, esto no es más que las listas de $r$ elementos no repetidos que
se pueden formar a partir de un conjunto de $n$ elementos. Por tanto, si lo
prefiere, puede llamarlas \emph{listas sin repetición}. Quizás, se podría
hablar de sublistas sin repetición de $r$ elementos TKTK.

Designaremos el número de permutaciones de orden $r$ del conjunto $A$ con $n$
elementos como $V(n, r)$, aunque hay quienes usan otras formas de
designarlo, como $V_n^r$ o, en inglés, $P(n, r)$ o $P_r^n$. También se dice
``permutaciones de $n$ elementos tomados de $r$ en $r$'', en lugar de decir
``permutaciones de $n$ elementos de orden $r$''.

Si $(a_1, a_2, a_3, \ldots, a_r)$ es una permutación de orden $r$ de $A$,
entonces podemos asociarle la aplicación inyectiva

\begin{equation*}
  \begin{array}{lrcl}
    \sigma:   & \{1, 2, 3, \ldots, r\}  & \longrightarrow & A     \\
              & i                       & \longmapsto     & a_i = \sigma(i)
  \end{array}
\end{equation*}

\noindent siendo $i \in \nset$ con $1 \leq i \leq r$. Será inyectiva porque
no se permite la repetición de elementos. También, a toda aplicación
inyectiva

$$ \sigma: \{1, 2, 3, \ldots, r\} \longrightarrow A $$

\noindent le podemos asignar la permutación $(\sigma(1), \sigma(2),
\sigma(3), \ldots, \sigma(r))$ de $r$ elementos no repetidos de un conjunto
$A$.

En realidad, ya que, como hemos explicado, se trata de listas, y estas se
definen a partir de aplicaciones TKTK. Por lo tanto, la razón de que se
cumplan ambas cosas es que en realidad todo viene de lo mismo: de la
definición de las listas como un tipo de aplicación inyectiva.

Por lo tanto, es equivalente definir una permutación de $A$ de orden $r$
tanto como una lista ordenada $(a_1, a_2, a_3, \ldots, a_r)$ de elementos no
repetidos de $A$ como una aplicación inyectiva del tipo

$$ \sigma: \{1, 2, 3, \ldots, r\} \longrightarrow A $$

\noindent Por esto, se tiene que las permutaciones (totales) del conjunto
$A$ pueden ser consideradas un caso particular de permutaciones (parciales)
de orden $r$ de $A$. Concretamente, las permutaciones (parciales) de $A$ de
orden $\card(A)$; con los valores de estos ejemplos, sería $r = n$.

\begin{theorem}
  Sean $A$ y $B$ dos conjuntos no vacíos con $\card(A) = r$, $\card(B) = n$
  y $r \leq n$. El número de aplicaciones inyectivas de $A$ a $B$ es

  $$ n(n - 1)(n - 2) \cdots (n - r + 1) $$

  \noindent o, lo que es lo mismo,

  $$ \frac{n!}{(n - r)!} $$
\end{theorem}

\begin{proof}
  Sean $A = \{a_1, a_2, a_3, \ldots, a_r\}$ y $B = \{b_1, b_2, b_3, \ldots,
  b_n\}$ para $r, n \in \nset^{+}$ siendo $1 \leq r \leq n$. Veamos las
  posibilidades para formar una aplicación inyectiva de $A$ a $B$.

  Para la imagen de $a_1$ disponemos de $n$ elementos diferentes de $B$,
  aunque se podría haber seguido otro orden y no empezar por $a_1$. Una vez
  elegida esta imagen, disponemos de $n - 1$ elementos diferentes de $B$
  para usar como imagen de $a_2$, pues, al ser inyectiva, dos elementos
  distintos de $A$ no pueden tener una misma imagen en $B$. Continuando con
  este proceso, si se fija, para un caso genérico en el que tengamos el
  elemento $a_i$ de $A$, se tendrán $n - (i - 1)$ posibilidades de elección
  de la imagen en $B$. 

  Así, llegamos a los últimos elementos a los que asignar una imagen.
  Llegamos al elemento $a_r$, para el que disponemos de $n - (r - 1)$
  elementos distintos de $B$, o, lo que es lo mismo, $n - r + 1$.

  Por el Principio del Producto, podemos formar

  $$ n(n - 1)(n - 2) \cdots (n - r + 1) $$

  \noindent aplicaciones inyectivas de $A$ a $B$ diferentes. Si se fija, esa
  fórmula se puede expresar también como

  $$ \frac{n!}{(n - r)!} $$
\end{proof}

Aunque

$$ n(n - 1)(n - 2) \cdots (n - r + 1) $$

\noindent y

$$ \frac{n!}{(n - r)!} $$

\noindent sean lo mismo, a la hora de calcular con valores concretos, puede
que la segunda forma le sea imposible ya que no tiene medios para calcular
un número tan enormemente grande, si $n$ es bastante grande. Como es
evidente, puede cancelar factores y volverá a la primera de las formas de
presentarla, con lo que quizás sí pueda hacer el cálculo.

\begin{corollary}
  Sean $A$ un conjunto finito con $n$ elementos siendo $n \in \nset^{+}$, y
  un número $r \in \nset^{+}$ tal que $r \leq n$. Entonces el número de
  permutaciones de orden $r$ de $A$, $P(n, r)$, es

  $$ P(n, r) = \frac{n!}{(n - r)!} $$
\end{corollary}

\begin{deffinition}[Permutaciones con Repetición]
  Sea $a_1, a_2, a_3, \ldots, a_r$ un conjunto de $r$ elementos de un
  conjunto $A$ finito de $n \in \nset^{+}$ elementos. Una \emph{permutación con
  repetición} de orden $r$ de $A$ es cualquiera de las posibles listas
  ordenadas $(a_1, a_2, a_3, \ldots, a_r)$ de tamaño $r$ de elementos de $A$
  en la que se permite la repetición de elementos.
\end{deffinition}

Como consecuencia de la definición anterior, diremos que dos permutaciones
con repetición son diferentes si algún elemento de una de las dos listas no
se encuentra en la otra o bien si las dos listas contienen los mismos
elementos en un orden diferente.

Como es evidente, se trata de permutaciones parciales con repetición de $n$
elementos de orden $r$. El caso de las permutaciones totales con repetición
sería también un caso particular de estas; concretamente, en el que $r = n$.

Designaremos al número de permutaciones con repetición de orden $r$ del
conjunto $A$ con $n$ elementos por $PR(n, r)$, pero puede tomar otras muchas
designaciones. Si las llama \emph{permutaciones}, se designaría por $PR(n, r)$.

Además, existe otra forma de llamar a las permutaciones con repetición de $n$
elementos de orden $r$. Verá que también se dice ``permutaciones con
repetición de $n$ elementos de $r$ en $r$''.

Advierta que aquí, al contrario de las permutaciones (sin repetición), no
imponemos la restricción de que el orden sea menor o igual que el tamaño del
conjunto sobre el que se toman estas, ya que no es necesario imponer esa
restricción.

Si $(a_1, a_2, a_3, \ldots, a_r)$ es una permutación con repetición de orden
$r$ de $A$, siendo $r \in \nset^{+}$, entonces podemos asociarle la
aplicación

\begin{equation*}
  \begin{array}{lrcl}
    \sigma:   & \{1, 2, 3, \ldots, r\}  & \longrightarrow & A     \\
              & i                       & \longmapsto     & \sigma(i) = a_i
  \end{array}
\end{equation*}

\noindent para $i \in \nset^{+}$ tal que $1 \leq i \leq r$. Y viceversa, es
decir, a cada aplicación $\sigma: \{1, 2, 3, \ldots, r\} \longrightarrow A$
le podemos asociar la permutación con repetición de orden $r$, $(\sigma(1),
\sigma(2), \sigma(3), \ldots, \sigma(r))$.

Por lo tanto, una permutación con repetición de orden $r$ de $A$ se puede
definir tanto como una lista de longitud $r$ de elementos de $A$ como por
una aplicación como la que acabamos de ver. De hecho, en realidad, siendo
rigurosos, esa aplicación es la definición del concepto de \emph{lista} o
$n$\emph{-tupla}; en este caso, $r$-tupla, pues es de tamaño $r$.

Una curiosidad de la que quizás se haya dado cuenta en algún ejercicio es
que $P(n, n-1) = P(n, n)$. Esto, si lo analizamos desde la combinatoria,
tiene sentido claramente, pues al hacer $n-1$ ordenaciones de $n$ elementos
siempre estoy dejando uno fuera. Por tanto, si hiciese $n$ ordenaciones de
$n$ elementos, serían las mismas que antes solo que poniendo ese elemento
que sobraba, pero el número de permutaciones (parciales o totales) en este
caso sería el mismo. Por ejemplo, esto se ve en los dos primeros ejemplos de
la página 36 de \cite{brualdi}.

\begin{deffinition}
  Sean $A$ y $B$ dos conjuntos no vacíos con $\card(A) = r$ y $\card(B) =
  n$. Entonces el número de aplicaciones con conjunto inicial $A$ y conjunto
  final $B$ es

  $$ n^r $$
\end{deffinition}

\begin{proof}
  Como vemos, se trataría de ver el número de aplicaiones, sin más
  restricciones. Al contrario de antes, esta no tiene por qué ser inyectiva.

  Sean $A = \{a_1, a_2, a_3, \ldots, a_r\}$ y $B = \{b_1, b_2, b_3, \ldots,
  b_n\}$ siendo $r \geq 1$. Para la imagen de $a_1$, disponemos de $n$
  elementos diferentes de $B$. Una vez elegida esta imagen, seguimos
  disponiendo de $n$ elementos para la imagen de $a_2$, y así sucesivamente.
  Luego, por el Principio del Producto, podemos formar

  $$ \prod_{i=1}^r n = n^r $$

  \noindent aplicaciones $f: A \longrightarrow B$.
\end{proof}

\begin{corollary}
  Sea $A$ un conjunto finito con $n \in \nset^{+}$ elementos y un número $r
  \in \nset^{+}$. El número de permutaciones con repetición de orden $r$ de
  $A$ es

  $$ PR(n, r) = n^r $$
\end{corollary}

Antes de seguir, veamos un ejemplo muy típico. El número de ``palabras''
diferentes de longitud 8 que se pueden formar en código binario, es decir,
código basado en un alfabeto de solo dos símbolos diferentes, que suelen ser
$\{0, 1\}$, vendría dado por las permutaciones con repetición de 2 elementos
de orden 8, que son

$$ PR(2, 8) = 2^8 = 256 $$

\noindent Advierta que, en este caso, $r > n$, lo cual no es ningún problema
cuando se trata de permutaciones con repetición.

Como seguramente sepa, la información que manejan internamente los
computadores se encuentra codificada en código binario. A los dígitos de
este código seles conoce como \emph{bit}, que es la acortación del sintagma
\emph{binary digit}, es decir, \emph{dígito binario}. Se les suele representar
por una \emph{b} minúscula: b. Las agrupaciones de palabras de 8 bits cobran
una relevancia especial en los computadores y suelen recibir en la
actualidad el nombre de \emph{byte},\footnotemark y se suelen representar
por una \emph{b} mayúscula: B.

\footnotetext{aunque quizás sea más correcto llamarlas \emph{octetos}, pues en
otras épocas el \emph{byte} tuvo otros tamaños distintos al que tiene ahora}

\begin{deffinition}[Combinación]
  Sean $A$ un conjunto finito con $n \in \nset^{+}$ elementos y un número $r
  \in \nset^{+}$ tal que $r \leq n$. Una \emph{combinación}
  (\emph{combination}) de $A$ de orden $r$ es un subconjunto de $A$ con $r$
  elementos.

  Como conjunto que es la combinación, no cuenta con elementos repetidos y
  el orden de sus elementos es indiferente, es decir, dos permutaciones de
  una combinación representarían a la misma combinación.

  Diremos, por tanto, que dos combinaciones son diferentes si algún elemento
  de una lista no se encuentra en la otra.
\end{deffinition}

Designaremos al número de combinaciones de $n$ elementos de orden $r$ por
$C(n, r)$, o, también, por

$$ {n \choose r} $$

\noindent A esta última expresión se la conoce como \emph{coeficiente binómico}
(\emph{binomial coefficient}) o, también, como \emph{número combinatorio}
(\emph{combinatorial number}).

También, verá que hay quien dice ``número de combinaciones de $n$ elementos
tomados de $r$ en $r$'', en lugar de ``[\ldots] de orden $r$''.

Algo de lo que es fácil darse cuenta es de que, tal y como se muestra en el
ejemplo 3-2.27, se puede establecer una fórmula para calcular las
combinaciones de $n$ elementos de orden $r$. Así, si, por ejemplo, tenemos
el conjunto

$$ S = \{1, 2, 3, 4\} $$

\noindent se puede ver, por fuerza bruta, que las combinaciones de orden 3
sobre ese conjunto son 4. Si a cada una de esas combinaciones les calculamos
sus permutaciones, vemos que son $P(3) = 3! = 6$. Aplicando el Principio del
Producto, dicha multiplicación nos dará las permutaciones de 4 elementos de
orden 3, es decir, $P(4, 3)$. Esa es la fórmula que podemos establecer y que
luego veremos para el caso general:

$$ C(4, 3) \cdot P(3) = P(4, 3) $$

\noindent o, lo que es lo mismo,

$$ C(4, 3) = \frac{P(4, 3)}{P(3)} $$

\begin{theorem}
  Para todo $n, r \in \nset^{+}$ siendo $1 \leq r \leq n$, se cumple

  $$ P(n, r) = P(r) \cdot C(n, r) $$

  \noindent Por lo tanto,

  $$ C(n, r) = \frac{P(n, r)}{P(r)} $$

  \noindent o, lo que es lo mismo

  $$ C(n, r) = \frac{n!}{(n-r)! \, r!} $$
\end{theorem}

\begin{proof}
  Tenemos que $C(n, r)$ es el número de formas de eleccionar $r$ objetos de
  un conjunto de $n$ objetos sin tener en cuenta el orden. Cada una de estas
  combinaciones se puede ordenar de $P(r) = r!$ formas diferentes, si ahora
  sí se desea tener en cuenta el orden. Por tanto, por el Principio del
  Producto, se tiene

  $$ P(n, r) = r! \, C(n, r) $$

  \noindent y, de aquí,

  $$ C(n, r) = \frac{P(n, r)}{P(r)} = \frac{n!}{(n-r)! \, r!} $$
\end{proof}

Respecto a los coeficientes binómicos, se dan ciertos casos particulares que
conviene conocer. Así, se tiene que

$$ {n \choose 0} = 1 $$

Se puede demostrar simplemente sustituyendo la expresión por la fracción y
teniendo en cuenta que, tal y como dijimos, se tiene que $0! = 1$.

Para todos estos casos particulares que veamos, recuerde que el coeficiente
binómico refleja las combinaciones. Así, para el caso anterior, se podría
expresar como que $C(n, 0) = 1$, y sobre esto también se podría hacer una
demostraación combinatoria. En esta, tendría que pensar en que existe una
única forma de seleccionar 0 elementos de un conjunto de $n$.

Otro caso particular interesante es

$$ {n \choose n} = 1 $$

Al igual que antes, se podría demostrar tanto por su expresión algebraica
como mediante una demostración combinatoria. Esta consistiría en pensar que
solo existe un subconjunto de $n$ elementos de un conjunto de $n$ elementos.

Algo en lo que debemos fijarnos también es lo siguiente:

$$ {n \choose k} = 0 \ \text{si} \ k >n $$

\noindent ya que no tiene sentido tomar un factorial de un número negativo.
En cuanto a la demostración combinatoria, podría pensar que existen 0
subconjuntos de $k$ elementos en un conjunto de $n$ elementos si $k > n$.
Otra forma de demostrarlo sería fijarse que, en el desarrollo de la
expresión,

$$ {n \choose k} = \frac{n!}{(n-k)! k!} = n(n-1)(n-2) \cdots (n - k + 1)
\quad \text{para} \ k > n $$

\noindent al darse la condición $k > n$, en algún momento se tendrá uno de
esos factores que valga 0, y, como sabe, basta con que uno de los factores
valga 0 para que el producto sea 0.

Quizás, se podría considerar que estamos modelando el mundo y lo más
conveniente sería decidir estos resultados para estos casos.

En el capítulo siguiente sen ve en mayor profundidad las propiedades de los
coeficientes binómicos.




Hay otros casos de particular importancia sobre los coeficientes binómicos.
TKTK.

Si se fija, estamos usando dos designadores para lo mismo. Tanto $C(n, r)$
como

$$ {n \choose r} $$

\noindent indican exactamente lo mismo, de ahí que haya textos en los que
prescinden de la primera de estas notaciones.

El ejemplo 3-2.29, que consiste en demostrar que el número de subconjuntos
de un conjunto $A$ arbitrario viene dado por la igualdad

$$ \card(\powset(A)) = 2^{\card(A)} $$

\noindent se podría explicar ahora de una forma algo más elegante, sin
necesidad de recurrir cosas que se vieron antes.

Cada $C(n, r)$ indica también los subconjuntos de tamaño $r$ en un conjunto
de $n$ elementos. Por tanto, aplicando el Principio de la Suma, el sumatorio
indicará $\card(\powset(A))$, si $A$ es el conjunto de tamaño $n$:
$\card(A)$.

Esto se puede ver también desde el punto de vista de las permutaciones con
repetición, como si se tratase de un código binario. Suponga que

$$ A = \{a_1, a_2, a_3, \ldots, a_n\} $$

Para todo elemento $a_i$ de $A$, creamos un código binario que se
corresponde, en una aplicación biyectiva, con un subconjunto de $A$. Por
ejemplo, el subconjunto $\{a_1, a_2\}$ se correspondería con el código
11000\ldots 0. Así, bastaría con calcular las permutaciones con repetición
de $n$ elementos de orden 2, es decir,

$$ PR(n, 2) = 2^n $$

\begin{deffinition}[Combinaciones con Repeticiones Sin Restricciones]
  Sean $A$ un conjunto finito con $n \in \nset^{+}$ elementos y un número $r
  \in \nset^{+}$. Una \emph{combinación con repetición} de orden $r$ de $A$ es
  un multiconjunto (\emph{multiset}) de $r$ elementos de $A$.

  Como multiconjunto que es, el orden no es relevante pero la multiplicidad
  de sus elementos sí. Es decir, diremos que dos combinaciones con
  repetición son diferentes si algún elemento de una de las dos listas no se
  encuentra en la otra o si uno aparece repetido más veces en un
  multiconjunto que en el otro.
\end{deffinition}

Designaremos al número de combinaciones con repetición de orden $r$ del
conjunto $A$ de tamaño $n$ por $CR(n, r)$. También se dice ``combinaciones
con repetición de $n$ elementos tomados de $r$ en $r$''.

Como veremos, el problema de encontrar el número de combinaciones con
repetición de orden $r$ de un conjunto $A$ de $n$ elementos es equivalente
al problema de encontrar el número de soluciones naturales de la ecuación
diofántica

$$ x_1 + x_2 + x_3 + \cdots + x_n = r $$

\noindent Puesto que se trata, tal y como hemos dicho, de una ecuación con
soluciones naturales, esta podrá adoptar dos formas. La anterior, que es la
comprimida, y su forma expandida, en la que se tiene una suma de unos:

$$ \underbrace{1 + 1 + \cdots + 1}_\text{$x_1$ veces} + \underbrace{1 + 1 +
\cdots + 1}_\text{$x_2$ veces} + \cdots + \underbrace{1 + 1 + \cdots +
1}_\text{$x_n$ veces} = r $$

\noindent siendo $x_i \in \nset^{+}$ para todo $i \in \nset$ tal que $1 \leq
i \leq n$. Si se fija, ambas expresiones representan a la misma ecuación.

Por tanto, el multiconjunto $\{x_1, x_2, x_3, \ldots, x_n\}$ será una
solución de la ecuación

$$ x_1 + x_2 + x_3 + \cdots + x_n = r $$

\noindent Y viceversa. La solución natural de esta ecuación es el
multiconjunto $\{x_1, x_2, x_3, \ldots, x_n\}$.

\begin{theorem}
  Dados $k, n \in \nset^{+}$. El número de soluciones naturales de la
  ecuación

  $$ x_1 + x_2 + x_3 + \cdots + x_n = k $$

  \noindent es $C(n + k - 1, k)$.
\end{theorem}

\begin{proof}
  Lo que se hará para demostrarlo es usar una cadena de $k$ unos a la que
  dividiremos con $n - 1$ barras. Una vez hecho esto, haremos uso de nuestro
  conocimiento de las combinaciones para calcular el número de situaciones
  posibles que se pueden presentar.

  Si se fija, todas las posibilidades que obtenemos así marcarán todas las
  soluciones de la ecuación, ya que todos esos unos suman $k$ y las barras
  marcan las agrupaciones en las distintas variables $x_i$ para $i \in
  \nset^{+}$ siendo $1 \leq i \leq n$. Y, recíprocamente, toda solución de
  la ecuación se puede representar de esa forma como cadena de $k$ unos
  agrupados con $n - 1$ barras, o, lo que es lo mismo, en $n$ agrupaciones.

  Como caso particular, advierta que dos o mas barras pueden aparecer
  juntas. Esto está permitido. Simplemente, marcaría que TKTK.

  Consideramos el conjunto de los $k$ unos y las $n-1$ barras como un todo.
  Este tiene $k + (n - 1)$ elementos. Por tanto, la solución sería hallar
  las distintas formas de poner esas $n-1$ fronteras entre todos los $k + (n
  - 1)$ elementos. En esas disposiciones no influye el orden ya que se
  cumple la propiedad conmutativa de la suma. Además, se admiten
  repeticiones, es decir, que dos o más barras puedan estar juntas. Por
  tanto, se trata de combinaciones (sin repetición) de $k + (n - 1)$
  elementos de orden $n-1$. Eso es $C(n + k - 1, n-1)$, o, lo que es lo
  mismo,

  $$ {n + k - 1 \choose n-1} $$

  Se podría haber visto desde otro punto de vista. Sobre el mismo conjunto
  de antes, con $k + (n-1)$ elementos, se toman las distintas disposiciones
  de $k$ elementos, es decir, de $k$ unos. Esto nos daría $C(n + k - 1, k)$,
  o, lo que es lo mismo,

  $$ {n + k - 1 \choose k} $$

  \noindent que da el mismo resultado que la expresión anterior. Esto es un
  resultado general de los coeficientes binómicos y se justificará en la
  sección siguiente.
\end{proof}

Lo que aquí llamamo \emph{combinaciones con repetición} en realidad son un caso
particular de estas. Concretamente, en las que las repeticiones no tienen
restricción, o que son infinitas, se podría decir, aunque siempre están
limitadas por el número total de TKTK. Se tienen, por tanto, también las
combinaciones con repetición con un esquema determinado de repeticiones, que
es lo que se ve en el capítulo siguiente en la sección de coeficientes
multinómicos. Esto, en \cite{brualdi} no se explica tan adelante y la
demostración de la fórmula de las combinaciones con repetición sin
restricciones hace uso de la otra, es decir, de la de las combinaciones con
repetición siguiendo un esquema de repetición; concretamente, en la pág. 52.

\begin{corollary}
  El número de combinaciones con repetición de orden $k$ de un conjunto de
  $n$ elementos es

  $$ CR(n, k) = C(n + k - 1, k) $$

  \noindent o, lo que es lo mismo,

  $$ {n + k - 1 \choose k} $$
\end{corollary}

\begin{deffinition}[Permutación Circular]
  Una \emph{permutación circular} (\emph{circular permutation}) de $n$ objetos
  de orden $r$ con $r \leq n$, cosa que designamos por $Q(n, r)$ o $Q_r^n$,
  es lo mismo que una permutación de $n$ elementos de orden $r$ solo que se
  consideran como una misma disposición a dos permutaciones tales que una
  sea una traslación de los elementos de la otra en alguno de los dos
  sentidos, de cualquier magnitud, considerando el pliegue (\emph{folding})
  de los elementos, es decir, si se sale de las posiciones, se vuelve por el
  otro extremo.
\end{deffinition}

Por tanto, se trata de algo muy parecido a las permutaciones.

Por eso creo que el llamarlas \emph{permutaciones} está mal. Es como si ahora
hubieran cambiado la terminología y pasasen a usar la equivalente al inglés,
llamando \emph{permutación} de $n$ elementos de orden $r$ a lo que llamaban
\emph{variación} de $n$ elementos de orden $r$.

Muchas veces, para expresar este concepto, se hace alusión a la disposición
de los elementos en una circunferencia y a rotaciones, pero estos no tienen
nada intrínseco con forma de circunferencia. Simplemente, es una forma
sencilla de explicarlo.

Se tiene una fórmula para $Q(n, r)$ que se puede obtener fácilmente. Basta
con tener en cuenta que si, para cada una de estas disposiciones, la
multiplico por los distintos puntos de partida de las posibles traslaciones,
es decir, por $r$, tengo el número de permutaciones de $n$ elementos de
orden $r$. Por tanto,

\begin{theorem}
  El número de permutaciones circulares de $n$ objetos distintos de orden
  $r$ siendo $r \leq n$, $Q(n, r)$, es

  $$ Q(n, r) = \frac{P(n, r)}{r} = C(n, r) \cdot (r-1)! $$
\end{theorem}

La demostración de la fórmula para las permutaciones la hicimos antes. La de
las combinaciones se deduce simplemente operando.

$$ Q(n, r) = \frac{P(n, r)}{r} = \frac{n!}{(n-r)! r} = \frac{n!}{(n-r)! r!}
\, (r-1)! = {n \choose r} \cdot (r-1)! = C(n, r) \cdot (r-1)! $$

















